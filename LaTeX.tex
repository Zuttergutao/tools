\documentclass[a4paper,twoside,12pt]{article}
% \documentclass[twoside,a4paper]{article}

\usepackage{ctex}
\usepackage{fontspec}
\usepackage[T1]{fontenc}
\usepackage[table]{xcolor} % 调用颜色宏包
\usepackage{setspace}
\usepackage{amssymb, amsfonts, eufrak}
\usepackage{soul}
\usepackage{lscape}
\usepackage{sectsty}
\usepackage{listings}
\usepackage{ulem}
\usepackage{pifont}
\usepackage{enumitem}
\usepackage{multirow} % 合并单元格
\usepackage{threeparttable} % 添加表注
\usepackage{graphicx} % 修改表格尺寸
\usepackage{wrapfig}  % 文字环绕表格
\usepackage{longtable} % 调用长表格
\usepackage{rotating}  % 旋转表格
\usepackage{caption}
\usepackage{subcaption} % 插入子图
\usepackage{tikz}  % 图形包
\usepackage{algorithm}
\usepackage{algorithmic}
\usetikzlibrary{positioning,arrows.meta,quotes}
\usetikzlibrary{shapes,snakes}
\usetikzlibrary{bayesnet}
\usepackage{tabularx}
\usepackage{tabulary}
\usepackage{dcolumn} % 小数点对齐
\usepackage{booktabs}
\usepackage{diagbox}
\usepackage[namelimits]{amsmath}
\usepackage[top=25mm,bottom=20mm,margin=25mm]{geometry} % 页边距宏包需要在fancyhdr之前引入,否则页面设置不合理
\usepackage{fancyhdr} % 页眉页脚脚注
\usepackage[nottoc]{tocbibind}
\pagestyle{fancy}
\lhead{}
\chead{}
\rhead{\bfseries \LaTeX -Notes } %页眉内容
\lfoot{Author: Casea} %页脚内容
\cfoot{} %页脚内容
\rfoot{\thepage} %在页脚处给出页码
\renewcommand{\headrulewidth}{0.4pt}
\renewcommand{\footrulewidth}{0.4pt}
\setlength\headwidth{\textwidth}
\usepackage[colorlinks=true,linkcolor=black]{hyperref} % 设置目录跳转

% \setlength\columnsep{1cm} %设置分栏间距
\definecolor{kugreen}{RGB}{50, 93, 61}
\definecolor{kugreenlys}{RGB}{132, 158, 139}
\definecolor{kugreenlyslys}{RGB}{173, 190, 177}
\definecolor{kugreenlyslyslys}{RGB}{214, 223, 216}
\renewcommand{\abstractname}{\Huge 摘要\\}
\lstset{ %
    language=TeX,
    basicstyle=\footnotesize,       % the size of the fonts that are used for the code
    numbers=none,                   % where to put the line-numbers
    numberstyle=\tiny\color{gray},  % the style that is used for the line-numbers
    stepnumber=1,                   % the step between two line-numbers. If it's 1, each line 
                                    % will be numbered
    numbersep=5pt,                  % how far the line-numbers are from the code
    backgroundcolor=\color{white},      % choose the background color. You must add \usepackage{color}
    showspaces=false,               % show spaces adding particular underscores
    showstringspaces=false,         % underline spaces within strings
    showtabs=false,                 % show tabs within strings adding particular underscores
    frame=single,                   % adds a frame around the code
    rulecolor=\color{black},        % if not set, the frame-color may be changed on line-breaks within not-black text (e.g. commens (green here))
    tabsize=2,                      % sets default tabsize to 2 spaces
    captionpos=b,                   % sets the caption-position to bottom
    breaklines=true,                % sets automatic line breaking
    breakatwhitespace=false,        % sets if automatic breaks should only happen at whitespace
    title=\lstname,                   % show the filename of files included with \lstinputlisting;
                                    % also try caption instead of title
    % keywordstyle=\color{blue},          % keyword style
    commentstyle=\color{teal},       % comment style
    stringstyle=\color{mauve},         % string literal style
    escapeinside={\%*}{*)},            % if you want to add LaTeX within your code
    morekeywords={*,...}               % if you want to add more keywords to the set
}
% \usepackage{sectsty} %文档标题字体格式
%\sectionfont{\fontfamily{phv}\fontseries{b}\fontsize{11pt}{20pt}\selectfont} %一级标题字体格式设置
%\subsectionfont{\fontfamily{phv}\fontseries{b}\fontsize{11pt}{20pt}\selectfont} %二级标题字体格式设置
%\subsubsectionfont{\fontfamily{phv}\fontseries{b}\fontsize{11pt}{20pt}\selectfont} %三级标题字体格式设置
% \sectionfont{\centering} % 标题居中
\title{\Huge\LaTeX  \, Learning \\ Notes  }
\author{Casea}
\date{2022/10/20}
\begin{document}
\maketitle
\newpage
\begin{abstract}
\vspace{1cm}
\noindent
{\large This is a Latex learning code!\\
Following \href{https://nbviewer.org/github/xinychen/latex-cookbook/blob/main/chapter-3/section7.ipynb}{\textcolor{black}{\uline{Chen's github repository}}} to learn how to write elegant \LaTeX.}
\end{abstract}
\newpage


\tableofcontents

\newpage



\part{代码结构}
\section{代码结构}
\subsection{列表}
%%%%%%%%%%%%%%%%%%%%%%%%%%%%%%%%%%%%%%%%%%%%%%%%%%%%%%%%%%%%%%%%%%%%%%%%%%%

\fbox{
    \parbox{0.8\linewidth}{
    比较常用的三种文档类型,包括article(常规文档)、report(报告)、book(书籍),
其中,report和book这两种文档类型的文档结构是一致的,
可以使用的结构命令有\textbackslash part\{\}、\textbackslash chapter\{\}、\textbackslash section\{\}、\textbackslash subsection\{\}、\textbackslash subsubsection\{\}、\textbackslash paragraph\{\}、\textbackslash subparagraph\{\},
而article文档类型中除了没有\textbackslash chapter\{\}这一结构命令之外,其他都与report和book文档类型是一样的。\copyright\
    }
}

\vspace{1em}
\begin{lstlisting}
\begin{itemize}
\item LaTeX is good
\item LaTeX is convenient
\end{itemize}
\end{lstlisting}


\subsection{换行}
%%%%%%%%%%%%%%%%%%%%%%%%%%%%%%%%%%%%%%%%%%%%%%%%%%%%%%%%%%%%%%%%%%%%%%%%%%%
\begin{spacing}{1.3} 
    换行 `\textbackslash\textbackslash`只换行无缩进 \par
    `\textbackslash par`自动换行加缩进 \par
    使用ctex包显示中文 \par
    需使用间距宏包setspace \par
    行间距设置为1.3倍    \par
    par 1  \par
\end{spacing}

\begin{lstlisting}
\begin{spacing}{1.3} 
换行 `\textbackslash\textbackslash`只换行无缩进 \par
`\textbackslash par`自动换行加缩进 \par
使用ctex包显示中文 \par
需使用间距宏包setspace \par
行间距设置为1.3倍    \par
par 1  \par
par 2  \par
\end{spacing}
\end{lstlisting}

\subsection{颜色}
\textcolor[rgb]{1,0,0}{Hello, LaTeXers! This is  first LaTeX document.}.

\begin{lstlisting}
\textcolor[rgb]{1,0,0}{Hello, LaTeXers! This is our first LaTeX document.}.
\end{lstlisting}

\subsection{纸张方向}
\begin{lstlisting}
\begin{landscape}
    \textcolor[rgb]{1,0,0.5}{Hello, LaTeXers! This is our first LaTeX document.}.
\end{landscape}
\end{lstlisting}

\subsection{段落}

\paragraph[]{PA}
Hello, LaTeXers! This is our first LaTeX document.
\subparagraph{pa1}
Hello, LaTeXers! This is our first LaTeX document.
\subparagraph{}
Hello, LaTeXers! This is our first LaTeX document.

\begin{lstlisting}
Hello, LaTeXers! This is our first LaTeX document.
\subparagraph{pa1}
Hello, LaTeXers! This is our first LaTeX document.
\subparagraph{}
Hello, LaTeXers! This is our first LaTeX document.
\end{lstlisting}

\subsection{字体}

{\kaishu 【楷书】

永和九年,岁在癸丑,暮春之初,会于会稽山阴之兰亭,修稧(禊)事也。群贤毕至,少长咸集。此地有崇山峻领(岭),茂林修竹;又有清流激湍,映带左右,引以为流觞曲水,列坐其次。虽无丝竹管弦之盛,一觞一咏,亦足以畅叙幽情。}

{\heiti 【黑体】

永和九年,岁在癸丑,暮春之初,会于会稽山阴之兰亭,修稧(禊)事也。群贤毕至,少长咸集。此地有崇山峻领(岭),茂林修竹;又有清流激湍,映带左右,引以为流觞曲水,列坐其次。虽无丝竹管弦之盛,一觞一咏,亦足以畅叙幽情。}

\begin{lstlisting}
{\kaishu 【楷书】
永和九年,岁在癸丑,暮春之初,会于会稽山阴之兰亭,修稧(禊)事也。群贤毕至,少长咸集。此地有崇山峻领(岭),茂林修竹;又有清流激湍,映带左右,引以为流觞曲水,列坐其次。虽无丝竹管弦之盛,一觞一咏,亦足以畅叙幽情。}
{\heiti 【黑体】
永和九年,岁在癸丑,暮春之初,会于会稽山阴之兰亭,修稧(禊)事也。群贤毕至,少长咸集。此地有崇山峻领(岭),茂林修竹;又有清流激湍,映带左右,引以为流觞曲水,列坐其次。虽无丝竹管弦之盛,一觞一咏,亦足以畅叙幽情。}
\end{lstlisting}
%%%%%%%%%%%%%%%%%%%%%%%%%%%%%%%%%%%%%%%%%%%%%%%%%%%%%%%%%%%%%%%%%%%%%%%%%%%%%%%%%%%%%%%%%%%%%%%%
\newpage
\part{文本编辑}
% 每一部分section从0开始
\setcounter{section}{0}

\section{标题作者日期摘要}
\begin{abstract}
    \begin{itemize}
        \item 使用\colorbox{lightgray}{\textbackslash title\{\} }命令设置标题 \\ 对于较长的文档标题可以使用\colorbox{lightgray}{\textbackslash\textbackslash} \\对标题内容进行分行
        \item 使用\colorbox{lightgray}{\textbackslash author\{\} }设置作者 \\ 作者之间使用\colorbox{lightgray}{\textbackslash and }进行分隔 
        \item 使用\colorbox{lightgray}{\textbackslash date\{\} }设置日期 \\ \colorbox{lightgray}{\{\}}不写任何信息表示默认使用当前日期
        \item 使用\colorbox{lightgray}{\textbackslash maketitle }创建标题
        \item 使用\colorbox{lightgray}{\textbackslash begin\{abstract\}} 撰写摘要 \\ 并在其后使用\colorbox{lightgray}{\textbackslash textbf\{\}}设置文档关键词
    \end{itemize}
\end{abstract}
\textbf{Keywords:标题,作者,日期,摘要}

\section{创建章节}

\subsection{设置章节自动编号深度}
用户也可以通过在导言区使用\colorbox{lightgray}{\textbackslash setcounter\{secnumdepth\}\{\}} 
计数器命令设置章节自动编号深度,从而达到批量取消自动编号的效果。
在\{\}中填写编号深度值,编号深度值从0开始设置,
\begin{itemize}
    \item \colorbox{lightgray}{\textbackslash setcounter\{secnumdepth\}\{0\}}表示自动编号章节层次仅包括\colorbox{lightgray}{\textbackslash part}和chapter;
    \item \colorbox{lightgray}{\textbackslash setcounter\{secnumdepth\}\{1\}}表示自动编号章节层次深入到\colorbox{lightgray}{\textbackslash section}级;
    \item \colorbox{lightgray}{\textbackslash setcounter\{secnumdepth\}\{2\}}表示自动编号章节层次深入到\colorbox{lightgray}{\textbackslash subsection}级;
    \item \colorbox{lightgray}{\textbackslash setcounter\{secnumdepth\}\{3\}}表示自动编号章节层次深入到\colorbox{lightgray}{\textbackslash subsubsection}级;
\end{itemize}
\subsection{改变字体样式}
改变字体样式需要使用\colorbox{lightgray}{titlesec}宏包,使用宏包命令\colorbox{lightgray}{\textbackslash titleformat*\{\}\{\}}
\subsection{标题居中}
调用\colorbox{lightgray}{sectsty}宏包,并用\colorbox{lightgray}{\textbackslash sectionfont \{\textbackslash centering\}}

\section{生成目录}
章节目录一般创建在摘要之后,在\LaTeX 中,使用\colorbox{lightgray}{\textbackslash tableofcontents}命令即可。
命令放在哪里,就会在哪里自动创建一个章节目录。默认情况下,该命令会根据用户定义的篇章节标题生成文章目录,
目录中将包含\colorbox{lightgray}{\textbackslash subsubsection}及其更高层次的结构标题。
但对于带星号的章节命令,其章节标题不会出现在目录中。
\subsection[层次别名]{调节章节层次深度}
在导言区使用\colorbox{lightgray}{\textbackslash setcounter\{tocdepth\}\{\}}命令指定目录中的章节层次深度
我们也可以为每个章节设置不同的目录层次,具体是通过在每个章节创建命令前,使用\\ \colorbox{lightgray}{\textbackslash addtocontents\{toc\}\{\textbackslash setcounter\{tocdepth\}\{\}\}}命令为该章节指定目录层次深度。
\subsection{引用链接}
如果想要为目录中的章节引用添加链接,使得点击链接后就能跳转到相应章节所在页面,那么只需要在导言区调用\colorbox{lightgray}{hyperref}宏包、并设置\colorbox{lightgray}{colorlinks=true}选项即可,此时文档中章节引用及其它交叉引用均会被自动添加链接(添加了链接的引用将显示为红色)。

\section{编辑段落}
\subsection{段落首行缩进}
\setlength{\parindent}{4em}
若想调整段落首行缩进的距离,可以使用\colorbox{lightgray}{\textbackslash setlength\{\textbackslash parindent\}\{长度\}}命令,在\{长度\}处填写需要设置的距离即可。\\
\noindent
如果不想让段落自动首行缩进, 在段落前使用命令\colorbox{lightgray}{\textbackslash noindent}即可。\\
\setlength{\parindent}{2em}
在段落设置在章节后面时,每一节后的第一段默认是不缩进的,为了使第一段向其他段一样缩进,可以在段落前使用\colorbox{lightgray}{\textbackslash hspace*\{\textbackslash parindent\}}命令,也可以在源文件的前导代码中直接调用宏包\colorbox{lightgray}{\textbackslash usepackage\{indentfirst\}}。
\subsection{段落间距调整}
LaTeX排版时,有时为了使段落与段落之间的区别更加明显,
\smallskip
我们可以在段落之间设置一定的间距,
\medskip
最简单的方式是使用\colorbox{lightgray}{\textbackslash smallskip},
\bigskip
\colorbox{lightgray}{\textbackslash medskip},\colorbox{lightgray}{\textbackslash bigskip},等命令

\subsection{段落添加文本框}
\fbox{
    \parbox{0.8\linewidth}{
有时因为文档全都是大段大段的文字,版面显得较为单调,这时,我们可以通过给文字加边框来让版面有所变化,不至于过于单调。在LaTeX中,我们可以使用\colorbox{lightgray}{\textbackslash fbox\{\}}命令对文本新增边框
}
}

\subsection{段落对齐方式}
\LaTeX 默认的对齐方式是两端对齐,
\begin{center}
    有时在进行文档排版的过程中,我们为了突出某一段落的内容,会选择将其居中显示,
    在LaTeX中,我们可以使用center环境对文本进行居中对齐。
\end{center}
\begin{flushright}
    另外还有一些出版商要求文档是左对齐或者右对齐,
\end{flushright}
\begin{flushleft}
    另外还有一些出版商要求文档是左对齐或者右对齐,这时我们同样可以使用\colorbox{lightgray}{flushleft}环境和\colorbox{lightgray}{flushright}环境将文档设置为左对齐或右对齐
\end{flushleft}

\section{文本编辑}
文本编辑是制作文档非常重要的一部分,在编辑文本的过程中,主要关注的内容有如何调整字体样式、字体设置、增加下划线、突出文字、调整字体大小、调整对齐格式等。
\subsection{调整字体样式}
调整文字的样式有很多对应的命令,这些命令包括\colorbox{lightgray}{\textbackslash textit、\textbackslash textbf、\textbackslash textsc、\\ \textbackslash texttt},

在使用的过程中,需要用到花括号\{\}。具体而言,\textbackslash textit对应着\textit{斜体字},\textbackslash textbf对应着\textbf{粗体字},\textbackslash textsc对应着\textsc{小型大写字母},\textbackslash texttt对应着\texttt{打印机字体(即等宽字体}。

如果想对文本中的英文字母进行全部大写,可用\textbackslash uppercase和\textbackslash MakeUppercase两个命令中的任意一个。 \\
\uppercase{Use uppercase command to force all uppercase.}
\MakeUppercase{Use MakeUppercase command to force all uppercase.}

\subsection{调整字体大小}
字体大小的设置一方面可以在申明文档类型的命令\textbackslash documentclass[]\{\}中指定具体的字体大小(如11pt、12pt)来实现,另一方面也可以通过一些简单的命令来调整。
\setlength{\parindent}{0em}

Produce {\tiny tiny word}

Produce {\scriptsize script size word}

Produce {\footnotesize footnote size word}

Produce {\normalsize normal size word}

Produce {\large large word}

Produce {\Large Large word}

Produce {\LARGE LARGE word}

Produce {\huge huge word}

Produce {\Huge Huge word}

\vspace{0.5cm}
\fbox{
    \parbox{0.8\linewidth}{
代码:

Produce \{\textbackslash tiny tiny word\}

Produce \{\textbackslash scriptsize script size word\}

Produce \{\textbackslash footnotesize footnote size word\}

Produce \{\textbackslash normalsize normal size word\}

Produce \{\textbackslash large large word\}

Produce \{\textbackslash Large Large word\}

Produce \{\textbackslash LARGE LARGE word\}

Produce \{\textbackslash huge huge word\}

Produce \{\textbackslash Huge Huge word\}

Produce \begin{large}large word\end{large}

Produce \begin{Large}large word\end{Large}

Produce \begin{LARGE}large word\end{LARGE}
    }
}

\vspace{0.5cm}

Produce \begin{large}large word\end{large}

Produce \begin{Large}large word\end{Large}

Produce \begin{LARGE}large word\end{LARGE}

\vspace{0.5cm}
\fbox{
    \parbox{0.8\linewidth}{
代码:

Produce \textbackslash begin\{large\}large word\textbackslash end\{large\}

Produce \textbackslash begin\{Large\}large word\textbackslash end\{Large\}

Produce \textbackslash begin\{LARGE\}large word\textbackslash end\{LARGE\}
    }
}

\subsection{调整字体颜色}
一般而言,文本默认的颜色是黑色,但有时候,我们可以根据需要改变字体的颜色,这通过\LaTeX 一些拓展的宏包就可以实现,例如\colorbox{lightgray}{xcolor}

使用颜色宏包时,我们也可以根据需要自定义颜色,相应的命令为\\ \colorbox{lightgray}{\textbackslash definecolor\{A\}\{B\}\{C\}},其中,位置A是颜色标签,位置B是制定颜色系统为RGB(英文缩写RGB是红色、绿色和蓝色三种颜色的英文单词首字母),位置C是具体的RGB数值。
\newpage
This is a simple example for using \LaTeX.

{\color{kugreen}This is a simple example for using \LaTeX.}

{\color{kugreenlys}This is a simple example for using \LaTeX.}

{\color{kugreenlyslys}This is a simple example for using \LaTeX.}

{\color{kugreenlyslyslys}This is a simple example for using \LaTeX.}

\noindent
\vspace{0.5cm}
\begin{lstlisting}
代码
\documentclass[12pt]{article}
\usepackage{color}
\definecolor{kugreen}{RGB}{50, 93, 61}
\definecolor{kugreenlys}{RGB}{132, 158, 139}
\definecolor{kugreenlyslys}{RGB}{173, 190, 177}
\definecolor{kugreenlyslyslys}{RGB}{214, 223, 216}
\begin{document}
This is a simple example for using \LaTeX.
{\color{kugreen}This is a simple example for using \LaTeX.}
{\color{kugreenlys}This is a simple example for using \LaTeX.}
{\color{kugreenlyslys}This is a simple example for using \LaTeX.}
{\color{kugreenlyslyslys}This is a simple example for using \LaTeX.}
\end{document}
\end{lstlisting}

\subsection{字体设置}
不管是英文还是中文,我们都会越到各种各样的字体,因此,使用LaTeX编译出想要的字体对于整个文档也非常重要。对于英文文档的编译,一般会用宏包fontspec设置具体的字体,调用格式为\colorbox{gray}{\textbackslash usepackage\{fontspec\}},需要说明的是,这个宏包只能设置英文的字体
\vspace{0.5cm}
\begin{lstlisting}
\setmainfont{Times New Roman}
\setsansfont{DejaVu Sans}
\setmonofont{Latin Modern Mono}
\setsansfont{[foo.ttf]}
\end{lstlisting}
\vspace{0.5cm}
在LaTeX中,编译文档一般默认的英文字体为\colorbox{lightgray}{Computer Modern},如果要将其调整为其他特定类型的字体,可以在前导代码中使用各种字体对应的工具包。
\newpage
\begin{lstlisting}
\documentclass[a4paper, 12pt]{article}
\usepackage[T1]{fontenc}
\usepackage[utf8]{inputenc}
\begin{document}
Hello, LaTeXers! This is our first LaTeX document.
\end{document}
\end{lstlisting}

如果文档输入的是中文,首先需要申明文档类型为ctex中的ctexart、ctexrep之类的。Part I 6节
介绍了部分中文字体的设置方式,宋体、仿宋、隶书、黑体、楷体、幼圆、微软雅黑等字体为目前常用
的大部分中文字体,下例将使用ctexart给出它们的设置方式。

\noindent 字体(默认宋体)\\
\fangsong 字体(仿宋) \songti 字体(宋体)  \heiti 字体(黑体)\\
\CJKfamily{zhkai} 字体(楷书) \CJKfamily{zhyou} 字体(幼圆) \CJKfamily{zhyahei} 字体(微软雅黑)\\

\begin{lstlisting}
代码:
\documentclass[12pt,a4paper,utf8]{ctexart}
\begin{document}
\noindent 字体(默认宋体)\\
\fangsong 字体(仿宋) \songti 字体(宋体) \lishu 字体(隶书) \heiti 字体(黑体)\\
\CJKfamily{zhkai} 字体(楷书) \CJKfamily{zhyou} 字体(幼圆) \CJKfamily{zhyahei} 字体(微软雅黑)\\
\end{document}
\end{lstlisting}

\subsection{下划线与删除线}
有时候,为了突出特定的文本,我们也会使用到各种下划线。最常用的下划线命令是\colorbox{gray}{\textbackslash underline},
然而,这个命令存在一个缺陷,即当文本过长,超过页面宽度时,下划线不会自动跳到下一行,
因此,我们需要用到一个叫\colorbox{gray}{ulem}的宏包,使用这个宏包中的命令\colorbox{gray}{\textbackslash uline}可以增加单下划线,
使用\colorbox{gray}{\textbackslash uuline}可以增加双下划线,而使用\colorbox{gray}{\textbackslash uwave}则可以增加波浪线
\newpage
Generate \underline{underlined} text. \\
Generate \uline{underlined} text. \\
Generate \uuline{underlined} text. \\
Generate \uwave{underlined} text. \\
Generate \st{stridethrough} text. \\
Generate \emph{emphasized} text. \\
Generate ``quote'' text \\
\begin{lstlisting}
代码:
\documentclass[12pt]{article}
\usepackage{ulem}
\begin{document}
Generate \underline{underlined} text. \\    
% 生成带下划线的文本(使用\underline命令)
Generate \uline{underlined} text. \\         
% 生成单下划线的文本(使用\uline命令)
Generate \uuline{double underlined} text. \\ 
% 生成单下划线的文本
Generate \uwave{wavy underlined} text. \\   
% 生成波浪线的文本
Generate \st{strikethrough} text.
\end{document}
\end{lstlisting}


\subsection{特殊字符}
\setlength{\parindent}{2em}
在\LaTeX 中,有很多特殊字符的编译需要遵循一定的规则,例如,
反斜杠 (backslash) 符号是LaTeX中定义和使用各类命令的基本符号,
如果要在文档中编译出反斜杠,可使用\colorbox{lightgray}{\textbackslash textbackslash}
百分号通常用于注释代码,如果要在文档中编译出百分号,可使用\colorbox{lightgray}{\textbackslash \%}
美元符号通常用于书写公式,如果要在文档中编译出美元符号,可使用\colorbox{lightgray}{\textbackslash \%}。
带圆圈数字可用于各类编号,我们可以根据需要插入这种特殊符号。
在\LaTeX 中,比较常用的一种生成带圆圈数字的方法是使用宏包\colorbox{lightgray}{pifont},
在前导代码中申明使用宏包,即\colorbox{lightgray}{\textbackslash usepackage\{pifont\}},根据宏包所提供的命令\colorbox{lightgray}{\textbackslash ding\{\}}可以生成从1到10的带圆圈数字,且圆圈样式也各异。
\noindent
How to write a number in a circle? \\
Type 1: \ding{172}-\ding{173}-\ding{174}-\ding{175}-\ding{176}-\ding{177}-\ding{178}-\ding{179}-\ding{180}-\ding{181} \\     % 样式1是从172开始
Type 2: \ding{182}-\ding{183}-\ding{184}-\ding{185}-\ding{186}-\ding{187}-\ding{188}-\ding{189}-\ding{190}-\ding{191} \\     % 样式2是从182开始
Type 3: \ding{192}-\ding{193}-\ding{194}-\ding{195}-\ding{196}-\ding{197}-\ding{198}-\ding{199}-\ding{200}-\ding{201} \\     % 样式3是从192开始
Type 4: \ding{202}-\ding{203}-\ding{204}-\ding{205}-\ding{206}-\ding{207}-\ding{208}-\ding{209}-\ding{210}-\ding{211} \\     % 样式4是从202开始

\begin{lstlisting}
\documentclass[12pt]{article}
\usepackage{pifont}
\begin{document}

How to write a number in a circle? \\
Type 1: \ding{172}-\ding{173}-\ding{174}-\ding{175}-\ding{176}-\ding{177}-\ding{178}-\ding{179}-\ding{180}-\ding{181} \\     % 样式1是从172开始
Type 2: \ding{182}-\ding{183}-\ding{184}-\ding{185}-\ding{186}-\ding{187}-\ding{188}-\ding{189}-\ding{190}-\ding{191} \\     % 样式2是从182开始
Type 3: \ding{192}-\ding{193}-\ding{194}-\ding{195}-\ding{196}-\ding{197}-\ding{198}-\ding{199}-\ding{200}-\ding{201} \\     % 样式3是从192开始
Type 4: \ding{202}-\ding{203}-\ding{204}-\ding{205}-\ding{206}-\ding{207}-\ding{208}-\ding{209}-\ding{210}-\ding{211} \\     % 样式4是从202开始

\end{document}    
\end{lstlisting}

\section{创建列表}
在内容表达上,列表是一种非常有效的方式,它将某一论述内容分成若干个条目进行罗列,
能达到简明扼要、醒目直观的表达效果。在论文写作中,列表不失为一种让内容清晰明了的论述方式。
按层次分,列表有单层列表和多层列表,多层列表无外乎是最外层列表中嵌套了一层甚至更多层列表。
按类型分,列表主要有三种类型,即无序列表、排序列表和阐述性列表,其中,
无序列表和排序列表是相对常用的列表类型,\LaTeX 针对这三种列表提供了一些基本环境
\newpage
\begin{itemize}
    \item 无序列表使用方法 
    \begin{lstlisting}
\begin{itemize}

    \item Item 1 % 条目1
    \item Item 2 % 条目2
    
\end{itemize}
    \end{lstlisting}
    \item 排序列表使用方法
    \begin{lstlisting}
\begin{enumerate}
    \item Item 1 % 条目1
    \item Item 2 % 条目2
\end{enumerate}
    \end{lstlisting}
    \item 阐述性列表 
    \begin{lstlisting}
\begin{description}
    \item Item 1 % 条目1
    \item Item 2 % 条目2
\end{description}
    \end{lstlisting}
\end{itemize}


\subsection{无序列表}
LaTeX中的无序列表环境一般用特定符号(如圆点、星号)作为列表中每个条目的起始标志,以示有别于常规文本。
可以忽略主次或者先后顺序关系的条目排列都可以使用无序列表环境来编写,
无序列表时很多文档最常用的列表类型,也被称为常规列表。
\vspace{0.5cm}
\begin{lstlisting}
代码
\documentclass[12pt]{article}
\begin{document}
\begin{itemize}
\item Python % 条目1
\item LaTeX  % 条目2
\item GitHub % 条目3
\end{itemize}
\end{document}
\end{lstlisting}

在无序列表环境中,每个条目都是以条目命令\colorbox{lightgray}{\textbackslash item}开头的,一般默认的起始符号是textbullet,即大圆点符号,当然,也可以根据需要调整起始符号
\begin{itemize}
    \item Python
    \item LaTeX
    \item[*] GitHub 
\end{itemize}

\begin{lstlisting}
代码
\documentclass[12pt]{article}
\begin{document}
\begin{itemize}

\item Python    % 条目1,起始符号为大圆点
\item LaTeX     % 条目2,起始符号为大圆点
\item[*] GitHub % 条目3,起始符号为星号

\end{itemize}
\end{document} 
\end{lstlisting}

如果想要将所有条目的符号都进行调整,并统一为某一个特定符号,则可以使用\colorbox{lightgray}{\textbackslash renewcommand}命令进行自定义。
其中,命令\colorbox{lightgray}{\textbackslash labelitemi}是由三部分组成,即label(标签)、item(条目)、i(一级),有时候,如果需要创建多级列表,则可以类似这里使用命令\colorbox{lightgray}{\textbackslash labelitemii}(对应于二级列表)甚至\colorbox{lightgray}{\textbackslash labelitemiii}(对应于三级列表)。
\begin{itemize}
    \renewcommand{\labelitemi}{\scriptsize$\blacksquare$}
    \item Python % 条目1
    \item LaTeX  % 条目2
    \item GitHub % 条目3
\end{itemize}
\begin{lstlisting}
\documentclass[12pt]{article}
\usepackage{amssymb}
\begin{document}
\begin{itemize}
    \renewcommand{\labelitemi}{\scriptsize$\blacksquare$}
    \item Python % 条目1
    \item LaTeX  % 条目2
    \item GitHub % 条目3
\end{itemize}
\end{document}    
\end{lstlisting}

\subsection{排序列表}
排序列表也被称为编号列表。在排序列表中,每个条目之前都有一个标号,它是由标志和序号两部分组成:
序号自上而下,从1开始升序排列;标志可以是括号或小圆点等符号。
相互之间有密切的关联,通常是按过程顺序或是重要程度排列的条目都可以采用排序列表环境编写。
排序列表环境\colorbox{lightgray}{enumerate}以序号作为列表的起始标志,
每个条目命令\colorbox{lightgray}{item}将在每个条目之前自动加上一个标号条目命令\colorbox{lightgray}{item}生成的默认标号样式为阿拉伯数字加小圆点。
排序列表同样可以进行相互嵌套,最多可以达到四层,为了便于区分,
不仅每层列表的条目段落都有不同程度的左缩进,二期每层列表中条目的标号也各不相同,
其中序号的计数形式与条目所在的层次有关,标志所用的符号除第2层是圆括号外,其他各层都是小圆点。
\begin{enumerate}
    \item pencil
    \item calculator
    \item rule
    \item notebook
        \begin{enumerate}
            \item notes
            \begin{enumerate}
                \item note A
                \begin{enumerate}
                    \item note A
                \end{enumerate}
                \item note B
            \end{enumerate}
            \item homework
            \item assessments
        \end{enumerate}
\end{enumerate}

\newpage
\begin{lstlisting}
代码
\documentclass[12pt]{article}
\begin{document}
\begin{enumerate}
\item pencil
\item calculator
\item ruler
\item notebook
    \begin{enumerate}
    \item notes
        \begin{enumerate}
        \item note A
            \begin{enumerate}
            \item note a
            \end{enumerate}
        \item note B
        \end{enumerate}
    \item homework
    \item assessments
    \end{enumerate}
\end{enumerate}
\end{document}    
\end{lstlisting}

\subsection{阐述性列表}
相比无序列表和排序列表,阐述性列表的使用频率较低,它常用于对一组专业术语进行解释说明。
阐述性列表环境为\colorbox{lightgray}{description}。在\colorbox{lightgray}{description}环境命令中,
每个词条都是需要分别进行阐述的词语,每个阐述可以是一个或多个文本段落。
这种形式很像词典,因此诸如名词解释说明之类的列表就可以采用解说列表环境来编写。
在阐述性列表环境中,被解释的词条的格式是用\colorbox{lightgray}{descriptionlabel}定义的。

\begin{description}
    \item[CNN] Convolutional Neural Networks
    \item[RNN] Recurrent Neural Networks
    \item[CRNN] Convolutional Recurrent Neural Networks  
\end{description}

\newpage
\begin{lstlisting}
代码
\documentclass[12pt]{article}
\begin{document}
\begin{description}
\item [CNN] Convolutional Neural Networks
\item [RNN] Recurrent Neural Network
\item [CRNN] Convolutional Recurrent Neural Network
\end{description}
\end{document}    
\end{lstlisting}


\subsection{自定义列表格式}
使用系统默认的\LaTeX 列表环境排版的列表与上下文之间以及列表条目之间都附加有一段垂直空白,
明显有别于列表环境之外的文本格式,通常列表中的条目内容都很简短,这样会造成很多空白,使得列表看起来很稀疏,
与前后文本之间的协调性较差。因此,我们需要自定义列表格式。
使用\colorbox{lightgray}{enumitem}宏包可以调整\colorbox{lightgray}{enumerate}或\colorbox{lightgray}{itemize}的上下左右缩进间距。

\subsubsection{垂直间距}
\begin{itemize}
    \item \colorbox{lightgray}{topsep} 列表环境与上文之间的距离
    \item \colorbox{lightgray}{parsep} 条目里面段落之间的距离
    \item \colorbox{lightgray}{itemsep}条目之间的距离
    \item \colorbox{lightgray}{partopsep} 条目与下面段落的距离
\end{itemize}
\subsubsection{水平间距}
\begin{itemize}
    \item \colorbox{lightgray}{leftmargin} 列表环境左边的空白长度
    \item \colorbox{lightgray}{rightmargin} 列表环境右边的空白长度
    \item \colorbox{lightgray}{labelsep} 标号与列表环境左侧的距离
    \item \colorbox{lightgray}{itemindent} 条目的缩进距离
    \item \colorbox{lightgray}{labelwidth} 标号的宽度
    \item \colorbox{lightgray}{listparindent} 条目下面段落的缩进距离
\end{itemize}


Default spacing:

\begin{itemize}

\item Python % 条目1
\item LaTeX  % 条目2
\item GitHub % 条目3

\end{itemize}

Custom Spacing:

\begin{itemize}[itemsep= 15 pt,topsep = 20 pt,itemindent=20pt]

\item Python % 条目1
\item LaTeX  % 条目2
\item GitHub % 条目3

\end{itemize}

\begin{lstlisting}
使用enumitem宏包调整无序列表间距
\documentclass[12pt]{article}
\usepackage{enumitem}
\begin{document}
Default spacing:
\begin{itemize}
\item Python % 条目1
\item LaTeX  % 条目2
\item GitHub % 条目3
\end{itemize}
Custom Spacing:
\begin{itemize}[itemsep= 15 pt,topsep = 20 pt]
\item Python % 条目1
\item LaTeX  % 条目2
\item GitHub % 条目3
\end{itemize}
\end{document}   
\end{lstlisting}

\section{创建页眉页脚脚注}
在大多数文档中,我们往往需要页眉、页脚及脚注来展示文档的附加信息,
例如时间、图形、页码、日期、公司微标、页眉示意图文档标题、文件名或作者姓名等信息。
在\LaTeX 中,我们常用\colorbox{lightgray}{Fancyhdr}工具包进行页眉、页脚的设置。

\newpage
\begin{lstlisting}
使用Fancyhdr工具包进行页眉、页脚的设置
\documentclass{article}
\usepackage{fancyhdr}
\pagestyle{fancy}
\lhead{}
\chead{}

\rhead{\bfseries latex-cookbook} %页眉内容
\lfoot{From: Xinyu Chen} %页脚内容
\cfoot{To: Jieling Jin} %页脚内容
\rfoot{\thepage} %在页脚处给出页码
\renewcommand{\headrulewidth}{0.4pt}
\renewcommand{\footrulewidth}{0.4pt}

\begin{document}
This is latex-cookbook!
\end{document}
\end{lstlisting}

如果想某一页不需要页眉页脚,可以在该页正文内容开始时使用\textbackslash thispagestyle\{empty\}命令,去除该页页眉页脚。
\vspace{0.5cm}
\begin{lstlisting}
\documentclass{article}
\usepackage{fancyhdr}
\pagestyle{fancy}
\lhead{}
\chead{}

\rhead{\bfseries latex-cookbook}
\lfoot{From: Xinyu Chen}
\cfoot{To: Jieling Jin}
\rfoot{\thepage}
\renewcommand{\headrulewidth}{0.4pt}
\renewcommand{\footrulewidth}{0.4pt}

\begin{document}
This is latex-cookbook!
\newpage
\thispagestyle{empty}
This is latex-cookbook!
\end{document}
\end{lstlisting}

脚注也是一篇文档中的重要内容,一般用来解释文档中的名词或相关出处。在\LaTeX 中,我们常用\colorbox{lightgray}{\textbackslash footnote\{\}}命令来添加脚注\footnote[2]{This is a learning course}。
\vspace{0.5cm}
\begin{lstlisting}
\documentclass{article}
\begin{document}
Casea \footnote{Tian jin university} 
\end{document}
\end{lstlisting}

在论文撰写过程中,我们有时会需要在表格中添加脚注,但是在table环境中,\colorbox{lightgray}{\textbackslash footnote\{\}}命令不起作用,这时我们可以使用\colorbox{lightgray}{\textbackslash minipage\{\}}环境来化解。

\begin{center}
    \begin{minipage}{.5\textwidth}
    \begin{tabular}{l|l}
        \textsc{Chapter} & \textsc{Author} \\ \hline
        \textit{Introduction} & Xinyu Chen\footnote{Xinyu Chen is a PhD from the University of Montreal.}  \\
        \textit{Methods} & Jieling Jin \footnote{Jieling Jin is a PhD from the Central South University.} \\
        \textit{Case Study} & Xinyu Chen \\
        \textit{Conclusion} & Jieling Jin
        \end{tabular}
    \end{minipage}
\end{center}

\begin{lstlisting}
使用minipage环境来给表格添加脚注。
\documentclass{article}
\begin{document}
\begin{center}
    \begin{minipage}{.5\textwidth}
    \begin{tabular}{l|l}
        \textsc{Chapter} & \textsc{Author} \\ \hline
        \textit{Introduction} & Xinyu Chen\footnote{Xinyu Chen is a PhD from the University of Montreal.}  \\
        \textit{Methods} & Jieling Jin \footnote{Jieling Jin is a PhD from the Central South University.} \\
        \textit{Case Study} & Xinyu Chen \\
        \textit{Conclusion} & Jieling Jin
        \end{tabular}
    \end{minipage}
    \end{center}
\end{document}
\end{lstlisting}

\newpage

\part{公式编辑}
\setcounter{section}{0}
\section{基本介绍}
由于\LaTeX 编辑的数学公式颜值非常高,很多理工科研究领域的顶级期刊甚至明确要求投稿论文必须按照给定的\LaTeX 
模板进行论文排版,这样做一方面能保证论文整体排版的美观程度,另一方面也能让生成出来的数学公式更加规范。
一般而言,使用\LaTeX 编辑公式的一系列规则与数学公式的编写原则是一致的,
例如,在\LaTeX 中,我们可以用\colorbox{lightgray}{\$\textbackslash frac\{\textbackslash partial f\}\{\textbackslash partial x\}\$}生成偏导数$\frac{\partial f}{\partial x}$

\subsection{数学公式环境}
\subsubsection{美元符号}
在\LaTeX 中生成数学公式也有一些基本规则,插入公式的方式有很多种,最基本的一种方式是使用美元符号,
这种方式不仅在\LaTeX 适用,在Markdown中也是适用的
\begin{itemize}
    \item 插入行内公式,可以直接在两个美元符号中间编辑需要的公式。
    \item 插入行间公式,输入四个美元符号,在四个美元符号中间编辑需要的公式,生成的数学公式会自动居中对齐。
\end{itemize}
$x+y=2$ is a simple linear equation  
$$x+y=2$$
\begin{lstlisting}
用美元符号分别在行内和行间生成一条简单的数学公式
\documentclass[12pt]{article}
\begin{document}
$x+y=2$ is a simple linear equation
$$x+y=2$$
\end{document}
\end{lstlisting}

\LaTeX 源文件中的美元符号一般都默认为申明数学公式环境,如果想要在文档中编译出美元符号,
需要在美元符号前加上一个反斜线,这种做法同样适用于百分号,百分号一般被默认为注释功能。
\subsubsection{equation 环境}
美元符号可以在行间插入公式,但却没办法对公式进行编号。
自动生成带有公式编号的行间公式需要用到数学公式环境\colorbox{lightgray}{\textbackslash begin\{equation\} \textbackslash end\{equation\}},
使用数学公式环境\colorbox{lightgray}{\textbackslash begin\{equation\} \textbackslash end\{equation\}}, \LaTeX 编译时会自动将公式进行居中对齐。

\begin{equation}
    x+y=2
\end{equation}

\begin{lstlisting}
\documentclass[12pt]{article}
\begin{document}
\begin{equation}
x+y=2
\end{equation}
\end{document}
\end{lstlisting}

\subsubsection{align 环境}
在\LaTeX 中,除了equation数学公式环境,还有其他几种数学公式环境可以使用。
我们要介绍的第一种是\colorbox{lightgray}{\textbackslash begin\{align\} \textbackslash end\{align\}},
它主要用于数组型的数学表达式,align环境可以将公式进行自动对齐,它也能对每一条数学表达式分别进行公式编号。 \\
\begin{align}
    x+y&=2 \\
    2x+y&=3
\end{align}
\newpage
\begin{lstlisting}
\documentclass[12pt]{article}
\usepackage{amsmath}
\begin{document}
%% 使用align环境
\begin{align}
x+y&=2 \\
2x+y&=3
\end{align}
\end{document}    
\end{lstlisting}

\begin{lstlisting}
\documentclass[12pt]{article}
\usepackage{amsmath}
\begin{document}
\begin{align*}
2x+1&=7 & 3y-2&=-5 & -5z+8&=-2 \\
  2x&=6 &   3y&=-3 &   -5z&=-10 \\
   x&=3 &    y&=-1 &     z&=2
\end{align*}
\end{document}
\end{lstlisting}

\begin{lstlisting}
使用\begin{align} \end{align}编译一个方程组,并且只对第2个方程进行编号。
\documentclass[12pt]{article}
\usepackage{amsmath}
\begin{document}
\begin{align}
x+y=2 \nonumber \\
2x+y=3
\end{align}
\end{document}
\end{lstlisting}

\subsubsection{gather 环境}
我们要介绍的第二种数学公式环境是\colorbox{lightgray}{\textbackslash begin\{gather\} \textbackslash end\{gather\}},它既可以将公式进行居中对齐,
也能对每一条数学表达式分别进行公式编号。同样的,如果想要移除公式编号,只需要在公式环境中加上星号即可。
\begin{gather}
    x+y=2 \\
    2x+y=3
\end{gather}

\begin{lstlisting}
使用\begin{gather} \end{gather}编译一个方程组。
\documentclass[12pt]{article}
\usepackage{amsmath}
\begin{document}
\begin{gather}
x+y=2 \\
2x+y=3
\end{gather}
\end{document}
\end{lstlisting}

\subsection{基本格式调整}
\setcounter{equation}{0}
\subsubsection{字符类型}
在文本编辑中,我们已经介绍了几种常见的字符类型,实际上,对于数学公式而言,
书写时也可以设置不同的字符类型。以X,Y,Z,x,y,z为例
\begin{itemize}
    \item 命令\textbackslash boldsymbol\{X,Y,Z,x,y,z\},编译后的效果为$\boldsymbol{X,Y,Z,x,y,z}$,使用之前需申明\textbackslash usepackage\{amsmath\};
    \item 命令\textbackslash mathrm\{X,Y,Z,x,y,z\},编译后的效果为$\mathrm{X,Y,Z,x,y,z}$;
    \item 命令\textbackslash mathit\{X,Y,Z,x,y,z\},编译后的效果为$\mathit{X,Y,Z,x,y,z}$;
    \item 命令\textbackslash mathbf\{X,Y,Z,x,y,z\},编译后的效果为$\mathbf{X,Y,Z,x,y,z}$;
    \item 命令\textbackslash mathsf\{X,Y,Z,x,y,z\},编译后的效果为$\mathsf{X,Y,Z,x,y,z}$;
    \item 命令\textbackslash mathtt\{X,Y,Z,x,y,z\},编译后的效果为$\mathtt{X,Y,Z,x,y,z}$;
    \item 命令\textbackslash boldmath\{X,Y,Z,x,y,z\},编译后的效果为$\boldmath{X,Y,Z,x,y,z}$;依赖于特定工具包,使用之前需申明\textbackslash usepackage\{amssymb\};
    \item 命令\textbackslash mathcal\{X,Y,Z\},编译后的效果为$\mathcal{X,Y,Z}$;
    \item 命令\textbackslash mathbb\{X,Y,Z\},依赖于特定工具包,使用之前需申明\textbackslash usepackage\{amssymb, amsfonts\},编译后的效果为$\mathbb{X,Y,Z}$,概率论与数理统计中常见的数学期望符号E也是用该命令编译的,即$\mathbb{E}$;
    \item 命令\textbackslash mathfrak\{X,Y,Z,x,y,z\},依赖于特定工具包,使用之前需申明\\ \textbackslash usepackage\{amssymb, amsfonts, eufrak\},编译后的效果为$\mathfrak{X,Y,Z,x,y,z}$。
\end{itemize}

\begin{equation}
    x^{2}+y^{2}-\sin z=4
\end{equation}
    
\boldmath
\begin{equation}
x^{2}+y^{2}-\sin z=4
\end{equation}
\unboldmath

\begin{lstlisting}
使用\boldmath和\unboldmath对如下公式进行加粗
\documentclass[12pt]{article}
\begin{document}
\begin{equation}
x^{2}+y^{2}-\sin z=4
\end{equation}
\boldmath
\begin{equation}
x^{2}+y^{2}-\sin z=4
\end{equation}
\unboldmath
\end{document}
\end{lstlisting}

除此之外,如果想在公式中插入正常的文本,可以使用\colorbox{lightgray}{\textbackslash text\{\}}命令,例如\\ \$T\_\{\textbackslash text\{start\}\}\$为$T_{\text{start}}$。
\setcounter{equation}{0}
\subsubsection{调整公式大小}
如果想对单个公式调整公式字符大小,在美元符号插入的公式中,
可以使用\textbackslash displaystyle、\textbackslash textstyle、\textbackslash scriptstyle和\textbackslash scriptscriptstyle等申明命令对公式大小进行调整,公式显示效果依次从小到大,这些命令一般放在公式前即可。
\begin{center}
$\displaystyle{f(x)=\sum_{i=1}^{n}\frac{1}{x_{i}}}$ \\
$\textstyle{f(x)=\sum_{i=1}^{n}\frac{1}{x_{i}}}$ \\
$\scriptstyle{f(x)=\sum_{i=1}^{n}\frac{1}{x_{i}}}$ \\
$\scriptscriptstyle{f(x)=\sum_{i=1}^{n}\frac{1}{x_{i}}}$ \\
\end{center}

\newpage

\begin{lstlisting}
使用\displaystyle、\textstyle、\scriptstyle和\scriptscriptstyle这四种命令分别书写函数f(x)=∑ni=11xi。
\documentclass[12pt]{article}
\begin{document}
$\displaystyle{f(x)=\sum_{i=1}^{n}\frac{1}{x_{i}}}$, 
$\textstyle{f(x)=\sum_{i=1}^{n}\frac{1}{x_{i}}}$,
$\scriptstyle{f(x)=\sum_{i=1}^{n}\frac{1}{x_{i}}}$,
$\scriptscriptstyle{f(x)=\sum_{i=1}^{n}\frac{1}{x_{i}}}$.
\end{document}
\end{lstlisting}
\vspace{-0.5cm}
\setlength{\parindent}{2em}
在各类公式环境(如equation、align、gather)中,可以外使用一系列字符大小命令进行调整,例如用\textbackslash begingroup \textbackslash endgroup限定字符区域。
%% Small size
\begingroup
\small
\begin{align}
    x+y=2 \\
    2x+y=3
\end{align}
%% Large size
\Large
\begin{align}
    x+y=2 \\
    2x+y=3
\end{align}
\endgroup

\begin{lstlisting}
在\begingroup \endgroup中使用字符大小命令\small和\Large对公式大小进行调整。
\documentclass[12pt]{article}
\usepackage{amsmath}
\begin{document}
%% Small size
\begingroup
\small
\begin{align}
    x+y=2 \\
    2x+y=3
\end{align}
%% Large size
\Large
\begin{align}
    x+y=2 \\
    2x+y=3
\end{align}
\endgroup
\end{document}
\end{lstlisting}

\subsubsection{其他格式调整}
在equation、align等公式环境中,我们也可以通过数组array环境对数学公式进行对齐 \par
对齐的方式有l(左侧对齐)、c(居中对齐)和r(右侧对齐)\par
当公式过长时,还有一些工具包提供的环境可以让公式进行自动跨行,以工具包breqn为例,在使用时,用\textbackslash begin\{dmath\} \textbackslash end\{dmath\}即可
\setcounter{equation}{0}
\begin{equation}
    \left\{\begin{array}{l}
    x+y=2 \\
    2x+y=3
    \end{array}\right.
\end{equation}

\begin{align}
    \left\{\begin{array}{l}
    x+y=2 \\
    2x+y=3
    \end{array}\right.
\end{align}

\begin{lstlisting}
使用\begin{array} \end{array}编译公式,并让公式居中对齐。
\documentclass[12pt]{article}
\usepackage{amsmath, mathtools}

\begin{document}

\begin{equation}
\begin{array}{c@{\qquad}c}
A = B + C
\qquad\Rightarrow
& D = E - F, \\ \\
G = H
\qquad\Rightarrow
& K = P + Q + M.
\end{array}
\end{equation}

\end{document}    
\end{lstlisting}

\newpage
\subsection{练习题}
\begin{center}
    $\displaystyle{\max_{0\leq x \leq {n-1}} \prod _{i=1}^{c}f_{i}(x)}$
\end{center}

\begin{align}
(a+b)^2&=a^2+2ab+b^2  \nonumber \\
(a-b)^2&=a^2-2ab+b^2  \\
(a+b)(a-b)&=a^2-b^2    \nonumber 
\end{align}

\begin{flalign}
    &x+y=2& \\
    &2x+y=3
\end{flalign}



\begin{equation*}
    f(x)=\left \{
        \begin{array}{c l}
            x, & x>0, \\
           -x, & x<0.
        \end{array}\right.
\end{equation*}

\begin{subequations}
    \begin{equation}
    \label{eq-a}
    a = b
    \end{equation}
    \begin{equation}
    \label{eq-b}
    c = d
    \end{equation}
\end{subequations}
\newpage
\begin{lstlisting}
代码
$\displaystyle{\max_{0\leq x \leq {n-1}} \prod _{i=1}^{c}f_{i}(x)}$
\begin{align}
(a+b)^2&=a^2+2ab+b^2  \nonumber \\
(a-b)^2&=a^2-2ab+b^2  \\
(a+b)(a-b)&=a^2-b^2    \nonumber 
\end{align}

\begin{flalign}
&x+y=2& \\
&2x+y=3
\end{flalign}

\begin{equation*}
    f(x)=\left \{
        \begin{array}{c l}
            x, & x>0, \\
           -x, & x<0.
        \end{array}\right.
\end{equation*}

\begin{subequations}
    \begin{equation}
    \label{eq-a}
    a = b
    \end{equation}
    \begin{equation}
    \label{eq-b}
    c = d
    \end{equation}
\end{subequations}
\end{lstlisting}

\newpage
\section{常用运算符号}
常用数学符号包括运算符号、标记符号、各类括号、空心符号及一些特殊函数。
\subsection{运算符号}
在初等数学中,最基本的运算规则是加减乘除。
在\LaTeX 中,加法符号和减法符号就是+和-;而乘法符号有两种,
第一种是\textbackslash times,对应于符号$\times$,第二种是\textbackslash cdot,对应于符号$\cdot$,
除法符号的命令为\textbackslash div。\par
\vspace{0.5cm}
\fbox{
    \parbox{0.8\linewidth}{
        $$3+5$$        % 加法
        $$3-5$$        % 减法
        $$3\times 5$$  % 乘法
        $$3\cdot 5$$   % 乘法
        $$3\div 5$$    % 除法
        $$3/5$$        % 除法
    }
}
\\
\begin{lstlisting}
书写加减乘除3+5、3−5、3×5、3⋅5、3÷5和3/5。

\documentclass[12pt]{article}
\begin{document}

$$3+5$$        % 加法
$$3-5$$        % 减法
$$3\times 5$$  % 乘法
$$3\cdot 5$$   % 乘法
$$3\div 5$$    % 除法
$$3/5$$        % 除法
\end{document}
\end{lstlisting}

\vspace{0.5cm}
在加减的基础上,命令\textbackslash pm(由plus和minus首字母构成)和\textbackslash mp(由minus和plus首字母构成)分别对应着符号$\pm$和$\mp$。与加减乘除同样常用的运算符号还有大于号、小于号等
对于集合而言,还有一些基本运算符号如$\cap$(\textbackslash cap)、$\cup$(\textbackslash cup)、
$\supset$(\textbackslash supset)、$\subset$(\textbackslash subset)、
$\supseteq$(\textbackslash supseteq)、$\in$(\textbackslash in)。
除此之外,与“属于”命令\textbackslash in相反的“不属于”命令为\textbackslash notin,编译效果为$\notin$。

\fbox{
    \parbox{0.8\linewidth}{
        $$x<y$$      % 小于
        $$x>y$$      % 大于
        $$x\leq y$$  % 小于或等于
        $$x\geq y$$  % 大于或等于
        $$x\ll y$$   % 远小于
        $$x\gg y$$   % 远大于
    }
}
\\
\begin{lstlisting}
书写x<y、x>y、x≤y、x≥y、x≪y和x≫y。
\documentclass[12pt]{article}
\begin{document}
$$x<y$$      % 小于
$$x>y$$      % 大于
$$x\leq y$$  % 小于或等于
$$x\geq y$$  % 大于或等于
$$x\ll y$$   % 远小于
$$x\gg y$$   % 远大于
\end{document}
\end{lstlisting}

\subsection{标记符号}
在数学公式的编辑中,还有一些基本数学符号及表达式也非常重要,例如分式、上标、下标。
\LaTeX 中用于书写分数和分式的基本命令为\textbackslash frac\{分子\}\{分母\},
根据场景需要,也可以选用\textbackslash dfrac\{分子\}\{分母\}和\textbackslash tfrac\{分子\}\{分母\}。
\par
\vspace{0.5cm}
\fbox{
    \parbox{0.8\linewidth}{
        $$\frac{3}{5}$$      % 分数
        $$\frac{x}{y}$$      % 分式1
        $$\frac{x+3}{y+5}$$  % 分式2
    }
}
\newpage
\begin{lstlisting}
书写分数35与分式xy、x+3y+5。

\documentclass[12pt]{article}
\begin{document}

$$\frac{3}{5}$$      % 分数
$$\frac{x}{y}$$      % 分式1
$$\frac{x+3}{y+5}$$  % 分式2

\end{document}
\end{lstlisting}

\fbox{
    \parbox{0.8\linewidth}{
        $$x^{3}$$
        $$x^{5}$$
        $$x^{x+5}$$
        $$x^{x^{3}+5}$$
    }
}
\\
\begin{lstlisting}
书写带上标的x3、x5、xx+5和xx3+5。

\documentclass[12pt]{article}
\begin{document}

$$x^{3}$$
$$x^{5}$$
$$x^{x+5}$$
$$x^{x^{3}+5}$$

\end{document}
\end{lstlisting}

\fbox{
    \parbox{0.8\linewidth}{
        $$x_{3}$$
        $$x_{5}$$
        $$x_{x+5}$$
        $$x_{x_{3}+5}$$
        $$x_{1},x_{2},\ldots,x_{n}$$
    }
}
\newpage
\begin{lstlisting}
书写带下标的x3、x5、xx+5、xx3+5和x1,x2,…,xn。

\documentclass[12pt]{article}
\begin{document}

$$x_{3}$$
$$x_{5}$$
$$x_{x+5}$$
$$x_{x_{3}+5}$$
$$x_{1},x_{2},\ldots,x_{n}$$

\end{document}
\end{lstlisting}

\fbox{
    \parbox{0.8\linewidth}{
        $$\hat{x}$$
        $$\bar{x}$$
        $$\tilde{x}$$
        $$\vec{x}$$
        $$\dot{x}$$
    }
}
\\
\begin{lstlisting}
\documentclass[12pt]{article}
\begin{document}

$$\hat{x}$$
$$\bar{x}$$
$$\tilde{x}$$
$$\vec{x}$$
$$\dot{x}$$

\end{document}
\end{lstlisting}

根号同样是数学表达式中的常见符号,在LaTeX中,根号的命令为\textbackslash sqrt\{\},
使用默认设置,生成的表达式为二次方根,如果想要设置为三次方根,
则需要调整默认设置,即\textbackslash sqrt[3]\{\},以此类推,可以设置四次方根等。

\fbox{
    \parbox{0.8\linewidth}{
        $$\sqrt{3}$$
        $$\sqrt[3]{5}$$
        $$\sqrt{x+y}$$
        $$\sqrt{x^{3}+y^{5}}$$
        $$\sqrt{1+\sqrt{x}}$$
    }
}
\\
\begin{lstlisting}
\documentclass[12pt]{article}
\begin{document}

$$\sqrt{3}$$
$$\sqrt[3]{5}$$
$$\sqrt{x+y}$$
$$\sqrt{x^{3}+y^{5}}$$
$$\sqrt{1+\sqrt{x}}$$

\end{document}
\end{lstlisting}

\fbox{
    \parbox{0.8\linewidth}{
        $$\frac{\sqrt{x+3}}{\sqrt{y+5}}$$
    }
}
\\
\begin{lstlisting}
\documentclass[12pt]{article}
\begin{document}

$$\frac{\sqrt{x+3}}{\sqrt{y+5}}$$

\end{document}    
\end{lstlisting}

\subsection{各类括号}
在数学表达式中,括号的用处和种类都非常多,例如最常见的小括号、中括号、大括号(即花括号)。

\fbox{
    \parbox{0.8\linewidth}{
        $$x\left(\frac{1}{y}+1\right)$$
        $$x\left[\frac{1}{y}+1\right]$$
        $$x\left\{\frac{1}{y}+1\right\}$$
    }
}
\\
\begin{lstlisting}
\documentclass[12pt]{article}
\begin{document}

$$x\left(\frac{1}{y}+1\right)$$
$$x\left[\frac{1}{y}+1\right]$$
$$x\left\{\frac{1}{y}+1\right\}$$

\end{document}
\end{lstlisting}

有时候,由于公式过长等原因,我们也可以在需要分行处插入\textbackslash \textbackslash 将括号内的公式切分成多行。
\par
\vspace{0.5cm}
\fbox{
    \parbox{0.8\linewidth}{
        \begin{equation}
            \begin{aligned}
            \Bigl(a+\frac{b}{2}+\frac{c}{3}++\frac{d}{4}+\frac{e}{5}++\frac{f}{6}++\frac{g}{7}++\frac{h}{8} \\
                  +\frac{i}{9}+\frac{j}{10}+\frac{k}{11}++\frac{l}{12}+\frac{m}{13}++\frac{n}{14}++\frac{o}{15}+\cdots\Bigr)
            \end{aligned}
        \end{equation}
    }
}
\\
\begin{lstlisting}
\documentclass[12pt]{article}
\usepackage{amsmath}
\begin{document}
\begin{equation}
\begin{aligned}
\Bigl(a+\frac{b}{2}+\frac{c}{3}++\frac{d}{4}+\frac{e}{5}++\frac{f}{6}++\frac{g}{7}++\frac{h}{8} \\
    +\frac{i}{9}+\frac{j}{10}+\frac{k}{11}++\frac{l}{12}+\frac{m}{13}++\frac{n}{14}++\frac{o}{15}+\cdots\Bigr)
\end{aligned}
\end{equation}
\end{document}
\end{lstlisting}

在这里,我们可以使用一系列命令代替最常用的\textbackslash left和\textbackslash right组合,
如\textbackslash bigl和\textbackslash bigr组合、\textbackslash Bigl和\textbackslash Bigr组合、
\textbackslash biggl和\textbackslash biggr组合、\textbackslash Biggl和\textbackslash Biggr组合来控制括号大小。

\fbox{
    \parbox{0.8\linewidth}{
        \begin{equation}
        \left(x+y=z \right)
        \bigl(x+y=z \bigr)
        \Bigl(x+y=z \Bigr)
        \biggl(x+y=z \biggr)
        \Biggl(x+y=z \Biggr) \nonumber
        \end{equation}
    }
}
\\
\begin{lstlisting}
\documentclass[12pt]{article}
\usepackage{amsmath}
\begin{document}
\begin{equation}
\left(x+y=z \right)
\bigl(x+y=z \bigr)
\Bigl(x+y=z \Bigr)
\biggl(x+y=z \biggr)
\Biggl(x+y=z \Biggr)
\end{equation}
\end{document}
\end{lstlisting}

在数学公式编辑中,除了以上常见的括号,也有一些广义的“括号”。
\par
\vspace{0.5cm}
\fbox{
    \parbox{0.8\linewidth}{
            $$x\left|\frac{1}{y}+1\right|$$
            $$x\left\|\frac{1}{y}+1\right\|$$
            $$\left<\frac{1}{x},\frac{1}{y}\right>$$
            $$\langle\frac{1}{x},\frac{1}{y}\rangle$$
    }
}
\\
\begin{lstlisting}
\documentclass[12pt]{article}
\begin{document}
$$x\left|\frac{1}{y}+1\right|$$
$$x\left\|\frac{1}{y}+1\right\|$$
$$\left<\frac{1}{x},\frac{1}{y}\right>$$
$$\langle\frac{1}{x},\frac{1}{y}\rangle$$
\end{document}
\end{lstlisting}

\fbox{
    \parbox{0.8\linewidth}{
            $$\left.\frac{dy}{dx}\right|_{x=1}$$
    }
}
\newpage
\begin{lstlisting}
\documentclass[12pt]{article}
\begin{document}

$$\left.\frac{dy}{dx}\right|_{x=1}$$

\end{document}
\end{lstlisting}

\subsection{空心符号}
在数学表达式中,我们有时候会用到一些约定俗成的空心符号表示集合,这包括:
\begin{itemize}
    \item 空心R符号$\mathbb{R}$表示由所有实数构成的集合
    \item 空心Z符号$\mathbb{Z}$表示由所有整数构成的集合
    \item 空心N符号$\mathbb{N}$表示由所有非负整数构成的集合,如果要表示正整数,使用符号$\mathbb{N}_{+}$即可
    \item 空心C符号$\mathbb{C}$表示由所有复数构成的集合
\end{itemize}
\par
需要注意的是,要想让\LaTeX 成功编译出这些空心符号,我们需要调用特定的工具包,即\textbackslash usepackage\{amsfonts\},一般而言,为了保证公式的编译不出现意外,还会用到其他工具包,即\textbackslash usepackage\{amsfonts, amssymb, amsmath\}
\par
\vspace{0.5cm}
\fbox{
    \parbox{0.8\linewidth}{
        $$X\in\mathbb{R}^{m\times n}$$
    }
}
\\
\begin{lstlisting}
\documentclass[12pt]{article}
\usepackage{amsfonts}

\begin{document}
$$X\in\mathbb{R}^{m\times n}$$
\end{document}
\end{lstlisting}

\fbox{
    \parbox{0.8\linewidth}{
        $$\mathbb{1},\mathbb{2},\mathbb{3},\mathbb{4},\mathbb{5}$$
    }
}
\newpage
\begin{lstlisting}
使用工具包bbold中的\mathbb命令书写空心的1、2、3、4、5.
\documentclass[12pt]{article}
\usepackage{bbold}

\begin{document}

$$\mathbb{1},\mathbb{2},\mathbb{3},\mathbb{4},\mathbb{5}$$

\end{document}
\end{lstlisting}

\subsection{特殊函数}
\fbox{
    \parbox{0.8\linewidth}{
        $$y=\log_{2}x$$
        $$y=\ln x$$
    }
}
\\
\begin{lstlisting}
\documentclass[12pt]{article}
\begin{document}
$$y=\log_{2}x$$
$$y=\ln x$$
\end{document}
\end{lstlisting}

\fbox{
    \parbox{0.8\linewidth}{
        $$\sum_{i=1}^{n}x_{i}$$
        $$\prod_{i=1}^{n}x_{i}$$
    }
}
\\
\begin{lstlisting}
\documentclass[12pt]{article}
\begin{document}
$$\sum_{i=1}^{n}x_{i}$$
$$\prod_{i=1}^{n}x_{i}$$
\end{document}
\end{lstlisting}


\fbox{
    \parbox{0.8\linewidth}{
        $$y=\sin x$$
        $$y=\arcsin x$$
        $$y=\cos x$$
        $$y=\arccos x$$
        $$y=\tan x$$
        $$y=\arctan x$$
    }
}
\\
\begin{lstlisting}
\documentclass[12pt]{article}
\begin{document}

$$y=\sin x$$
$$y=\arcsin x$$
$$y=\cos x$$
$$y=\arccos x$$
$$y=\tan x$$
$$y=\arctan x$$

\end{document}
\end{lstlisting}

\section{希腊字母}
我们在初等数学中便已经学习到了一些常用的希腊字母,例如最常见的$\pi$(对应于\textbackslash pi),
圆周率$\pi$约等于3.14,圆的面积为$\pi r^{2}$、周长为$2\pi r$。
在几何学中,我们习惯用各种希腊字母表示度数,
如$\alpha$(对应于\textbackslash alpha)、$\beta$(对应于\textbackslash beta)、
$\theta$(对应于\textbackslash theta)、$\phi$对应于\textbackslash phi)、
$\psi$(对应于\textbackslash psi)、$\varphi$(对应于\textbackslash varphi),
使用希腊字母既方便,也容易区分于常用的x,y,z等其他变量。 \\
实际上,这些希腊字母也可以用来作为变量,在概率论与数理统计中常常出现的变量就包括:
\begin{itemize}
    \item 正态分布中的$\mu$(命令为\textbackslash mu)、$\sigma$(命令为\textbackslash sigma);
    \item 泊松分布中的$\lambda$(命令为\textbackslash lambda);
    \item 通常表示自由度的希腊字母为$\nu$(命令为\textbackslash nu)
\end{itemize}
另外,在不等式中经常用到的希腊字母有$\delta$(命令为\textbackslash delta)和$\epsilon$(命令为\textbackslash epsilon)。
除了这些,希腊字母还有$\gamma$(命令为\textbackslash gamma)、$\eta$(命令为\textbackslash eta)、
$\kappa$(命令为\textbackslash kappa)、$\rho$(命令为\textbackslash rho)、$\tau$(命令为\textbackslash tau)和$\omega$(命令为\textbackslash omega)。
当然,前面提到的这些希腊字母在用途上并没有严格的界定,很多时候,我们书写数学表达式时可以根据需要选用适当的希腊字母。
\par
\vspace{0.5cm}
\fbox{
    \parbox{0.8\linewidth}{
        $$S=\pi ab$$
    }
}
\\
\begin{lstlisting}
\documentclass[12pt]{article}
\begin{document}

$$S=\pi ab$$ % 椭圆面积公式

\end{document}
\end{lstlisting}

\fbox{
    \parbox{0.8\linewidth}{
        $$a^{\alpha}b^{\beta}\cdots k^{\kappa}l^{\lambda}\leq a\alpha+b\beta+\cdots+k\kappa+l\lambda$$
    }
}
\\
\begin{lstlisting}
\documentclass[12pt]{article}
\begin{document}

$$a^{\alpha}b^{\beta}\cdots k^{\kappa}l^{\lambda}\leq a\alpha+b\beta+\cdots+k\kappa+l\lambda$$

\end{document}
\end{lstlisting}

\fbox{
    \parbox{0.8\linewidth}{
        $$\phi\left(\frac{x_{1}+x_{2}+\cdots+x_{n}}{n}\right)\leq\frac{\phi\left(x_{1}\right)+\phi\left(x_{2}\right)+\cdots+\phi\left(x_{n}\right)}{n}$$
    }
}
\\
\begin{lstlisting}
\documentclass[12pt]{article}
\begin{document}
$$\phi\left(\frac{x_{1}+x_{2}+\cdots+x_{n}}{n}\right)\leq\frac{\phi\left(x_{1}\right)+\phi\left(x_{2}\right)+\cdots+\phi\left(x_{n}\right)}{n}$$
\end{document}
\end{lstlisting}

与英文字母类似的是,有些希腊字母不但有小写字母,还有大写字母。有时,根据需要,我们还会对需要斜体的大写字母,具体如下
\begin{itemize}
    \item 命令\textbackslash Gamma对应于希腊字母$\Gamma$,命令\textbackslash varGamma对应于$\varGamma$;
    \item 命令\textbackslash Delta对应于希腊字母$\Delta$,命令\textbackslash varDelta对应于$\varDelta$;
    \item 命令\textbackslash Theta对应于希腊字母$\Theta$,命令\textbackslash varTheta对应于$\varTheta$;
    \item 命令\textbackslash Lambda对应于希腊字母$\Lambda$,命令\textbackslash varLambda对应于$\varLambda$;
    \item 命令\textbackslash Pi对应于希腊字母$\Pi$,命令\textbackslash varPi对应于$\varPi$;
    \item 命令\textbackslash Sigma对应于希腊字母$\Sigma$,命令\textbackslash varSigma对应于$\varSigma$;
    \item 命令\textbackslash Phi对应于希腊字母$\Phi$,命令\textbackslash varPhi对应于$\varPhi$;
    \item 命令\textbackslash Omega对应于希腊字母$\Omega$,命令\textbackslash varOmega对应于$\varOmega$。
\end{itemize}
\par 

从这些大写希腊字母中可以看到:大写希腊字母的命令是将小写希腊字母的命令首字母进行大写,但这些大写希腊字母与小写希腊字母的区别却不仅仅是尺寸不同;当大写希腊字母作为变量时,可以采用斜体字。
\par
\vspace{0.5cm}
\fbox{
    \parbox{0.8\linewidth}{
        $$\Delta x+\Delta y$$
        $$(i,j,k)\in\Omega$$
    }
}
\\
\begin{lstlisting}
\documentclass[12pt]{article}
\begin{document}

$$\Delta x+\Delta y$$
$$(i,j,k)\in\Omega$$

\end{document}
\end{lstlisting}

\section{微积分}
事实上,数学公式的范畴极为广泛,我们所熟知的大学数学课程中,微积分、线性代数、概率论与数理统计中数学表达式的符号系统均大不相同。
本节将主要介绍如何使用\LaTeX 对微积分中的数学表达式进行书写和编译

\subsection{极限}
求极限是整个微积分中的基石,例如$\lim_{x\to 2}x^{2}$对应的\LaTeX 代码为\$\textbackslash lim\_\{x\textbackslash to 2\}x\^\{2\}\$

\fbox{
    \parbox{0.8\linewidth}{
        $$\lim_{x\to-\infty}\frac{3x^{2}-2}{3x-2x^{2}}=\lim_{x\to-\infty}\frac{x^{2}\left(3-\frac{2}{x^{2}}\right)}{x^{2}\left(\frac{3}{x}-2\right)}=\lim_{x\to-\infty}\frac{3-\frac{2}{x^{2}}}{\frac{3}{x}-2}=-\frac{3}{2}$$
    }
}
\\
\begin{lstlisting}
\documentclass[12pt]{article}
\begin{document}

$$\lim_{x\to-\infty}\frac{3x^{2}-2}{3x-2x^{2}}=\lim_{x\to-\infty}\frac{x^{2}\left(3-\frac{2}{x^{2}}\right)}{x^{2}\left(\frac{3}{x}-2\right)}=\lim_{x\to-\infty}\frac{3-\frac{2}{x^{2}}}{\frac{3}{x}-2}=-\frac{3}{2}$$

\end{document}
\end{lstlisting}

\fbox{
    \parbox{0.8\linewidth}{
        $\lim_{\Delta t\to0}\frac{s(t+\Delta t)+s(t)}{\Delta t}$ \& $\displaystyle{\lim_{\Delta t\to0}\frac{s(t+\Delta t)+s(t)}{\Delta t}}$
    }
}
\\
\begin{lstlisting}
\documentclass[12pt]{article}
\begin{document}

$\lim_{\Delta t\to0}\frac{s(t+\Delta t)+s(t)}{\Delta t}$ \& $\displaystyle{\lim_{\Delta t\to0}\frac{s(t+\Delta t)+s(t)}{\Delta t}}$

\end{document}
\end{lstlisting}


\subsection{导数}
在微积分中,给定函数f(x)后,我们能够将其导数定义为
$$f^\prime(a)=\lim_{x\to a}\frac{f(x)-f(a)}{x-a}$$
为了让分数的形式在直观上不显得过大,可以用$$f^\prime(a)=\lim\limits_{x\to a}\frac{f(x)-f(a)}{x-a}$$,其中,\textbackslash lim和\textbackslash limits两个命令需要配合使用。
需要注意的是,f\^\textbackslash prime(x)中的\textbackslash prime命令是标准写法,有时候也可以写作f'(x)。
\vspace{0.5cm}
\\
\fbox{
    \parbox{0.8\linewidth}{
        $$f^\prime(x)=\lim_{\Delta x\to 0}\frac{f(x+\Delta x)-f(x)}{\Delta x}$$
    }
}
\newpage
\begin{lstlisting}
\documentclass[12pt]{article}
\begin{document}

$$f^\prime(x)=\lim_{\Delta x\to 0}\frac{f(x+\Delta x)-f(x)}{\Delta x}$$

\end{document}
\end{lstlisting}

\fbox{
    \parbox{0.8\linewidth}{
        $$f^\prime(x)=15x^{4}+6x^{2}$$
    }
}
\\
\begin{lstlisting}
\documentclass[12pt]{article}
\begin{document}

$$f^\prime(x)=15x^{4}+6x^{2}$$

\end{document}
\end{lstlisting}

\fbox{
    \parbox{0.8\linewidth}{
        $$\frac{\mathrm{d}}{\mathrm{d}x}f(x)=15x^{4}+6x^{2}$$
        $$\frac{\mathrm{d}^{2}}{\mathrm{d}^{2}x}f(x)=60x^{3}+12x$$
    }
}
\\
\begin{lstlisting}
\documentclass[12pt]{article}
\begin{document}
$$\frac{\mathrm{d}}{\mathrm{d}x}f(x)=15x^{4}+6x^{2}$$
$$\frac{\mathrm{d}^{2}}{\mathrm{d}^{2}x}f(x)=60x^{3}+12x$$
\end{document}
\end{lstlisting}

\fbox{
    \parbox{0.8\linewidth}{
        $$\frac{\partial}{\partial x}f(x,y)=15x^{4}y^{2}+6x^{2}y$$
        $$\frac{\partial}{\partial y}f(x,y)=6x^{5}y+2x^{3}$$
    }
}
\\
\begin{lstlisting}
\documentclass[12pt]{article}
\begin{document}
$$\frac{\partial}{\partial x}f(x,y)=15x^{4}y^{2}+6x^{2}y$$
$$\frac{\partial}{\partial y}f(x,y)=6x^{5}y+2x^{3}$$
\end{document}
\end{lstlisting}

\fbox{
    \parbox{0.8\linewidth}{
        $$z=\mu\,\frac{\partial y}{\partial x}\bigg|_{x=0}$$
    }
}
\\
\begin{lstlisting}
\documentclass[12pt]{article}
\begin{document}

$$z=\mu\,\frac{\partial y}{\partial x}\bigg|_{x=0}$$

\end{document}
\end{lstlisting}

\subsection{积分}

积分的标准写法为$\int_{a}^{b}f(x)\,\mathrm{d}x$,代码为\textbackslash int\_\{a\}\^\{b\}f\(x\)\textbackslash,\textbackslash mathrm\{d\}x,其中,\textbackslash int表示积分,是英文单词integral的缩写形式,使用\textbackslash ,的目的是引入一个空格。
\par
\vspace{0.5cm}
\fbox{
    \parbox{0.8\linewidth}{
        $$\int\frac{\mathrm{d}x}{\sqrt{a^{2}+x^{2}}}=\frac{1}{a}\arcsin\left(\frac{x}{a}\right)+C$$
        $$\int\tan^{2}x\,\mathrm{d}x=\tan x-x+C$$
    }
}
\\
\begin{lstlisting}
\documentclass[12pt]{article}
\begin{document}

$$\int\frac{\mathrm{d}x}{\sqrt{a^{2}+x^{2}}}=\frac{1}{a}\arcsin\left(\frac{x}{a}\right)+C$$
$$\int\tan^{2}x\,\mathrm{d}x=\tan x-x+C$$

\end{document}
\end{lstlisting}

\fbox{
    \parbox{0.8\linewidth}{
        $$\int_{a}^{b}\left[\lambda_{1}f_{1}(x)+\lambda_{2}f_{2}(x)\right]\,\mathrm{d}x=\lambda_{1}\int_{a}^{b}f_{1}(x)\,\mathrm{d}x+\lambda_{2}\int_{a}^{b}f_{2}(x)\,\mathrm{d}x$$
        $$\int_{a}^{b}f(x)\,\mathrm{d}x=\int_{a}^{c}f(x)\,\mathrm{d}x+\int_{c}^{b}f(x)\,\mathrm{d}x$$        
    }
}
\newpage
\begin{lstlisting}
\documentclass[12pt]{article}
\begin{document}

$$\int_{a}^{b}\left[\lambda_{1}f_{1}(x)+\lambda_{2}f_{2}(x)\right]\,\mathrm{d}x=\lambda_{1}\int_{a}^{b}f_{1}(x)\,\mathrm{d}x+\lambda_{2}\int_{a}^{b}f_{2}(x)\,\mathrm{d}x$$
$$\int_{a}^{b}f(x)\,\mathrm{d}x=\int_{a}^{c}f(x)\,\mathrm{d}x+\int_{c}^{b}f(x)\,\mathrm{d}x$$

\end{document}
\end{lstlisting}

\fbox{
    \parbox{0.8\linewidth}{
        \begin{equation}
            \begin{aligned}
            V&=2\pi\int_{0}^{1} x\left\{1-(x-1)^{2}\right\}\,\mathrm{d}x \\
            &=2\pi\int_{0}^{2}\left\{-x^{3}+2 x^{2}\right\}\,\mathrm{d}x \\
            &=2\pi\left[-\frac{1}{4} x^{4}+\frac{2}{3} x^{3}\right]_{0}^{2} \\
            &=8\pi/3
            \end{aligned}
            \end{equation}
        }
}

\begin{lstlisting}
\documentclass[12pt]{article}
\begin{document}

\begin{equation}
\begin{aligned}
V&=2\pi\int_{0}^{1} x\left\{1-(x-1)^{2}\right\}\,\mathrm{d}x \\
&=2\pi\int_{0}^{2}\left\{-x^{3}+2 x^{2}\right\}\,\mathrm{d}x \\
&=2\pi\left[-\frac{1}{4} x^{4}+\frac{2}{3} x^{3}\right]_{0}^{2} \\
&=8\pi/3
\end{aligned}
\end{equation}

\end{document} 
\end{lstlisting}

上述介绍的都是一重积分,在微积分课程中,还有二重积分、三重积分等,对于一重积分,
\LaTeX 提供的基本命令为\textbackslash int,二重积分为\textbackslash iint,三重积分为\textbackslash iiint、四重积分为\textbackslash iiiint,当积分为五重或以上时,一般使用\textbackslash idotsint,即$\idotsint$

\fbox{
    \parbox{0.8\linewidth}{
        $$\iint\limits_{D}f(x,y)\,\mathrm{d}\sigma$$
        $$\iiint\limits_{\Omega}\left(x^{2}+y^{2}+z^{2}\right)\,\mathrm{d}v$$
    }
}
\\
\begin{lstlisting}
\documentclass[12pt]{article}
\begin{document}

$$\iint\limits_{D}f(x,y)\,\mathrm{d}\sigma$$
$$\iiint\limits_{\Omega}\left(x^{2}+y^{2}+z^{2}\right)\,\mathrm{d}v$$

\end{document}
\end{lstlisting}

在积分中,有一种特殊的积分符号是在标准的积分符号上加上一个圈,可用来表示计算曲线曲面积分,即$\oint_{C}f(x)\,\mathrm{d}x+g(y)\,\mathrm{d}y$,\\
代码为\textbackslash oint\_\{C\}f(x)\,\textbackslash mathrm\{d\}x+g(y)\textbackslash ,\textbackslash mathrm\{d\}y。

\subsection{练习题}
\fbox{
    \parbox{0.8\linewidth}{
        \begin{equation}
            \begin{aligned}
                f(x)=&\frac{f(x_{0})}{0!}+\frac{f^\prime(x_0)}{1!}(x-x_{0})^2 \\
                     &+\dots+\frac{f^(n)(x_{0})}{n!}(x-x_{0})^{n}+R_{n}(x) \nonumber
            \end{aligned}
        \end{equation}
    }
}
\\
\begin{lstlisting}
\documentclass[12pt]{article}
\begin{document}
\begin{equation}
    \begin{aligned}
        f(x)=&\frac{f(x_{0})}{0!}+\frac{f^\prime(x_0)}{1!}(x-x_{0})^2 \\
             &+\dots+\frac{f^(n)(x_{0})}{n!}(x-x_{0})^{n}+R_{n}(x) \nonumber
    \end{aligned}
\end{equation}
\end{document}
\end{lstlisting}

\section{线性代数}
\subsection{矩阵}
使用\LaTeX 时,我们可以用\textbackslash begin\{array\} \textbackslash end\{array\}环境来书写矩阵。

\fbox{
    \parbox{0.8\linewidth}{
        $$\left[\begin{array}{ccc} 1 & 2 & 3 \\ 4 & 5 & 6 \\ \end{array}\right]$$
        $$\left[\begin{array}{c|cc} 1 & 2 & 3 \\ \hline 4 & 5 & 6 \\ \end{array}\right]$$
    }
}
\\
\begin{lstlisting}
    \documentclass[12pt]{article}
    \begin{document}
    $$\left[\begin{array}{ccc} 1 & 2 & 3 \\ 4 & 5 & 6 \\ \end{array}\right]$$
    $$\left[\begin{array}{c|cc} 1 & 2 & 3 \\ \hline 4 & 5 & 6 \\ \end{array}\right]$$
    \end{document}
\end{lstlisting}

另外,除了\textbackslash begin\{array\} \textbackslash end\{array\},
我们还可以用\textbackslash begin\{matrix\} \textbackslash end\{matrix\}、
\textbackslash begin\{pmatrix\} \textbackslash end\{pmatrix\}、\textbackslash begin\{Vmatrix\}
 \textbackslash end\{Vmatrix\}等一系列环境来书写矩阵。
\par 
\vspace{0.5cm}
\fbox{
    \parbox{0.8\linewidth}{
        $$\begin{smallmatrix} 1 & 2 & 3 \\ 4 & 5 & 6 \\ \end{smallmatrix}$$
        $$\begin{matrix} 1 & 2 & 3 \\ 4 & 5 & 6 \\ \end{matrix}$$
        $$\begin{pmatrix} 1 & 2 & 3 \\ 4 & 5 & 6 \\ \end{pmatrix}$$
        $$\begin{bmatrix} 1 & 2 & 3 \\ 4 & 5 & 6 \\ \end{bmatrix}$$
        $$\begin{Bmatrix} 1 & 2 & 3 \\ 4 & 5 & 6 \\ \end{Bmatrix}$$
        $$\begin{vmatrix} 1 & 2 & 3 \\ 4 & 5 & 6 \\ \end{vmatrix}$$
        $$\begin{Vmatrix} 1 & 2 & 3 \\ 4 & 5 & 6 \\ \end{Vmatrix}$$
    }
}
\\
\begin{lstlisting}
\documentclass[12pt]{article}
\usepackage{mathtools}
\begin{document}
$$\begin{smallmatrix} 1 & 2 & 3 \\ 4 & 5 & 6 \\ \end{smallmatrix}$$
$$\begin{matrix} 1 & 2 & 3 \\ 4 & 5 & 6 \\ \end{matrix}$$
$$\begin{pmatrix} 1 & 2 & 3 \\ 4 & 5 & 6 \\ \end{pmatrix}$$
$$\begin{bmatrix} 1 & 2 & 3 \\ 4 & 5 & 6 \\ \end{bmatrix}$$
$$\begin{Bmatrix} 1 & 2 & 3 \\ 4 & 5 & 6 \\ \end{Bmatrix}$$
$$\begin{vmatrix} 1 & 2 & 3 \\ 4 & 5 & 6 \\ \end{vmatrix}$$
$$\begin{Vmatrix} 1 & 2 & 3 \\ 4 & 5 & 6 \\ \end{Vmatrix}$$
\end{document}
\end{lstlisting}

\fbox{
    \parbox{0.8\linewidth}{
        $$\mathbf{A}=\begin{bmatrix}
            a_{11} & a_{12} & \cdots & a_{1n} \\
            a_{21} & a_{22} & \cdots & a_{2n} \\
            \vdots & \vdots & \ddots & \vdots \\
            a_{m1} & a_{m2} & \cdots & a_{mn} 
            \end{bmatrix}$$
    }
}
\newpage
\begin{lstlisting}
\documentclass[12pt]{article}
\begin{document}
$$\mathbf{A}=\begin{bmatrix}
a_{11} & a_{12} & \cdots & a_{1n} \\
a_{21} & a_{22} & \cdots & a_{2n} \\
\vdots & \vdots & \ddots & \vdots \\
a_{m1} & a_{m2} & \cdots & a_{mn} 
\end{bmatrix}$$
\end{document}
\end{lstlisting}

当然,有时候,也可以用大写字母直接表示矩阵,用加粗的小写字母表示向量,用小写字母表示标量。
需要注意的是,张量作为矩阵的延伸,一般用$\mathcal{X}$或者加粗的$\boldsymbol{\mathcal{X}}$表示,对应的代码分别为\textbackslash mathcal\{X\}和\textbackslash boldsymbol\{\textbackslash mathcal\{X\}\}
\par 
\vspace{0.5cm}
\fbox{
    \parbox{0.8\linewidth}{
        $$\begin{bmatrix} 1 & 2 \\ 3 & 4 \\ 5 & 6\\ \end{bmatrix}
        \begin{bmatrix} 7 \\ 8 \\ \end{bmatrix}
        =\begin{bmatrix} 1\times7+2\times8 \\ 3\times7+4\times8 \\
        5\times7+6\times8 \\ \end{bmatrix}
        =7\begin{bmatrix} 1 \\ 3 \\ 5 \\ \end{bmatrix}
        +8\begin{bmatrix} 2 \\ 4 \\ 6 \\ \end{bmatrix}
        =\begin{bmatrix} 23 \\ 53 \\ 83 \\ \end{bmatrix}$$
    }
}
\\
\begin{lstlisting}
\documentclass[12pt]{article}
\usepackage{mathtools}
\begin{document}
$$\begin{bmatrix} 1 & 2 \\ 3 & 4 \\ 5 & 6\\ \end{bmatrix}
\begin{bmatrix} 7 \\ 8 \\ \end{bmatrix}
=\begin{bmatrix} 1\times7+2\times8 \\ 3\times7+4\times8 \\
5\times7+6\times8 \\ \end{bmatrix}
=7\begin{bmatrix} 1 \\ 3 \\ 5 \\ \end{bmatrix}
+8\begin{bmatrix} 2 \\ 4 \\ 6 \\ \end{bmatrix}
=\begin{bmatrix} 23 \\ 53 \\ 83 \\ \end{bmatrix}$$

\end{document}
\end{lstlisting}

\fbox{
    \parbox{0.8\linewidth}{
        $$\boldsymbol{y}:=\boldsymbol{y}+
        \left[\begin{array}{c|c|c} A_{1} & \cdots & A_{n} \end{array}\right]
        \left[\begin{array}{c} \boldsymbol{x}_{1} \\ \vdots \\ \boldsymbol{x}_{n} \\ \end{array}\right]
        =\boldsymbol{y}+\sum_{i=1}^{n}A_{i}\boldsymbol{x}_{i}$$
    }
}
\newpage
\begin{lstlisting}
\documentclass[12pt]{article}
\usepackage{mathtools}
\begin{document}
$$\boldsymbol{y}:=\boldsymbol{y}+
\left[\begin{array}{c|c|c} A_{1} & \cdots & A_{n} \end{array}\right]
\left[\begin{array}{c} \boldsymbol{x}_{1} \\ \vdots \\ \boldsymbol{x}_{n} \\ \end{array}\right]
=\boldsymbol{y}+\sum_{i=1}^{n}A_{i}\boldsymbol{x}_{i}$$  
\end{lstlisting}

\fbox{
    \parbox{0.8\linewidth}{
        $$\begin{bmatrix} 0 & 0 & 1 \\ 0 & 1 & 0 \\ 1 & 0 & 0 \\ \end{bmatrix}
        \begin{bmatrix} a & b & c \\ b & d & e \\ c& e & f \\ \end{bmatrix}
        =\begin{bmatrix} c & e & f \\ b & d & e \\ a & b & c \\ \end{bmatrix}$$
    }
}
\\
\begin{lstlisting}
\documentclass[12pt]{article}
\usepackage{mathtools}
\begin{document}
$$\begin{bmatrix} 0 & 0 & 1 \\ 0 & 1 & 0 \\ 1 & 0 & 0 \\ \end{bmatrix}
\begin{bmatrix} a & b & c \\ b & d & e \\ c& e & f \\ \end{bmatrix}
=\begin{bmatrix} c & e & f \\ b & d & e \\ a & b & c \\ \end{bmatrix}$$
\end{document}
\end{lstlisting}

\subsection{符号}
作用于矩阵的符号可分为标记符号和运算符号,就标记符号而言,
\begin{itemize}
    \item 矩阵的逆,如$\mathbf{A}^{-1}$,代码为\textbackslash mathbf\{A\}\^\{-1\};
    \item 矩阵的伪逆,写作$\mathbf{A}^{+}和\mathbf{A}^{\dagger}$,代码分别为\textbackslash mathbf\{A\}\^\{+\}和\textbackslash mathbf\{A\}\^\{\textbackslash dagger\};
    \item 矩阵的转置,写作$\mathbf{A}^{T}和\mathbf{A}^{\top}$,代码分别为\textbackslash mathbf\{A\}\^\{T\}和\textbackslash mathbf\{A\}\^\{\textbackslash top\};
    \item 酉矩阵的转置,写作$\mathbf{A}^{H}$,代码分别为\textbackslash mathbf\{A\}\^\{H\};
    \item 矩阵的秩,写作$\operatorname{rank}\left(\mathbf{A}\right)$,代码为\textbackslash operatorname\{rank\}\textbackslash left(\textbackslash mathbf\{A\}\textbackslash right);
    \item 矩阵的迹,写作$\operatorname{Tr}\left(\mathbf{A}\right)$,代码为\textbackslash operatorname\{Tr\}\textbackslash left(\textbackslash mathbf\{A\}\textbackslash right);
    \item 矩阵的行列式,写作$\det\left(\mathbf{A}\right)$,代码为\textbackslash det\textbackslash left(\textbackslash mathbf\{A\}\textbackslash right);
\end{itemize}

\fbox{
    \parbox{0.8\linewidth}{
        $$\left(\mathbf{A}\mathbf{B}\right)^{-1}=\mathbf{B}^{-1}\mathbf{A}^{-1}$$
        $$\left(\mathbf{A}+\mathbf{B}\right)^{\top}=\mathbf{A}^{\top}+\mathbf{B}^{\top}$$
        $$\left(\mathbf{A}+\mathbf{B}\right)^{H}=\mathbf{A}^{H}+\mathbf{B}^{H}$$
        $$\operatorname{Tr}\left(\mathbf{A}+\mathbf{B}\right)=
        \operatorname{Tr}\left(\mathbf{A}\right)+\operatorname{Tr}\left(\mathbf{B}\right)$$
        $$\det\left(\mathbf{A}\mathbf{B}\right)=\det\left(\mathbf{A}\right)\det\left(\mathbf{B}\right)$$        
    }
}
\\
\begin{lstlisting}
\documentclass[12pt]{article}
\usepackage{mathtools}
\begin{document}
$$\left(\mathbf{A}\mathbf{B}\right)^{-1}=\mathbf{B}^{-1}\mathbf{A}^{-1}$$
$$\left(\mathbf{A}+\mathbf{B}\right)^{\top}=\mathbf{A}^{\top}+\mathbf{B}^{\top}$$
$$\left(\mathbf{A}+\mathbf{B}\right)^{H}=\mathbf{A}^{H}+\mathbf{B}^{H}$$
$$\operatorname{Tr}\left(\mathbf{A}+\mathbf{B}\right)=
\operatorname{Tr}\left(\mathbf{A}\right)+\operatorname{Tr}\left(\mathbf{B}\right)$$
$$\det\left(\mathbf{A}\mathbf{B}\right)=\det\left(\mathbf{A}\right)\det\left(\mathbf{B}\right)$$
\end{document}
\end{lstlisting}

\fbox{
    \parbox{0.8\linewidth}{
        $$\frac{\partial}{\partial\mathbf{X}}\operatorname{Tr}
        \left(\mathbf{A}\mathbf{X}\mathbf{B}\right)
        =\mathbf{A}^{\top}\mathbf{B}^{\top}$$
    }
}
\\
\begin{lstlisting}
\documentclass[12pt]{article}
\usepackage{mathtools}
\begin{document}
$$\frac{\partial}{\partial\mathbf{X}}\operatorname{Tr}
\left(\mathbf{A}\mathbf{X}\mathbf{B}\right)
=\mathbf{A}^{\top}\mathbf{B}^{\top}$$
\end{document}
\end{lstlisting}

\fbox{
    \parbox{0.8\linewidth}{
        $$p\left(\mathbf{x}\right)=
        \frac{1}{\sqrt{\operatorname{det}\left(2\pi\mathbf{\Sigma}\right)}}
        \exp\left[-\frac{1}{2}\left(\mathbf{x}-\mathbf{\mu}\right)^{\top}
        \mathbf{\Sigma}^{-1}\left(\mathbf{x}-\mathbf{\mu}\right)\right]$$
    }
}
\newpage
\begin{lstlisting}
\documentclass[12pt]{article}
\begin{document}

$$p\left(\mathbf{x}\right)=
\frac{1}{\sqrt{\operatorname{det}\left(2\pi\mathbf{\Sigma}\right)}}
\exp\left[-\frac{1}{2}\left(\mathbf{x}-\mathbf{\mu}\right)^{\top}
\mathbf{\Sigma}^{-1}\left(\mathbf{x}-\mathbf{\mu}\right)\right]$$

\end{document}
\end{lstlisting}

\fbox{
    \parbox{0.8\linewidth}{
        $$p\left(\mathbf{x}\right)=\sum_{k=1}^{K}\rho_{k}\frac{1}
        {\sqrt{\operatorname{det}\left(2\pi\mathbf{\Sigma}_{k}\right)}}
        \exp\left[-\frac{1}{2}\left(\mathbf{x}-\mathbf{\mu}_{k}\right)^{\top}
        \mathbf{\Sigma}_{k}^{-1}\left(\mathbf{x}-\mathbf{\mu}_{k}\right)\right]$$
    }
}
\\
\begin{lstlisting}
\documentclass[12pt]{article}
\begin{document}
$$p\left(\mathbf{x}\right)=\sum_{k=1}^{K}\rho_{k}\frac{1}
{\sqrt{\operatorname{det}\left(2\pi\mathbf{\Sigma}_{k}\right)}}
\exp\left[-\frac{1}{2}\left(\mathbf{x}-\mathbf{\mu}_{k}\right)^{\top}
\mathbf{\Sigma}_{k}^{-1}\left(\mathbf{x}-\mathbf{\mu}_{k}\right)\right]$$
\end{document}
\end{lstlisting}

\fbox{
    \parbox{0.8\linewidth}{
        \begin{equation}
            \mathbf{X}\otimes\mathbf{Y}=\left[\begin{array}{cccc}
            x_{11}\mathbf{Y} & x_{12}\mathbf{Y} & \cdots & x_{1n}\mathbf{Y} \\
            x_{21}\mathbf{Y} & x_{22}\mathbf{Y} & \cdots & x_{2n}\mathbf{Y} \\
            \vdots & \vdots & \ddots & \vdots \\
            x_{m1}\mathbf{Y} & x_{m2}\mathbf{Y} & \cdots & x_{mn}\mathbf{Y}
            \end{array}\right]
        \end{equation}
    }
}
\\
\begin{lstlisting}
\documentclass[12pt]{article}
\begin{document}
\begin{equation}
\mathbf{X}\otimes\mathbf{Y}=\left[\begin{array}{cccc}
x_{11}\mathbf{Y} & x_{12}\mathbf{Y} & \cdots & x_{1n}\mathbf{Y} \\
x_{21}\mathbf{Y} & x_{22}\mathbf{Y} & \cdots & x_{2n}\mathbf{Y} \\
\vdots & \vdots & \ddots & \vdots \\
x_{m1}\mathbf{Y} & x_{m2}\mathbf{Y} & \cdots & x_{mn}\mathbf{Y}
\end{array}\right]
\end{equation}
\end{document}
\end{lstlisting}

\fbox{
    \parbox{0.8\linewidth}{
        \begin{equation}
            \begin{aligned}
            &\mathbf{A}\mathbf{X}+\mathbf{X}\mathbf{B}=\mathbf{C} \\
            \Rightarrow\quad&\operatorname{vec}\left(\mathbf{X}\right)=
            \left(\mathbf{I}\otimes\mathbf{A}+\mathbf{B}^{\top}\otimes\mathbf{I}\right)^{-1}
            \operatorname{vec}\left(\mathbf{C}\right)
            \end{aligned}
        \end{equation}
    }
}
\\
\begin{lstlisting}
\documentclass[12pt]{article}
\begin{document}
\begin{equation}
\begin{aligned}
&\mathbf{A}\mathbf{X}+\mathbf{X}\mathbf{B}=\mathbf{C} \\
\Rightarrow\quad&\operatorname{vec}\left(\mathbf{X}\right)=
\left(\mathbf{I}\otimes\mathbf{A}+\mathbf{B}^{\top}\otimes\mathbf{I}\right)^{-1}
\operatorname{vec}\left(\mathbf{C}\right)
\end{aligned}
\end{equation}
\end{document}
\end{lstlisting}

\subsection{范数}
\fbox{
    \parbox{0.8\linewidth}{
        $$\frac{1}{2}\mathbf{x}^{\top}\mathbf{A}^{\top}\mathbf{A}\mathbf{x}
        -\mathbf{x}^{\top}\mathbf{A}^{\top}\mathbf{b}
        =\frac{1}{2}\left\|\mathbf{A}\mathbf{x}-\mathbf{b}\right\|_{2}^{2}
        -\frac{1}{2}\left\|\mathbf{b}\right\|_{2}^{2}$$
    }
}
\\
\begin{lstlisting}
\documentclass[12pt]{article}
\begin{document}
$$\frac{1}{2}\mathbf{x}^{\top}\mathbf{A}^{\top}\mathbf{A}\mathbf{x}
-\mathbf{x}^{\top}\mathbf{A}^{\top}\mathbf{b}
=\frac{1}{2}\left\|\mathbf{A}\mathbf{x}-\mathbf{b}\right\|_{2}^{2}
-\frac{1}{2}\left\|\mathbf{b}\right\|_{2}^{2}$$
\end{document}
\end{lstlisting}

\fbox{
    \parbox{1\linewidth}{
        \begin{equation}
            \begin{aligned}
            &\left\|\left[\begin{array}{cc|cc} a_{11} & a_{12} & a_{13} & a_{14} \\
            a_{21} & a_{22} & a_{23} & a_{24} \\
            \hline a_{31} & a_{32} & a_{33} & a_{34} \\
            a_{41} & a_{42} & a_{43} & a_{44} \\
            \hline a_{51} & a_{52} & a_{53} & a_{54} \\
            a_{61} & a_{62} & a_{63} & a_{64} \\
            \end{array}\right]
            -\left[\begin{array}{cc} b_{11} & b_{12} \\
            b_{21} & b_{22} \\ b_{31} & b_{32} \\
            \end{array}\right]\otimes
            \left[\begin{array}{cc} c_{11} & c_{12} \\
            c_{21} & c_{22} \\ \end{array}\right]\right\|_{F} \\
            =&\left\|\left[\begin{array}{cc|cc}
            a_{11} & a_{12} & a_{13} & a_{14} \\
            a_{21} & a_{22} & a_{23} & a_{24} \\
            \hline a_{31} & a_{32} & a_{33} & a_{34} \\
            a_{41} & a_{42} & a_{43} & a_{44} \\
            \hline a_{51} & a_{52} & a_{53} & a_{54} \\
            a_{61} & a_{62} & a_{63} & a_{64} \\
            \end{array}\right]
            -\left[\begin{array}{c} b_{11} \\ b_{21} \\ b_{31} \\
            b_{12} \\ b_{22} \\ b_{32} \\ \end{array}\right]
            \otimes\left[\begin{array}{cccc}
            c_{11} & c_{21} & c_{12} & c_{22}  \nonumber\\
            \end{array}\right]\right\|_{F}
            \end{aligned}
            \end{equation}
    }
}
\\
\begin{lstlisting}
\documentclass[12pt]{article}
\begin{document}
\begin{equation}
\begin{aligned}
&\left\|\left[\begin{array}{cc|cc} a_{11} & a_{12} & a_{13} & a_{14} \\
a_{21} & a_{22} & a_{23} & a_{24} \\
\hline a_{31} & a_{32} & a_{33} & a_{34} \\
a_{41} & a_{42} & a_{43} & a_{44} \\
\hline a_{51} & a_{52} & a_{53} & a_{54} \\
a_{61} & a_{62} & a_{63} & a_{64} \\
\end{array}\right]
-\left[\begin{array}{cc} b_{11} & b_{12} \\
b_{21} & b_{22} \\ b_{31} & b_{32} \\
\end{array}\right]\otimes
\left[\begin{array}{cc} c_{11} & c_{12} \\
c_{21} & c_{22} \\ \end{array}\right]\right\|_{F} \\
=&\left\|\left[\begin{array}{cc|cc}
a_{11} & a_{12} & a_{13} & a_{14} \\
a_{21} & a_{22} & a_{23} & a_{24} \\
\hline a_{31} & a_{32} & a_{33} & a_{34} \\
a_{41} & a_{42} & a_{43} & a_{44} \\
\hline a_{51} & a_{52} & a_{53} & a_{54} \\
a_{61} & a_{62} & a_{63} & a_{64} \\
\end{array}\right]
-\left[\begin{array}{c} b_{11} \\ b_{21} \\ b_{31} \\
b_{12} \\ b_{22} \\ b_{32} \\ \end{array}\right]
\otimes\left[\begin{array}{cccc}
c_{11} & c_{21} & c_{12} & c_{22} \\
\end{array}\right]\right\|_{F}
\end{aligned}
\end{equation}
\end{document}
\end{lstlisting}

\subsection{练习题}
\begin{equation}
    \mathbf{a}=\vec{a}=\begin{bmatrix}
        a_{1} \\
        \vdots\\
        a_{n}
    \end{bmatrix}
\end{equation}

\begin{lstlisting}
\documentclass[12pt]{article}
\usepackage{amsmath}
\begin{document}
%% 提示:公式中的字符加粗使用\boldsymbol{}命令,带箭头的向量使用\vec{}或者\overrightarrow{}命令
\begin{equation}
    \begin{equation}
        \mathbf{a}=\vec{a}=\begin{bmatrix}
            a_{1} \\
            \vdots\\
            a_{n}
        \end{bmatrix}
    \end{equation}
\end{equation}
\end{document}
\end{lstlisting}

\newpage
\begin{equation}
    \nabla f(\mathbf{x})=\begin{pmatrix}
        \frac{\partial f(\mathbf{x})}{\partial x_{1}} \\
        \vdots \\
        \frac{\partial f(\mathbf{x})}{\partial x_{n}} \nonumber
    \end{pmatrix}
\end{equation}

\begin{lstlisting}
\documentclass[12pt]{article}
\usepackage{amssymb, amsfonts}
\begin{document}
%% 提示:梯度对应的命令为\nabla
\begin{equation}
    \begin{equation}
        \nabla f(\mathbf{x})=\begin{pmatrix}
            \frac{\partial f(\mathbf{x})}{\partial x_{1}} \\
            \vdots \\
            \frac{\partial f(\mathbf{x})}{\partial x_{n}} \nonumber
        \end{pmatrix}
    \end{equation}
\end{equation}
\end{document}
\end{lstlisting}

\section{概率论与数理统计}
概率论与数理统计是许多方向开展科学研究的基础,
不管是描述客观存在的数据,还是刻画变量之间的关联规则,凭借概率论知识都能得心应手。
概率论的数学公式也有其自身特点,本节主要介绍概率论与数理统计范畴内常用的数学公式在LaTeX中的写法。

\subsection{概率论基础}
概率论中有一个重要的准则叫做贝叶斯准则,贝叶斯准则的基础为贝叶斯公式,用来描述两个条件概率之间的关系。\\

\fbox{
    \parbox{0.8\linewidth}{
        $$p\left(\theta\mid y\right)
        =\frac{p\left(\theta,y\right)}{p\left(y\right)}
        =\frac{p\left(\theta\right)p\left(y\mid\theta\right)}{p\left(y\right)}$$
        $$p\left(\theta\mid y\right)\propto p\left(\theta\right)p\left(y\mid\theta\right)$$
    }
}
\newpage
\begin{lstlisting}
贝叶斯公式
\documentclass[12pt]{article}
\begin{document}
$$p\left(\theta\mid y\right)
=\frac{p\left(\theta,y\right)}{p\left(y\right)}
=\frac{p\left(\theta\right)p\left(y\mid\theta\right)}{p\left(y\right)}$$
$$p\left(\theta\mid y\right)\propto p\left(\theta\right)p\left(y\mid\theta\right)$$
\end{document}
\end{lstlisting}

\fbox{
    \parbox{0.8\linewidth}{
        $$\mathbb{E}\left(x\right)=\int xp\left(x\right)dx$$
        $$\mathbb{V}\left(x\right)=
        \int\left(x-\mathbb{E}\left(x\right)\right)^{2}p\left(x\right)dx$$
    }
}
\\
\begin{lstlisting}
期望方差
\documentclass[12pt]{article}
\begin{document}
$$\mathbb{E}\left(x\right)=\int xp\left(x\right)dx$$
$$\mathbb{V}\left(x\right)=
\int\left(x-\mathbb{E}\left(x\right)\right)^{2}p\left(x\right)dx$$
\end{document}
\end{lstlisting}

\fbox{
    \parbox{0.8\linewidth}{
        $$p\left(y\right)=\int p\left(y,\theta\right)\,d\theta
        =\int p\left(\theta\right)p\left(y\mid\theta\right)\,d\theta$$
    }
}
\\
\begin{lstlisting}
\documentclass[12pt]{article}
\begin{document}

$$p\left(y\right)=\int p\left(y,\theta\right)\,d\theta
=\int p\left(\theta\right)p\left(y\mid\theta\right)\,d\theta$$

\end{document}
\end{lstlisting}

\subsection{概率分布}
概率分布,是指用于表述随机变量取值的概率规律,是概率论中最常见的表达式之一

\fbox{
    \parbox{0.8\linewidth}{
        $$x\sim\mathcal{N}\left(\mu,\sigma^{2}\right)$$
        $$p\left(x\right)=\frac{1}{\sqrt{2\pi}\sigma}
        \exp\left(-\frac{1}{2\sigma^{2}}\left(x-\mu\right)^{2}\right)$$
    }
}
\\
\begin{lstlisting}
正态分布
\documentclass[12pt]{article}
\begin{document}
$$x\sim\mathcal{N}\left(\mu,\sigma^{2}\right)$$
$$p\left(x\right)=\frac{1}{\sqrt{2\pi}\sigma}
\exp\left(-\frac{1}{2\sigma^{2}}\left(x-\mu\right)^{2}\right)$$
\end{document}
\end{lstlisting}

\fbox{
    \parbox{0.8\linewidth}{
        \begin{equation}
        p(y)=\frac{\operatorname{Poisson}(y\mid\theta)
        \operatorname{Gamma}(\theta\mid\alpha,\beta)}
        {\operatorname{Gamma}(\theta\mid\alpha+y,1+\beta)}
        =\frac{\Gamma(\alpha+y)\beta^{\alpha}}
        {\Gamma(\alpha)y!(1+\beta)^{\alpha+y}}
        \end{equation}
    }
}
\\
\begin{lstlisting}
\documentclass[12pt]{article}
\begin{document}
\begin{equation}
p(y)=\frac{\operatorname{Poisson}(y\mid\theta)
\operatorname{Gamma}(\theta\mid\alpha,\beta)}
{\operatorname{Gamma}(\theta\mid\alpha+y,1+\beta)}
=\frac{\Gamma(\alpha+y)\beta^{\alpha}}
{\Gamma(\alpha)y!(1+\beta)^{\alpha+y}}
\end{equation}
\end{document}
\end{lstlisting}

\fbox{
    \parbox{0.8\linewidth}{
        $$\theta\mid y\sim\operatorname{Gamma}\left(\alpha+\sum_{i=1}^{n}y_{i},\beta+\sum_{i=1}^{n}x_{i}\right)$$
    }
}
\\
\begin{lstlisting}
\documentclass[12pt]{article}
\begin{document}
$$\theta\mid y\sim\operatorname{Gamma}\left(\alpha+\sum_{i=1}^{n}y_{i},\beta+\sum_{i=1}^{n}x_{i}\right)$$
\end{document}
\end{lstlisting}

\fbox{
    \parbox{0.8\linewidth}{
        $$\sigma^{2}\mid y\sim\operatorname{Inv}-\chi^{2}\left(n,s^{2}\right)$$
    }
}
\\
\begin{lstlisting}
\documentclass[12pt]{article}
\begin{document}
$$\sigma^{2}\mid y\sim\operatorname{Inv}-\Chi^{2}\left(n,s^{2}\right)$$
\end{document}
\end{lstlisting}

\fbox{
    \parbox{1\linewidth}{
        $$p\left(\beta\mid\mu_{1},\mu_{2},\tau_{1},\tau_{2},\rho\right)
        =\prod_{j=1}^{J}\mathcal{N}\left(\begin{pmatrix}\beta_{1j} \\ \beta_{2j} \end{pmatrix}
        \bigg|\begin{pmatrix} \mu_{1} \\ \mu_{2} \end{pmatrix},
        \begin{pmatrix} \tau_{1}^{2} & \rho\tau_{1}\tau_{2} \\
        \rho\tau_{1}\tau_{2} & \tau_{2}^{2} \end{pmatrix}\right)$$
    }
}
\\
\begin{lstlisting}
\documentclass[12pt]{article}
\begin{document}
$$p\left(\beta\mid\mu_{1},\mu_{2},\tau_{1},\tau_{2},\rho\right)
=\prod_{j=1}^{J}\mathcal{N}\left(\begin{pmatrix}\beta_{1j} \\ \beta_{2j} \end{pmatrix}
\bigg|\begin{pmatrix} \mu_{1} \\ \mu_{2} \end{pmatrix},
\begin{pmatrix} \tau_{1}^{2} & \rho\tau_{1}\tau_{2} \\
\rho\tau_{1}\tau_{2} & \tau_{2}^{2} \end{pmatrix}\right)$$
\end{document}
\end{lstlisting}

\fbox{
    \parbox{0.8\linewidth}{
        \begin{equation}
            \begin{aligned}
            y_{ij}&\sim\mathcal{N}\left(\alpha_{j}+x_{ij}\beta_{j},\sigma_{y}^{2}\right) \\
            \begin{pmatrix}\alpha \\ \beta \end{pmatrix}&
            \sim\mathcal{N}\left(\begin{pmatrix} \mu_{\alpha} \\
            \mu_{\beta} \end{pmatrix},
            \begin{pmatrix} \sigma_{\alpha}^{2} &
            \rho\sigma_{\alpha}\sigma_{\beta} \\
            \rho\sigma_{\alpha}\sigma_{\beta} &
            \sigma_{\beta}^{2} \end{pmatrix}\right)
            \end{aligned}
        \end{equation}
    }
}
\\
\begin{lstlisting}
\documentclass[12pt]{article}
\begin{document}
\begin{equation}
\begin{aligned}
y_{ij}&\sim\mathcal{N}\left(\alpha_{j}+x_{ij}\beta_{j},\sigma_{y}^{2}\right) \\
\begin{pmatrix}\alpha \\ \beta \end{pmatrix}&
\sim\mathcal{N}\left(\begin{pmatrix} \mu_{\alpha} \\
\mu_{\beta} \end{pmatrix},
\begin{pmatrix} \sigma_{\alpha}^{2} &
\rho\sigma_{\alpha}\sigma_{\beta} \\
\rho\sigma_{\alpha}\sigma_{\beta} &
\sigma_{\beta}^{2} \end{pmatrix}\right)
\end{aligned}
\end{equation}
\end{document}
\end{lstlisting}

\subsection{练习题}

\begin{equation}
    p\left(\alpha,\beta\mid y\right)\propto \left(\alpha,\beta\right)=\prod_{j=1}^{J} \frac{\Gamma\left(\alpha + \beta\right)}{\Gamma(\alpha)\Gamma(\beta)} \frac{\Gamma(\alpha + y_{i})+\Gamma(\beta+n_{j}-y_{j})}{\Gamma(\alpha+\beta+n_{j})} \nonumber
\end{equation} 

\begin{lstlisting}
\documentclass[12pt]{article}
\begin{document}
\begin{equation}
    p\left(\alpha,\beta\mid y\right)\propto \left(\alpha,\beta\right)=\prod_{j=1}^{J} \frac{\Gamma\left(\alpha + \beta\right)}{\Gamma(\alpha)\Gamma(\beta)} \frac{\Gamma(\alpha + y_{i})+\Gamma(\beta+n_{j}-y_{j})}{\Gamma(\alpha+\beta+n_{j})}
\end{equation}
\end{document}
\end{lstlisting}

\subsection{优化理论}
优化理论是科学研究的另一个重要方向,它的关键在于使用数学模型对现实问题建模,并在若干约束的条件下,求问题的最优解,
主要分为单目标优化、多目标优化两种类型。不管是单目标优化还是多目标优化,其数学表达式的主要内容均包括目标函数及约束条件两个主要部分。
\par 
\vspace{0.5cm}
\fbox{
    \parbox{0.8\linewidth}{
        \begin{equation}
            \begin{aligned}
                &\min_{x}\,f(x) \\
                &\text{s.t.}\,x\in\mathcal{X},\quad g_{j}\left(x\right)\leq 0,\,j=1,\ldots,r
            \end{aligned}
        \end{equation}
    }
}
\\
\begin{lstlisting}
\documentclass[12pt]{article}
\begin{document}
\begin{equation}
\begin{aligned}
&\min_{x}\,f(x) \\
&\text{s.t.}\,x\in\mathcal{X},\quad g_{j}\left(x\right)\leq 0,\,j=1,\ldots,r
\end{aligned}
\end{equation}
\end{document}
\end{lstlisting}

\fbox{
    \parbox{0.8\linewidth}{
        $$x^{*}\in\operatorname{arg}\min_{x\in\mathcal{X}}~L(x,\lambda^{*})$$
    }
}
\\
\begin{lstlisting}
\documentclass[12pt]{article}
\begin{document}
$$x^{*}\in\operatorname{arg}\min_{x\in\mathcal{X}}~L(x,\lambda^{*})$$
\end{document}
\end{lstlisting}


$$\operatorname{conv}\left(\{\boldsymbol{x}_{1},\cdots,\boldsymbol{x}_{m}\}\right)
=\left\{\sum_{i=1}^{m}\alpha_{i}\boldsymbol{x}_{i}
\bigm|\alpha_{i}\geq 0,\,i=1,\ldots,m,\,\sum_{i=1}^{m}\alpha_{i}=1\right\}$$ \nonumber
\\
\begin{lstlisting}
\documentclass[12pt]{article}
\usepackage{mathtools}
\begin{document}
$$\operatorname{conv}\left(\{\boldsymbol{x}_{1},\cdots,\boldsymbol{x}_{m}\}\right)
=\left\{\sum_{i=1}^{m}\alpha_{i}\boldsymbol{x}_{i}
\bigm|\alpha_{i}\geq 0,\,i=1,\ldots,m,\,\sum_{i=1}^{m}\alpha_{i}=1\right\}$$
\end{document}
\end{lstlisting}

\fbox{
    \parbox{0.8\linewidth}{
        \begin{equation}
            \begin{aligned}
            &f^{\star}(y)=\sup_{x\in\mathbb{R}^{n}}~\left\{xy-f(x)\right\},\quad y\in\mathbb{R}^{n} \\
            &f^{\star\star}(x)=\sup_{y\in\mathbb{R}^{n}}~\left\{xy-f^{\star}(y)\right\},\quad x\in\mathbb{R}^{n}
            \end{aligned}
        \end{equation}
    }
}
\\
\begin{lstlisting}
\documentclass[12pt]{article}
\usepackage{amssymb, amsfonts}
\begin{document}

\begin{equation}
\begin{aligned}
&f^{\star}(y)=\sup_{x\in\mathbb{R}^{n}}~\left\{xy-f(x)\right\},\quad y\in\mathbb{R}^{n} \\
&f^{\star\star}(x)=\sup_{y\in\mathbb{R}^{n}}~\left\{xy-f^{\star}(y)\right\},\quad x\in\mathbb{R}^{n}
\end{aligned}
\end{equation}   
\end{document}
\end{lstlisting}

\fbox{
    \parbox{0.8\linewidth}{
        \begin{equation}
            f(x)=\inf_{z\in\mathbb{R}^{m}}~F(x,z),\quad x\in\mathbb{R}^{n}
        \end{equation}
    }
}
\\
\begin{lstlisting}
documentclass[12pt]{article}
\usepackage{amssymb, amsfonts}
\begin{document}
\begin{equation}
f(x)=\inf_{z\in\mathbb{R}^{m}}~F(x,z),\quad x\in\mathbb{R}^{n}
\end{equation}
\end{document}
\end{lstlisting}

\fbox{
    \parbox{0.8\linewidth}{
        \begin{equation}
            p(u)=\inf_{x\in\mathcal{X}}\sup_{z\in\mathcal{Z}}~\left\{\phi(x,z)-u'z\right\},\quad u\in\mathbb{R}^{m}
    \end{equation}
    }
}
\\
\begin{lstlisting}
\documentclass[12pt]{article}
\usepackage{amssymb, amsfonts}
\begin{document}
\begin{equation}
p(u)=\inf_{x\in\mathcal{X}}\sup_{z\in\mathcal{Z}}~\left\{\phi(x,z)-u'z\right\},\quad u\in\mathbb{R}^{m}
\end{equation}
\end{document}
\end{lstlisting}
\newpage
\subsection{练习题}

\begin{equation}
    \begin{aligned}
        q(\mu)=& \inf_{u\in \mathbb{R}^{m}}\left\{p(u)+u'\mu \right\}  \\
               & \inf_{u\in \mathbb{R}^{m}} \inf_{x\in \mathcal{X}}\left\{p_{x}(u)+u'\mu \right\} \\
               & \inf_{x\in \mathcal{X}} \inf_{u\in \mathbb{R}^{m}}\left\{p_{x}(u)+u'\mu \right\} \\
               & \inf_{x\in \mathcal{X}} \left\{-p_{x}^{*}(-\mu) \right\} \nonumber
    \end{aligned}
\end{equation}
\\
\begin{lstlisting}
\begin{equation}
    \begin{aligned}
        q(\mu)=& \inf_{u\in \mathbb{R}^{m}}\left\{p(u)+u'\mu \right\}  \\
               & \inf_{u\in \mathbb{R}^{m}} \inf_{x\in \mathcal{X}}\left\{p_{x}(u)+u'\mu \right\} \\
               & \inf_{x\in \mathcal{X}} \inf_{u\in \mathbb{R}^{m}}\left\{p_{x}(u)+u'\mu \right\} \\
               & \inf_{x\in \mathcal{X}} \left\{-p_{x}^{*}(-\mu) \right\} \nonumber
    \end{aligned}
\end{equation}    
\end{lstlisting}
\newpage

\part{表格制作}
\setcounter{section}{0}
\section{基本介绍}
表格是展现数据的一种常用方式。\LaTeX 提供了多种表格环境用于制作各类表格,例如,\colorbox{lightgray}{tabular}、\colorbox{lightgray}{tabular*}、\colorbox{lightgray}{tabularx}、\colorbox{lightgray}{tabulary}、\colorbox{lightgray}{table}、\colorbox{lightgray}{longtable}等。
其中比较常用的方法是将tabular环境嵌入到table环境中,可以创建包含表格内容、表格标题、引用标签等属性的完整表格。
\subsection{\colorbox{lightgray}{tabular}环境:创建表格内容}
通过创建tabular环境可以定义表格内容、对齐方式、外观样式等,使用方式与前面章节中介绍的使用array环境制作数表(即矩阵)的方式类似。例如:

$$\left[\begin{array}{c|c|c} a & b & c \\ \hline d & e & f \\ \hline g & h & i \\ \end{array}\right]$$
\begin{lstlisting}
代码
$$\left[\begin{array}{c|c|c} a & b & c \\ \hline d & e & f \\ \hline g & h & i \\ \end{array}\right]$$
\end{lstlisting}

将上述代码中的array环境改写为tabular环境,得到如下所示的代码语句:
\begin{center}
    \begin{tabular} {c|c|c}
        a&b&c\\
        \hline 
        d&e&f\\
        \hline
        g&h&i
    \end{tabular}
\end{center}

这里制作出来的表格是文本模式下的,跟array环境制作的数学模式略有不同,不同之处在于,
array环境制作的数表是属于数学公式,而使用tabular环境制作得到的表格则属于文本内容,
但两者的用法及命令格式极其相似。在tabular环境下:
\begin{itemize}
    \item 在\textbackslash begin\{tabular\}命令后的\{\}内设置表格的列类型参数,包括:
        \begin{itemize}
            \item 设置每列的单元格对齐方式。对齐方式选项包括l、c和r,即left、center和right的首字母,
            分别对应左对齐、居中对齐和右对齐,每个字母对应一列;
            \item 创建表格列分隔线。表格列分隔线以|符号表示,|符号的个数表示列分隔线中线的个数,
            如|表示使用单线分隔列,||表示使用双线分隔列,以此类推。分割线符号可以设置在列对齐方式选项的左侧或右侧,分别表示创建列的左分隔线和右分隔线。
        \end{itemize}
    \item 使用\textbackslash \textbackslash 符号表示一行内容的结束;
    \item 使用\&符号划分行内的单元格;
    \item 使用\textbackslash hline命令创建行分隔线。
\end{itemize}
\begin{lstlisting}
\documentclass[12pt]{article}
\begin{document}
\begin{tabular}{c|c|c}
a & b & c \\
\hline
d & e & f \\
\hline
g & h & i \\
\end{tabular}
\end{document}
\end{lstlisting}

在tabular环境中,行分割线亦也可以通过\textbackslash usepackage\{booktabs\}调用booktabs宏包,
并分别使用\textbackslash toprule,\textbackslash midrule和\textbackslash bottomrule命令来添加不同粗细的横线。
其中,在调用booktabs的情况下,可以通过\textbackslash cmidrule[thickness]\{a-b\}来实现自定义横线,[thickness]控制横线的粗细,\{a-b\}指定横线需要横跨的列序号。

\begin{center}
\begin{tabular}{l|cccr}
    \hline
    & $x=1$ & $x=2$ & $x=3$ & $x=4$ \\
    \hline
    $y=x$ & 1 & 2 & 3 & 4 \\
    $y=x^2$ & 1 & 4 & 9 & 16 \\
    $y=x^3$ & 1 & 8 & 27 & 64 \\
    \hline      
\end{tabular}
\end{center}

\begin{lstlisting}
简单表格
\begin{center}
    \begin{tabular}{l|cccr}
        \hline
        & $x=1$ & $x=2$ & $x=3$ & $x=4$ \\
        \hline
        $y=x$ & 1 & 2 & 3 & 4 \\
        $y=x^2$ & 1 & 4 & 9 & 16 \\
        $y=x^3$ & 1 & 8 & 27 & 64 \\
        \hline      
    \end{tabular}
\end{center}        
\end{lstlisting}

\subsection{\colorbox{lightgray}{table}环境:自动编号与浮动表格}

使用table环境嵌套tabular环境,能够为创建的表格进行自动递增编号。此外,可以使用\textbackslash caption\{\}命令设置表格标题、
使用\textbackslash label\{\}命令为表格建立索引标签、使用\textbackslash centering命令将表格置于文档中间,如下所示:
\begin{lstlisting}
\begin{table}
    \centering
        \caption{Title of a table.}
        \label{Label of the table}
        \begin{tabular}
        % 表格内容
        \end{tabular}
\end{table}
\end{lstlisting}

事实上,在\textbackslash begin\{table\} \textbackslash end\{table\}环境中创建的表格属于浮动元素:
浮动元素(floating)是指不能跨页分割的元素,比如图片和表格。一般而言,浮动元素的显示位置未必是代码的位置,
比如,当页面空间不足时,LaTeX会根据内置的算法尝试将浮动元素放置到后面的页面中,避免出现内容跨页分割或者页面大量留白的情况,
从而创建更协调也更专业的文档。\\
通过在\textbackslash begin\{table\}[]的[]中设置位置控制参数,可以为浮动表格指定期望放置位置,各参数值及其含义如下:
\begin{itemize}
    \item h:英文单词here的首写字母,表示代码当前位置;
    \item t:英文单词top的首写字母,表示页面顶部位置;
    \item b:英文单词bottom的首写字母,表示页面底部位置;
    \item p:英文单词page的首写字母,表示后面的页面;
    \item !:!参数一般与其它位置参数配合使用,表示当空间足够时,强制将表格放在指定位置。如!h表示将表格强制放到当前页面,但当页面空间不足时,也可能被放置到后续页面中;
    \item H:表示将表格强制放在代码当前位置,具有比!h更严格的效果,使用时需要先在导言区使用\textbackslash usepackage\{float\}声明语句调用float宏包。
\end{itemize}

根据需要,浮动元素的位置控制参数一般可以设置为h、b、t、p、!和H的任意无序组合。该参数的缺省值为tbp,此时LaTeX会尝试将表格放在页面的顶端或者底端,否则会将表格放在下一页。
\begin{table}[h]
\centering
    \caption{Title of a table.}
    \label{first label}
    \begin{tabular}{l|cccr}
        \hline
        & $x=1$ & $x=2$ & $x=3$ & $x=4$ \\
        \hline
        $y=x$ & 1 & 2 & 3 & 4 \\
        $y=x^{2}$ & 1 & 4 & 9 & 16 \\
        $y=x^{3}$ & 1 & 8 & 27 & 64 \\
        \hline
    \end{tabular}
\end{table}

\begin{lstlisting}
表格嵌入到table环境中,创建了一个位置居中、有标题、索引、自动编号的表格。
\documentclass[12pt]{article}
\begin{document}
\begin{table}[h]
\centering
    \caption{Title of a table.}
    \label{first label}
    \begin{tabular}{l|cccr}
        \hline
        & $x=1$ & $x=2$ & $x=3$ & $x=4$ \\
        \hline
        $y=x$ & 1 & 2 & 3 & 4 \\
        $y=x^{2}$ & 1 & 4 & 9 & 16 \\
        $y=x^{3}$ & 1 & 8 & 27 & 64 \\
        \hline
    \end{tabular}
\end{table}
\end{document}
\end{lstlisting}

Table~\ref{table1} shows the values of some basic functions.
\begin{table}[htbp] % 设置位置参数
    \centering
    \caption{The values of some basic functions.}
    \label{table1}
    \begin{tabular}{l|cccr}
        \hline
        & $x=1$ & $x=2$ & $x=3$ & $x=4$ \\
        \hline
        $y=x$ & 1 & 2 & 3 & 4 \\
        $y=x^{2}$ & 1 & 4 & 9 & 16 \\
        $y=x^{3}$ & 1 & 8 & 27 & 64 \\
        \hline
    \end{tabular}
    
\end{table}
\begin{lstlisting}
在table环境中将表格的位置控制参数设置为htbp。
\documentclass[12pt]{article}
\begin{document}
Table~\ref{table1} shows the values of some basic functions.
\begin{table}[htbp] % 设置位置参数
    \centering
    \caption{The values of some basic functions.}
    \begin{tabular}{l|cccr}
        \hline
        & $x=1$ & $x=2$ & $x=3$ & $x=4$ \\
        \hline
        $y=x$ & 1 & 2 & 3 & 4 \\
        $y=x^{2}$ & 1 & 4 & 9 & 16 \\
        $y=x^{3}$ & 1 & 8 & 27 & 64 \\
        \hline
    \end{tabular}
    \label{table1}
\end{table}
\end{document}
\end{lstlisting}

\section{合并单元格}
如果需要合并单元格,首先应在导言区声明\textbackslash usepackage\{multirow\}以导入multirow宏包,
并使用\textbackslash multicolumn命令合并同行不同列的单元格、使用\textbackslash multirow命令合并同列不同行的单元格。
\subsection{合并不同列的单元格}
合并不同列的单元格时,应在tabular环境中使用\textbackslash multicolumn\{合并列数\}\{合并后的列类型参数\}\{单元格内容\}语句定义合并单元格。此时,合并后的单元格的列类型将由\textbackslash multicolumn给出,而非\textbackslash begin\{tabular\}中预设的列类型参数。
\par 
\vspace{0.5cm}
\begin{center}
\begin{tabular}{|l|l|l|l|}
    \hline
    Column1 & Column2 & Column3 & Column4 \\
    \hline
    \multicolumn{2}{|c|}{A1 and A2} & A3 & A4 \\
    \hline
    B1 & B2 & B3 & B4 \\
    \hline
    C1 & C2 & C3 & C4 \\
    \hline
\end{tabular}
\end{center}

\begin{lstlisting}
在tabular环境中使用\multicolumn命令合并不同列的单元格。
\documentclass[12pt]{article}
\usepackage{multirow}
\begin{document}
\begin{tabular}{|l|l|l|l|}
    \hline
    Column1 & Column2 & Column3 & Column4 \\
    \hline
    \multicolumn{2}{|c|}{A1 and A2} & A3 & A4 \\
    \hline
    B1 & B2 & B3 & B4 \\
    \hline
    C1 & C2 & C3 & C4 \\
    \hline
\end{tabular}
\end{document}
\end{lstlisting}
\subsection{合并不同行的单元格}
合并不同行的单元格时使用的语句为\textbackslash multirow\{合并行数\}\{合并后的宽度\}\{单元格内容\}。
如果把\{合并后的宽度\}参数设置为\{*\},那么LaTeX会根据文本内容自动设置单元格宽度。在绘制行分隔线时,使用\textbackslash hline命令会创建一条横跨表格左右两端的横线,显然不适用于合并单元格后的行。
此时应用\textbackslash cline\{起始列号-终止列号\}命令,通过指定行分隔线的起始列和终止列,从而定制跨越了部分列的行分隔线。\\
合并多行的单元格时,除了第一个单元格处使用\textbackslash multirow命令定义单元格,其余被合并的单元格处均留空。\par
\begin{center}
\begin{tabular}{|l|l|l|l|}
    \hline
    Column1 & Column2 & Column3 & Column4 \\
    \hline
    \multirow{2}{*}{A1 and B1} & A2 & A3 & A4 \\
    \cline{2-4} % 创建一条从第2列到第4列的行分隔线
    & B2 & B3 & B4 \\
    \hline
    C1 & C2 & C3 & C4 \\
    \hline
\end{tabular}
\end{center}

\begin{lstlisting}
在tabular环境中使用\multirow命令合并不同列的单元格,并使用\cline命令定制行分隔线的起始点。
\documentclass[12pt]{article}
\usepackage{multirow}
\begin{document}
\begin{tabular}{|l|l|l|l|}
    \hline
    Column1 & Column2 & Column3 & Column4 \\
    \hline
    \multirow{2}{*}{A1 and B1} & A2 & A3 & A4 \\
    \cline{2-4} % 创建一条从第2列到第4列的行分隔线
    & B2 & B3 & B4 \\
    \hline
    C1 & C2 & C3 & C4 \\
    \hline
\end{tabular}
\end{document}
\end{lstlisting}

\subsection{合并多行多列的单元格}
通过嵌套使用\textbackslash multicolumn和\textbackslash multirow命令可以实现对多行多列单元格的合并操作,
具体语句为\textbackslash multicolumn\{合并列数\}\{合并后的列类型参数\}\{\textbackslash multirow\{合并行数\}\{合并后的宽度\}\{单元格内容\}\}。
在同时合并涉及多行多列的单元格时,除了第一行使用\textbackslash multicolumn和\textbackslash multirow嵌套命令定义单元格,
其余被合并的行处均使用内容为空的\textbackslash multicolumn命令。\par
\begin{center}
\begin{tabular}{|l|l|l|l|}
    \hline
    Column1 & Column2 & Column3 & Column4 \\
    \hline
    \multicolumn{2}{|c|}{\multirow{2}{*}{A1, A2, B1 and B2}} & A3 & A4 \\ % 合并多行多列的单元格
    \cline{3-4} % 创建一条从第3列到第4列的行分隔线
    \multicolumn{2}{|c|}{} & B3 & B4 \\
    \hline
    C1 & C2 & C3 & C4 \\
    \hline
\end{tabular}
\end{center}

\begin{lstlisting}
在tabular环境中嵌套使用\multicolumn和\multirow命令合并多行多列的单元格。
\documentclass[12pt]{article}
\usepackage{multirow}
\begin{document}
\begin{tabular}{|l|l|l|l|}
    \hline
    Column1 & Column2 & Column3 & Column4 \\
    \hline
    \multicolumn{2}{|c|}{\multirow{2}{*}{A1, A2, B1 and B2}} & A3 & A4 \\ % 合并多行多列的单元格
    \cline{3-4} % 创建一条从第3列到第4列的行分隔线
    \multicolumn{2}{|c|}{} & B3 & B4 \\
    \hline
    C1 & C2 & C3 & C4 \\
    \hline
\end{tabular}
\end{document}
\end{lstlisting}

\section{插入斜线与标注}
\subsection{插入斜线}
在制作斜线表头或填充空白单元格时,经常需要用到斜线。在\LaTeX 中,
我们可以通过调用diagbox宏包及其提供的\textbackslash diagbox[参数]\{单元格内容1\}...\{单元格内容n\}命令
将一个单元格划分为n个部分(即插入(n-1)条斜线),并且可以在[]中设置不同参数,从而对斜线宽度、高度、方向等属性进行调整,主要包括:
\begin{itemize}
    \item width:设置斜线宽度;
    \item height:设置斜线高度;
    \item font:设置单元格字体大小和字体类型;
    \item linewidth:设置线宽;
    \item linecolor:设置线的颜色(需结合xcolor或其他宏包使用);
    \item dir:设置斜线方向,包括NW(默认)、NE、SW和SE,分别表示西北方向、东北方向、西南方向、东南方向。当仅插入一个斜线时,dir=NW与dir=SE、dir=NE与dir=SW效果相同,分别表示插入反斜线和斜线;当插入两个斜线时,如\textbackslash diagbox[设置dir参数]\{A\}\{B\}\{C\},使用NW、NE、SW和SE的效果如下图所示
\end{itemize}
\begin{center}
    \diagbox[dir=NW]{A}{B}{C} \qquad \diagbox[dir=NE]{A}{B}{C} \qquad \diagbox[dir=SW]{A}{B}{C} \qquad \diagbox[dir=SE]{A}{B}{C}
\end{center}

\begin{table}[htbp]
    \centering
    \caption{The value of some basic functions.}
    \label{table3}
    \begin{tabular}{l|cccc}
        \hline 
        \diagbox[dir=NW]{$y$}{$value$}{$x$}& $x=1$ & $x=2$ & $x=3$ & $x=4$ \\
        \hline
        $y=x$ & 1 & 2 & 3 & 4 \\
        $y=x^2$ & 1 & 4 & 9 & 16 \\
        $y=x^3$ & 1 & 8 & 27 & 64 \\
        \hline
    \end{tabular}
\end{table}

\begin{lstlisting}
使用\usepackage{diagbox}宏包中的\diagbox命令制作双斜线表头。
\documentclass[12pt]{article}
\usepackage{diagbox}
\begin{document}
\begin{table}[htbp] % 设置位置参数
    \centering
    \caption{The values of some basic functions.}
    \begin{tabular}{l|cccr}
        \hline
        \diagbox[width=5em]{$y$}{value}{$x$} & $x=1$ & $x=2$ & $x=3$ & $x=4$ \\
        \hline
        $y=x$ & 1 & 2 & 3 & 4 \\
        $y=x^{2}$ & 1 & 4 & 9 & 16 \\
        $y=x^{3}$ & 1 & 8 & 27 & 64 \\
        \hline
    \end{tabular}
    \label{table1}
\end{table}
\end{document}
\end{lstlisting}

\subsection{插入表注}
表格中的文本应当尽可能地保持简洁明了。因此,在保持简明的基础上,可以采用注释的方式以添加必要的细节对文本内容进行说明补充。
通常,在以表格为载体的内容中,为了保持表格内容的完整性和独立性,我们往往不采用脚注\textbackslash footnote\{\}的形式,
而是将注释添加在表格底部(称之为表注)。在\LaTeX 中添加表注的方式有多种,其中比较常用的一种是使用threeparttable宏包及其相关命令,
可以在表格底部生成与表格内容同宽的表注,并且当注释内容过长时可以实现自动换行,相比于其它方式更协调一致。 \par
具体是在tabular环境外嵌套一层threeparttable环境,并在tabular环境之后将表注内容添加在tablenotes环境中,
由此得到的表注将会显示在表格底部。如果需要将表格内容与表注建立关联关系,可以在表格内容的相应位置使用\textbackslash tnote\{索引标记\}添加表注的索引标记,
并且在tablenotes环境中使用item[索引标记]命令创建这项表注。


\begin{table}
    \centering
    \begin{threeparttable}
        \begin{tabular}{l|cccr}
            \toprule
            & $x=1$ & $x=2$ & $x=3$ & $x=4$ \\
            \midrule
            $y=x$ & 1\tnote{*} & 2 & 3 & 4 \\
            $y=x^{2}$ & 1 & 4 & 9 & 16 \\
            $y=x^{3}$ & 1 & 8 & 27 & 64 \\
            \bottomrule
        \end{tabular}
        \begin{tablenotes}
            \footnotesize
            \item[1] This is a remark example.
            \item[2] This is another remark example and with a very long content, but the contents will be wrapped.
            \item[*] This is 1.
        \end{tablenotes}
    \end{threeparttable}
\end{table}

\begin{lstlisting}
使用threeparttable宏包添加表注。
\documentclass[12pt]{article}
\usepackage{booktabs}
\usepackage{threeparttable}
\begin{document}
\begin{table}
    \centering
    \begin{threeparttable}
        \begin{tabular}{l|cccr}
            \toprule
            & $x=1$ & $x=2$ & $x=3$ & $x=4$ \\
            \midrule
            $y=x$ & 1\tnote{*} & 2 & 3 & 4 \\
            $y=x^{2}$ & 1 & 4 & 9 & 16 \\
            $y=x^{3}$ & 1 & 8 & 27 & 64 \\
            \bottomrule
        \end{tabular}
        \begin{tablenotes}
            \footnotesize
            \item[1] This is a remark example.
            \item[2] This is another remark example and with a very long content, but the contents will be wrapped.
            \item[*] This is 1.
        \end{tablenotes}
    \end{threeparttable}
\end{table}
\end{document}
\end{lstlisting}

\section{调整表格样式}
通过调用一些宏包及命令可以定制表格样式,从而创建更符合要求的表格。对表格样式的调整可以分为以下7个方面:表格尺寸、单元格自动对齐与换行、小数点对齐、行高、列宽、线宽、以及表格字体大小。

\subsection{表格尺寸}
如果想要修改表格尺寸,首先使用\textbackslash usepackage\{graphicx\}语句调用graphicx宏包,并使用\textbackslash resizebox\{宽度\}\{高度\}\{内容\}命令,
该命令以tabular环境构建的表格作为内容。为了避免产生不协调的尺寸,在设置参数时只需要设置\{宽度\}和\{高度\}中的其中一个即可,另一个以!作为参数,表示根据宽高比进行自动调整。
\par
This is the description for the following table. This is the description for the following table. This is the description for the following table. This is the description for the following table.
\begin{table}[h]
\centering
    \caption{Title of a table.}
    \label{label4}
    \resizebox{0.8\textwidth}{!}{
    \begin{tabular}{|l|l|l|l|}
        \hline
        Column1 & Column2 & Column3 & Column4 \\
        \hline
        A1 & A2 & A3 & A4 \\
        \hline
        B1 & B2 & B3 & B4 \\
        \hline
        C1 & C2 & C3 & C4 \\
        \hline
    \end{tabular}}
\end{table}  

\begin{lstlisting}
\documentclass[12pt]{article}
\usepackage{graphicx}
\begin{document}
This is the description for the following table. 
\begin{table}[h]
\centering
    \caption{Title of a table.}
    \label{first label}
    \resizebox{0.8\textwidth}{!}{
    \begin{tabular}{|l|l|l|l|}
        \hline
        Column1 & Column2 & Column3 & Column4 \\
        \hline
        A1 & A2 & A3 & A4 \\
        \hline
        B1 & B2 & B3 & B4 \\
        \hline
        C1 & C2 & C3 & C4 \\
        \hline
    \end{tabular}}
\end{table}   
\end{document}
\end{lstlisting}

\subsection{单元格自动对齐与换行}
使用列类型参数l、c或r可以对每列的单元格设置左对齐、横向居中对齐和右对齐,但由此创建的单元格不仅无法设置顶部对齐、
纵向居中对齐、以及底部对齐方式,而且单元格内容不论长短都被拉长为一行,显得不够灵活。下面介绍几种方式用于实现单元格自动对齐与换行。
\subsubsection{使用array宏包实现单元格自动对齐与换行}
首先在导言区使用\textbackslash usepackage\{array\}语句声明调用array宏包,该宏包提供了以下6个列类型参数分别对应不同的对齐方式:
\begin{itemize}
    \item 首先在导言区使用\textbackslash usepackage\{array\}语句声明调用array宏包,该宏包提供了以下6个列类型参数分别对应不同的对齐方式:
    \item m\{列宽\}:单元格内容将根据设置的列宽自动换行,并且对齐方式为纵向居中对齐
    \item b\{列宽\}:单元格内容将根据设置的列宽自动换行,并且对齐方式为底部对齐
    \item >\{\textbackslash raggedright\textbackslash arraybackslash\}:将一列的单元格内容设置为左对齐
    \item >\{\textbackslash centering\textbackslash arraybackslash\}:将一列的单元格内容设置为横向居中对齐
    \item >\{\textbackslash raggedleft\textbackslash arraybackslash\}:将一列的单元格内容设置为右对齐
\end{itemize}
默认情况下,如果单独使用p、m或b参数,默认为左对齐。我们可以对上述参数进行组合使用,从而获得不同的对齐效果。
需要注意的是,此时应使用\textbackslash tabularnewline取代\textbackslash \textbackslash 符号作为表格一行的结束。
\begin{table}[h]
    \centering
        \caption{Title of a table.}
        \label{label6}
        \begin{tabular}{|>{\raggedright\arraybackslash}m{2.3cm}|>{\centering\arraybackslash}m{2.3cm}|>{\centering}m{2.3cm}|>{\raggedleft\arraybackslash}m{2.3cm}|}
            \hline
            Column1 & Column2 Column2 & Column3 Column3 Column3 & Column4 Column4 Column4 Column4 \tabularnewline
            \hline
            Value1 & Value2 Value2 & Value3 Value3 Value3 & Value4 Value4 Value4 Value4 \tabularnewline
            \hline
            Value1 & Value2 Value2 & Value3 Value3 Value3 & Value4 Value4 Value4 Value4 \tabularnewline
            \hline
        \end{tabular}
\end{table}
通过调用array宏包的方式虽然可以实现自动换行,但常常需要经过反复试验才能获得想要的宽度,
更方便的方式是使用tabularx宏包或tabulary宏包及其相关命令自动计算列宽。对于涉及文本的表格,更推荐使用tabulary宏包。下面分别介绍通过这两个宏包及其命令如何实现自动换行
\begin{lstlisting}
\documentclass[12pt]{article}
\usepackage{array}
\begin{document}
\begin{table}[h]
\centering
    \caption{Title of a table.}
    \label{first label}
    \begin{tabular}{|>{\raggedright\arraybackslash}m{2.3cm}|>{\centering\arraybackslash}m{2.3cm}|>{\centering}m{2.3cm}|>{\raggedleft\arraybackslash}m{2.3cm}|}
        \hline
        Column1 & Column2 Column2 & Column3 Column3 Column3 & Column4 Column4 Column4 Column4 \tabularnewline
        \hline
        Value1 & Value2 Value2 & Value3 Value3 Value3 & Value4 Value4 Value4 Value4 \tabularnewline
        \hline
        Value1 & Value2 Value2 & Value3 Value3 Value3 & Value4 Value4 Value4 Value4 \tabularnewline
        \hline
    \end{tabular}
\end{table}
\end{document}
\end{lstlisting}

\subsubsection{使用tabularx宏包实现自动换行}
首先在导言区声明调用tabularx宏包,然后使用\textbackslash begin\{tabularx\} \textbackslash end\{tabularx\}环境取代
\textbackslash begin\{tabular\} \textbackslash end\{tabular\}环境创建表格内容,tabularx环境的使用方式与tabular类似,
不同之处主要在于:\textbackslash begin\{tabularx\}\{表格宽度\}\{列类型\}中应设置表格宽度;
在tabularx环境中,对于需要自动换行的列,其列类型应设置为大写的X。X参数可以与>\{\textbackslash raggedright\textbackslash arraybackslash\}、
>\{\textbackslash centering \textbackslash arraybackslash\}或>\{\textbackslash raggedleft\textbackslash arraybackslash\}进行组合使用,从而修改单元格的对齐方式。

\begin{table}[h]
    \centering
        \caption{Title of a table.}
        \label{first label}
        \begin{tabularx}{\linewidth}{|X|X|X|>{\centering\arraybackslash}X|} % 将需要自动换行的列的列类型参数设为X
            \hline
            Column1 & Column2 & Column3 & Column4 \\
            \hline
            This is Value1. This is Value1. & This is Value2. This is Value2. & This is Value3. This is Value3. & This is Value4. This is Value4. \\
            \hline
            This is Value1. This is Value1. This is Value1. & This is Value2. This is Value2. This is Value2. & This is Value3. This is Value3. This is Value3. & This is Value4. This is Value4. This is Value4. \\
            \hline
        \end{tabularx}
\end{table}

\begin{lstlisting}
调用tabularx宏包并设置列类型参数X从而实现单元格内容自动换行。
\documentclass[12pt]{article}
\usepackage{array}
\usepackage{tabularx} % 调用tabularx宏包
\begin{document}
\begin{table}[h]
\centering
    \caption{Title of a table.}
    \label{first label}
    \begin{tabularx}{\linewidth}{|X|X|X|>{\centering\arraybackslash}X|} % 将需要自动换行的列的列类型参数设为X
        \hline
        Column1 & Column2 & Column3 & Column4 \\
        \hline
        This is Value1. This is Value1. & This is Value2. This is Value2. & This is Value3. This is Value3. & This is Value4. This is Value4. \\
        \hline
        This is Value1. This is Value1. This is Value1. & This is Value2. This is Value2. This is Value2. & This is Value3. This is Value3. This is Value3. & This is Value4. This is Value4. This is Value4. \\
        \hline
    \end{tabularx}
\end{table}
\end{document}
\end{lstlisting}

\subsubsection{使用tabulary宏包实现自动换行}
类似地,调用tabulary宏包并使用\textbackslash begin\{tabulary\}\{表格宽度\}\{列类型\} \textbackslash end\{tabulary\}环境创建表格。对于需要自动换行的列,只需将列类型改为大写字母即可,即,大写L表示左对齐并自动换行、大写C表示居中对齐并自动换行、大写R表示右对齐并自动换行。
\begin{table}[h]
    \centering
        \caption{Title of a table.}
        \label{first label}
        \begin{tabulary}{\linewidth}{|L|C|C|R|} % 将需要自动换行的列的列类型参数改为大写
            \hline
            Column1 & Column2 & Column3 & Column4 \\
            \hline
            This is Value1. This is Value1. & This is Value2. This is Value2. & This is Value3. This is Value3. & This is Value4. This is Value4. \\
            \hline
            This is Value1. This is Value1. This is Value1. & This is Value2. This is Value2. This is Value2. & This is Value3. This is Value3. This is Value3. & This is Value4. This is Value4. This is Value4. \\
            \hline
        \end{tabulary}
    \end{table}
\begin{lstlisting}
调用tabulary宏包并设置大写列类型参数(L、C和R)从而实现单元格内容自动换行。
\documentclass[12pt]{article}
\usepackage{array}
\usepackage{tabulary} % 调用tabulary宏包
\begin{document}
\begin{table}[h]
\centering
    \caption{Title of a table.}
    \label{first label}
    \begin{tabulary}{\linewidth}{|L|C|C|R|} % 将需要自动换行的列的列类型参数改为大写
        \hline
        Column1 & Column2 & Column3 & Column4 \\
        \hline
        This is Value1. This is Value1. & This is Value2. This is Value2. & This is Value3. This is Value3. & This is Value4. This is Value4. \\
        \hline
        This is Value1. This is Value1. This is Value1. & This is Value2. This is Value2. This is Value2. & This is Value3. This is Value3. This is Value3. & This is Value4. This is Value4. This is Value4. \\
        \hline
    \end{tabulary}
\end{table}
\end{document}
\end{lstlisting}

\subsubsection{使用\textbackslash parbox命令实现人工换行}
我们也可以通过使用\textbackslash parbox命令对表格内容进行强制换行:

\begin{center}    
    \begin{tabular}{|c|c|c|c|}
        \hline
        a & b & c & d \\ 
        \hline
        a & b & c & \parbox[t]{5cm}{In probability theory and statistics, the continuous uniform distribution\\ or rectangular distribution is a family of symmetric probability distributions.} \\ 
        \hline
    \end{tabular} 
\end{center} 
\par
\begin{lstlisting}
用\parbox命令来实现单元格中文本强制换行。
\documentclass{article} 
\begin{document}
\begin{center}    
    \begin{tabular}{|c|c|c|c|}
        \hline
        a & b & c & d \\ 
        \hline
        a & b & c & \parbox[t]{5cm}{In probability theory and statistics, the continuous uniform distribution\\ or rectangular distribution is a family of symmetric probability distributions.} \\ 
        \hline
    \end{tabular} 
\end{center}
\end{document}
\end{lstlisting}

\subsection{小数点对齐}
为了更好地描述数据,在表格中常常将数据在小数点处进行对齐,在\LaTeX 中我们可以通过使用dcolumn包实现这一目的。
这个包提供了一个名为D的列类型,可以方便实现基于小数点的数字对齐以及基于其它符号的对齐,
使用方式为D\{输入符号\}\{输出符号\}\{符号后的数字位数\}。对于基于小数点的数字对齐,输入符号一般为“.”;
有时需要根据特定符号进行数字对齐,比如千分位逗号,这时输入符号即为“,”。例如,D\{.\}\{\textbackslash cdot\}\{2\}表示将某列的数据根据“.”符号对齐,输出时将该符号显示为点乘符号,并且显示2个小数位数。
\par
列类型D可以像其它列类型一样在表格环境的开始命令处直接进行设置,但会导致语句过长,
所以一般使用array宏包的\textbackslash newcolumntype命令定义一个新的列类型,并为这个列类型赋予一个比较短的名称以方便调用。
定义新的列类型的语句为\textbackslash newcolumntype\{新列类型名称\}[新列类型的参数个数]\{定义新列类型\},
例如:\textbackslash newcolumntype\{d\}[1]\{D\{.\}\{\textbackslash cdot\}\{\#1\}\}表示创建一个名为d的新列类型,该列类型的内容为D\{.\}\{\textbackslash cdot\}\{符号后的数字位数\},其中数字位数是传给d的参数。

\newcolumntype{d}[1]{D{.}{\cdot}{#1}}
\begin{center}
\begin{tabular}{|l|c|r|d{3}|}
    \hline
    Left & Center & Right & \mathrm{Decimal}\\
    \hline
    1.1 & 1.1 & 1.1 & 1.1\\
    \hline
    33.3 & 33.3 & 33.3 & 33.3\\
    \hline
    3.333 & 3.333 & 3.333 & 3.333\\
    \hline
\end{tabular}
\end{center}
\par
\begin{lstlisting}
调用dcolumn宏包和列类型D来实现表格数据的小数点对齐。
\documentclass[12pt]{article}
\usepackage{dcolumn}
\newcolumntype{d}[1]{D{.}{\cdot}{#1}}
\begin{document}
    \begin{tabular}{|l|c|r|d{3}|}
        \hline
        Left & Center & Right & \mathrm{Decimal}\\
        \hline
        1.1 & 1.1 & 1.1 & 1.1\\
        \hline
        33.3 & 33.3 & 33.3 & 33.3\\
        \hline
        3.333 & 3.333 & 3.333 & 3.333\\
        \hline
    \end{tabular}
\end{document}
\end{lstlisting}
\subsection{行高}
如果需要调整表格整体行高,可以在导言区使用\textbackslash renewcommand\{\textbackslash arraystretch\}\{行高倍数\}命令,
从而根据设置的行高倍数在默认值的基础上对行高进行扩大或缩小。\\
另一种调整行高的方式是通过在每行的结束标志\textbackslash \textbackslash 后加上行高增减量选项,即\textbackslash \textbackslash [行高增减量],从而在默认值的基础上对各行行高进行增减。
\\
\renewcommand{\arraystretch}{2}
\begin{center}
\begin{tabular}{|c|c|c|c|}
    \hline
    Column1 & Column2 & Column3 & Column4\\
    \hline
    A1 & A2 & A3 & A4\\
    \hline
    B1 & B2 & B3 & B4\\
    \hline
    C1 & C2 & C3 & C4\\
    \hline
\end{tabular}
\end{center}
\par
\begin{lstlisting}
使用\renewcommand{\arraystretch}{2}命令将表格整体行高设为两倍行距。
\documentclass[12pt]{article}
\renewcommand{\arraystretch}{2}
\begin{document}
\begin{tabular}{|c|c|c|c|}
    \hline
    Column1 & Column2 & Column3 & Column4\\
    \hline
    A1 & A2 & A3 & A4\\
    \hline
    B1 & B2 & B3 & B4\\
    \hline
    C1 & C2 & C3 & C4\\
    \hline
\end{tabular}
\end{document}
\end{lstlisting}

\begin{lstlisting}
使用\\[行高增减量]命令为表格各行设置不同的行高。
\documentclass[12pt]{article}
\begin{document}
\begin{tabular}{|c|c|c|c|}
    \hline
    Column1 & Column2 & Column3 & Column4\\[0cm]
    \hline
    A1 & A2 & A3 & A4\\[0.2cm]
    \hline
    B1 & B2 & B3 & B4\\[0.4cm]
    \hline
    C1 & C2 & C3 & C4\\[0.6cm]
    \hline
\end{tabular}
\end{document}
\end{lstlisting}
\subsection{列宽}
也可以在导言区使用\textbackslash setlength\{\textbackslash tabcolsep\}\{文本和列分隔线的间距\}命令修改表格列宽,默认情况下,单元格内容与列分隔线的间距为6pt。

\begin{lstlisting}
使用\setlength{\tabcolsep}{12pt}命令将表格单元格文本和列分隔线的间距设为12pt。
\documentclass[12pt]{article}
\setlength{\tabcolsep}{12pt}
\begin{document}
\begin{tabular}{|c|c|c|c|}
    \hline
    Column1 & Column2 & Column3 & Column4\\
    \hline
    A1 & A2 & A3 & A4\\
    \hline
    B1 & B2 & B3 & B4\\
    \hline
    C1 & C2 & C3 & C4\\
    \hline
\end{tabular}
\end{document}
\end{lstlisting}

\subsection{线宽}
通过在导言区使用\textbackslash setlength\{\textbackslash arrayrulewidth\}\{线宽\}命令,可以修改表格线宽,默认为0.4pt。
然而当线宽设置过大时,可能导致表格线交叉处不连续的情况。对此,在导言区调用xcolor宏包、并设置table选项可以解决。
\\
\begin{lstlisting}
在导言区使用\usepackage[table]{xcolor}命令调用设置了table选项的xcolor宏包,并使用\setlength{\arrayrulewidth}{线宽}命令设置表格线宽。
\documentclass[12pt]{article}
\usepackage[table]{xcolor} % 调用设置了table选项的xcolor宏包
\setlength{\arrayrulewidth}{2pt} % 修改表格线宽
\begin{document}
\begin{tabular}{|l|l|l|l|}
    \hline
    Column1 & Column2 & Column3 & Column4\\
    \hline
    A1 & A2 & A3 & A4\\
    \hline
    B1 & B2 & B3 & B4\\
    \hline
    C1 & C2 & C3 & C4\\
    \hline
\end{tabular}
\end{document}
\end{lstlisting}

\subsection{表格字体大小}
在文本编辑中我们知道,调整字体大小的方式既有全局方式也有局部方式,其中,全局方式是通过在文档类型中指定字体大小,
例如\textbackslash documentclass[12pt]\{article\},而局部方式则是通过一系列设置字体大小的命令,
例如\textbackslash large、Large、huge、\textbackslash fontsize等,在全局字体大小的基础上作进一步的调整。类似地,在使用LaTeX创建表格时,我们也可以对表格字体大小做全局或局部调整。

% 正常字体大小
\begin{table}[htp]
    \centering
    \caption{正常字体}
    \begin{tabular}{l|cccr}
        \hline
        & $x=1$ & $x=2$ & $x=3$ & $x=4$ \\
        \hline
        $y=x$ & 1 & 2 & 3 & 4 \\
        $y=x^{2}$ & 1 & 4 & 9 & 16 \\
        $y=x^{3}$ & 1 & 8 & 27 & 64 \\
        \hline
    \end{tabular}
\end{table}

% Large字体大小
\begin{table}[htp]
    \Large
    \centering
    \caption{Large字体}
    \begin{tabular}{l|cccr}
        \hline
        & $x=1$ & $x=2$ & $x=3$ & $x=4$ \\
        \hline
        $y=x$ & 1 & 2 & 3 & 4 \\
        $y=x^{2}$ & 1 & 4 & 9 & 16 \\
        $y=x^{3}$ & 1 & 8 & 27 & 64 \\
        \hline
    \end{tabular}
\end{table}
\newpage
\begin{lstlisting}
使用\Large命令调整表格局部字体大小。
\documentclass[12pt]{article}
\begin{document}

% 正常字体大小
\begin{table}[htp]
    \centering
    \begin{tabular}{l|cccr}
        \hline
        & $x=1$ & $x=2$ & $x=3$ & $x=4$ \\
        \hline
        $y=x$ & 1 & 2 & 3 & 4 \\
        $y=x^{2}$ & 1 & 4 & 9 & 16 \\
        $y=x^{3}$ & 1 & 8 & 27 & 64 \\
        \hline
    \end{tabular}
\end{table}

% Large字体大小
\begin{table}[htp]
    \Large
    \centering
    \begin{tabular}{l|cccr}
        \hline
        & $x=1$ & $x=2$ & $x=3$ & $x=4$ \\
        \hline
        $y=x$ & 1 & 2 & 3 & 4 \\
        $y=x^{2}$ & 1 & 4 & 9 & 16 \\
        $y=x^{3}$ & 1 & 8 & 27 & 64 \\
        \hline
    \end{tabular}
\end{table}
\end{document}
\end{lstlisting}
\newpage
\begin{lstlisting}
使用\fontsize命令通过具体设置来调整表格局部字体大小。
\documentclass[12pt]{article}
\begin{document}
% 正常字体大小
\begin{table}[htp]
    \centering
    \begin{tabular}{l|cccr}
        \hline
        & $x=1$ & $x=2$ & $x=3$ & $x=4$ \\
        \hline
        $y=x$ & 1 & 2 & 3 & 4 \\
        $y=x^{2}$ & 1 & 4 & 9 & 16 \\
        $y=x^{3}$ & 1 & 8 & 27 & 64 \\
        \hline
    \end{tabular}
\end{table}
% 将字体大小设为18pt、行距设为24pt
\begin{table}[htp]
    \fontsize{18pt}{24pt}\selectfont
    \centering
    \begin{tabular}{l|cccr}
        \hline
        & $x=1$ & $x=2$ & $x=3$ & $x=4$ \\
        \hline
        $y=x$ & 1 & 2 & 3 & 4 \\
        $y=x^{2}$ & 1 & 4 & 9 & 16 \\
        $y=x^{3}$ & 1 & 8 & 27 & 64 \\
        \hline
    \end{tabular}
\end{table}
\end{document}
\end{lstlisting}

\subsection{文字环绕表格}
如果想要实现文字环绕表格效果,可以使用wrapfig宏包,并使用其提供的wraptable环境嵌套tabular环境创建表格,从而达到文字环绕表格的效果。

In descriptive statistics, a box plot or boxplot is a method for graphically depicting groups of numerical data through their quartiles. Box plots may also have lines extending from the boxes (whiskers) indicating variability outside the upper and lower quartiles, hence the terms box-and-whisker plot and box-and-whisker diagram. Outliers may be plotted as individual points. Box plots are non-parametric: they display variation in samples of a statistical population without making any assumptions of the underlying statistical distribution (though Tukey's boxplot assumes symmetry for the whiskers and normality for their length).

\begin{wraptable}{r}{8cm}
\centering

\begin{tabular}{lcccc}
    \hline
     & $x=1$ & $x=2$ & $x=3$ & $x=4$ \\
    \hline
    $y=x$ &\multicolumn{1}{c}{1}  & 2 & 3 & 4 \\
    $y=x^{2}$ & 1 & \multicolumn{1}{c}{4} & 9 & 16 \\
    $y=x^{3}$ & 1 & 8 & \multicolumn{1}{c}{27} & 64 \\
    \hline
\end{tabular}

\end{wraptable}
The spacings between the different parts of the box indicate the degree of dispersion (spread) and skewness in the data, and show outliers. In addition to the points themselves, they allow one to visually estimate various L-estimators, notably the interquartile range, midhinge, range, mid-range, and trimean. Box plots can be drawn either horizontally or vertically. Box plots received their name from the box in the middle, and from the plot that they are.

\vspace{0.5cm}
\begin{lstlisting}
使用wraptable环境嵌套tabular环境创建表格,实现文字环绕表格;并使用\begin{wraptable}{r}{8cm}将表格置于文字右侧,同时将表格和文字的距离设为8cm。
\documentclass[12pt]{article}
\usepackage{wrapfig}
\begin{document}
In descriptive statistics, a box plot or boxplot is a method for graphically depicting groups of numerical data through their quartiles. Box plots may also have lines extending from the boxes (whiskers) indicating variability outside the upper and lower quartiles, hence the terms box-and-whisker plot and box-and-whisker diagram. Outliers may be plotted as individual points. Box plots are non-parametric: they display variation in samples of a statistical population without making any assumptions of the underlying statistical distribution (though Tukey's boxplot assumes symmetry for the whiskers and normality for their length).
\begin{wraptable}{r}{8cm}
\centering
\begin{tabular}{lcccc}
    \hline
        & $x=1$ & $x=2$ & $x=3$ & $x=4$ \\
    \hline
    $y=x$ &\multicolumn{1}{c}{1}  & 2 & 3 & 4 \\
    $y=x^{2}$ & 1 & \multicolumn{1}{c}{4} & 9 & 16 \\
    $y=x^{3}$ & 1 & 8 & \multicolumn{1}{c}{27} & 64 \\
    \hline
\end{tabular}
\end{wraptable}
The spacings between the different parts of the box indicate the degree of dispersion (spread) and skewness in the data, and show outliers. In addition to the points themselves, they allow one to visually estimate various L-estimators, notably the interquartile range, midhinge, range, mid-range, and trimean. Box plots can be drawn either horizontally or vertically. Box plots received their name from the box in the middle, and from the plot that they are.
\end{document}
\end{lstlisting}

\section{创建彩色表格}
有时,根据表达需要,表格中的内容需要突出显示,彩色表格即为突出显示的一种重要方式。通过对表格的单元格、行或列填充颜色,
可以创建不同的彩色表格。为此,首先应在导言区使用\textbackslash usepackage[table]\{xcolor\}声明语句,通过调用xcolor宏包提供的相关命令可以实现颜色填充。
\par
填充单元格时,使用\textbackslash cellcolor\{单元格填充颜色\}单元格内容命令定义单元格内容即可

\begin{center}
\begin{tabular}{|l|l|l|l|}
    \hline
    Column1 & Column2 & Column3 & Column4\\
    \hline
    \cellcolor{red!80}A1 & A2 & A3 & A4\\ % 使用\cellcolor命令设置单元格填充颜色
    \hline
    \cellcolor{red!50}B1 & B2 & B3 & B4\\
    \hline
    \cellcolor{red!20}C1 & C2 & C3 & C4\\
    \hline
\end{tabular}
\end{center}

\begin{lstlisting}
在导言区使用\usepackage[table]{xcolor}命令调用设置了table选项的xcolor宏包,并使用\cellcolor命令定义具有颜色填充效果的单元格。
\documentclass[12pt]{article}
\usepackage[table]{xcolor} % 调用设置了table选项的xcolor宏包
\begin{document}
\begin{tabular}{|l|l|l|l|}
    \hline
    Column1 & Column2 & Column3 & Column4\\
    \hline
    \cellcolor{red!80}A1 & A2 & A3 & A4\\ % 使用\cellcolor命令设置单元格填充颜色
    \hline
    \cellcolor{red!50}B1 & B2 & B3 & B4\\
    \hline
    \cellcolor{red!20}C1 & C2 & C3 & C4\\
    \hline
\end{tabular}
\end{document}
\end{lstlisting}

为了达到更好的可视化效果,有时候需要为表格的奇数行和偶数行交替设置不同的填充颜色,
那么只需要在tabular环境前使用\textbackslash rowcolors\{开始填充的行编号\}\{第一个行填充颜色\}\{第二个行填充颜色\}命令即可

\begin{center}
\rowcolors{2}{red!50}{red!20} % 设置表格交替填充行颜色
\begin{tabular}{|l|l|l|l|}
    \hline
    Column1 & Column2 & Column3 & Column4\\
    \hline
    A1 & A2 & A3 & A4\\
    \hline
    B1 & B2 & B3 & B4\\
    \hline
    C1 & C2 & C3 & C4\\
    \hline
\end{tabular}
\end{center}

\begin{lstlisting}
\documentclass[12pt]{article}
\usepackage[table]{xcolor} % 调用设置了table选项的xcolor宏包
\begin{document}

\rowcolors{2}{red!50}{red!20} % 设置表格交替填充行颜色
\begin{tabular}{|l|l|l|l|}
    \hline
    Column1 & Column2 & Column3 & Column4\\
    \hline
    A1 & A2 & A3 & A4\\
    \hline
    B1 & B2 & B3 & B4\\
    \hline
    C1 & C2 & C3 & C4\\
    \hline
\end{tabular}

\end{document}
\end{lstlisting}
\newpage
\begin{lstlisting}
当然,我们也可以设置列填充颜色,只需要在列类型参数中加上>{\columncolor{列填充颜色}}即可:
\documentclass[12pt]{article}
\usepackage[table]{xcolor} % 调用设置了table选项的xcolor宏包
\begin{document}
\begin{tabular}{|>{\columncolor{red!50}}l|>{\columncolor{red!20}}l|>{\columncolor{red!50}}l|>{\columncolor{red!20}}l|} % 设置列填充颜色
    \hline
    Column1 & Column2 & Column3 & Column4\\
    \hline
    A1 & A2 & A3 & A4\\
    \hline
    B1 & B2 & B3 & B4\\
    \hline
    C1 & C2 & C3 & C4\\
    \hline
\end{tabular}
\end{document}
\end{lstlisting}

\section{创建三线表格}
booktabs宏包提供了更美观的行分隔线创建命令,常用于创建三线表格。其中,\textbackslash toprule命令常用于创建表格顶线、
\textbackslash bottomrule命令常用于创建表格底线、\textbackslash midrule命令常用于创建表格标题栏和表格内容的分隔线、
以及\textbackslash cmidrule\{起始列号-终止列号\}命令用于创建标题栏内部的分隔线并设置分隔线的跨越范围。 \\

\rowcolors{2}{white}{white}
\renewcommand{\arraystretch}{1}
\begin{center}
    \begin{tabular}{cccc}
        \toprule
        \multicolumn{2}{c}{\textbf{Type1}} & \\
        \cmidrule{1-2}
        Column1 & Column2 & Column3 & Column4\\
        \midrule
        A1 & A2 & A3 & A4\\
        B1 & B2 & B3 & B4\\
        C1 & C2 & C3 & C4\\
        \bottomrule
    \end{tabular}
\end{center}
\vspace{0.5cm}
\begin{lstlisting}
调用booktabs宏包及其相关命令创建三线表。
\documentclass[12pt]{article}
\usepackage{booktabs}
\usepackage{multirow}
\begin{document}
\begin{tabular}{cccc}
    \toprule
    \multicolumn{2}{c}{\textbf{Type1}} & \\
    \cmidrule{1-2}
    Column1 & Column2 & Column3 & Column4\\
    \midrule
    A1 & A2 & A3 & A4\\
    B1 & B2 & B3 & B4\\
    C1 & C2 & C3 & C4\\
    \bottomrule
\end{tabular}
\end{document}
\end{lstlisting}

\section{创建跨页表格}
当表格太长时,使用tabular环境创建的表格会被裁剪掉页面放不下的部分。如果想让表格在行数太多时实现自动分页,
可以通过调用longtable宏包并使用longtable环境创建表格。

在longtable环境中创建表格的方式与使用table和tabular嵌套环境类似,我们也能使用\textbackslash caption、\textbackslash label命令分别创建表格标题和索引标签。
不同之处主要在于,longtable环境中可以设置跨页表格在每一页的重复表头和表尾,按照使用顺序,包括以下四个命令:
\begin{itemize}
    \item \textbackslash endfirsthead:\textbackslash begin\{longtable\}和\textbackslash endfirsthead之间的内容只会出现在表格第一页的表头部分;
    \item \textbackslash endhead:\textbackslash endfirsthead和\textbackslash endhead之间的内容将会出现在表格除第一页之外的表头部分;
    \item \textbackslash endfoot:\textbackslash endhead和\textbackslash endfoot之间的内容将会出现在除表格最后一页之外的表尾部分;
    \item \textbackslash endlastfoot:\textbackslash endfoot和\textbackslash endlastfoot之间的内容只会出现在表格最后一页的表尾部分。
\end{itemize}
以上四个命令需要放置在longtable环境的开始处。
\begin{longtable}[c]{cccc}
    % 创建表格第一页的表头部分
    \caption{Title of a table}\\
    \hline
    Column1 & Column2 & Column3 & Column4\\
    \hline
    \endfirsthead
    % 创建表格除第一页之外的表头部分
    \caption{Title of a table - Continued}\\
    \hline
    Column1 & Column2 & Column3 & Column4\\
    \hline
    \endhead
    % 创建表格除最后一页之外的表尾部分
    \hline
    \endfoot
    % 创建表格最后一页的表尾部分
    \multicolumn{4}{c}{\textbf{End of table.}}\\
    \hline
    \endlastfoot
    % 表格内容
    A1 & A2 & A3 & A4\\
    B1 & B2 & B3 & B4\\
    C1 & C2 & C3 & C4\\
    A1 & A2 & A3 & A4\\
    B1 & B2 & B3 & B4\\
    C1 & C2 & C3 & C4\\
    C1 & C2 & C3 & C4\\
    A1 & A2 & A3 & A4\\
    B1 & B2 & B3 & B4\\
    C1 & C2 & C3 & C4\\
    C1 & C2 & C3 & C4\\
    A1 & A2 & A3 & A4\\
    B1 & B2 & B3 & B4\\
    C1 & C2 & C3 & C4\\
    C1 & C2 & C3 & C4\\
    A1 & A2 & A3 & A4\\
    B1 & B2 & B3 & B4\\
    C1 & C2 & C3 & C4\\
    B1 & B2 & B3 & B4\\
    C1 & C2 & C3 & C4\\
    C1 & C2 & C3 & C4\\
    A1 & A2 & A3 & A4\\
    B1 & B2 & B3 & B4\\
    C1 & C2 & C3 & C4\\
    C1 & C2 & C3 & C4\\
    A1 & A2 & A3 & A4\\
    B1 & B2 & B3 & B4\\
    C1 & C2 & C3 & C4\\
    C1 & C2 & C3 & C4\\
    A1 & A2 & A3 & A4\\
    B1 & B2 & B3 & B4\\
    C1 & C2 & C3 & C4\\
    A1 & A2 & A3 & A4\\
    B1 & B2 & B3 & B4\\
    C1 & C2 & C3 & C4\\
    \hline
\end{longtable}

\begin{lstlisting}
调用longtable宏包及环境创建跨页表格。
\documentclass[12pt]{article}
\usepackage{longtable}
\usepackage{multirow}
\begin{document}
\begin{longtable}[c]{cccc}
    % 创建表格第一页的表头部分
    \caption{Title of a table}\\
    \hline
    Column1 & Column2 & Column3 & Column4\\
    \hline
    \endfirsthead
    % 创建表格除第一页之外的表头部分
    \caption{Title of a table - Continued}\\
    \hline
    Column1 & Column2 & Column3 & Column4\\
    \hline
    \endhead
    % 创建表格除最后一页之外的表尾部分
    \hline
    \endfoot
    % 创建表格最后一页的表尾部分
    \multicolumn{4}{c}{\textbf{End of table.}}\\
    \hline
    \endlastfoot
    % 表格内容
    A1 & A2 & A3 & A4\\
    B1 & B2 & B3 & B4\\
    C1 & C2 & C3 & C4\\
    A1 & A2 & A3 & A4\\
    ... % 省略中间部分
    B1 & B2 & B3 & B4\\
    C1 & C2 & C3 & C4\\
    \hline
\end{longtable}
\end{document}
\end{lstlisting}

\section{旋转表格}
当表格列数太多时,横向表格的展现效果较差,这时需要将表格旋转90度,以纵向表格的形式展现。
在LaTeX中可以通过调用rotating宏包,并使用sidewaystable环境取代table环境、嵌套tabular环境创建纵向表格(表格逆时针旋转90度)。

\begin{sidewaystable}
    \centering
        \caption{Title of a table.}
        \label{label10}
        \begin{tabular}{cccc}
            \toprule
            Column1 & Column2 & Column3 & Column4\\
            \midrule
            A1 & A2 & A3 & A4\\
            B1 & B2 & B3 & B4\\
            C1 & C2 & C3 & C4\\
            \bottomrule
        \end{tabular}
\end{sidewaystable}

\begin{lstlisting}
调用rotating宏包及sidewaystable环境创建纵向表格。
\documentclass[12pt]{article}
\usepackage{rotating}
\usepackage{booktabs}
\begin{document}
\begin{sidewaystable}[h]
\centering
    \caption{Title of a table.}
    \label{first label}
    \begin{tabular}{cccc}
        \toprule
        Column1 & Column2 & Column3 & Column4\\
        \midrule
        A1 & A2 & A3 & A4\\
        B1 & B2 & B3 & B4\\
        C1 & C2 & C3 & C4\\
        \bottomrule
    \end{tabular}
\end{sidewaystable}
\end{document}
\end{lstlisting}
\newpage
\section{导入表格}
一个表格有时候可能涉及大量数据,这时在\LaTeX 中手动输入数据并创建表格显然不够灵活,对此,
可以通过导入csv文件数据的方式创建表格。\LaTeX 提供了csvsimple、pgfplotstable、csvtools等宏包可以帮助用户
实现基于csv文件快速导入表格,其中,使用csvsimple宏包及其命令是一种比较常用的方式,我们下面将对其展开介绍。

\subsection{快速创建表格}
首先在导言区使用\textbackslash usepackage\{csvsimple\}语句声明对csvsimple宏包的调用,
然后在文档主体内容中使用\textbackslash csvautotabular\{csv文件名或文件路径\}命令即可导入csv文件从而快速创建表格。
作为导入数据的csv文件,既可以预先放在指定目录下,也可以在filecontents环境中创建.
\par 
\vspace{0.1cm}
\begin{lstlisting}
通过上述语句,创建了一个名为“dataimport.csv”的csv文件。
\begin{filecontents}{dataimport.csv}
    COLUMNa,COLUMNb,COLUMNc,COLUMNd
    1.1,2.2,3.3,4.4
    11.1,22.2,33.3,44.4
    1.111,2.222,3.333,4.444
\end{filecontents} 
\end{lstlisting}

\begin{lstlisting}
使用filecontents环境创建一个名为“dataimport.csv”的csv文件,并调用csvsimple宏包及\csvautotabular命令快速根据csv文件创建表格。
\begin{filecontents}{dataimport.csv}
COLUMNa,COLUMNb,COLUMNc,COLUMNd
1.1,2.2,3.3,4.4
11.1,22.2,33.3,44.4
1.111,2.222,3.333,4.444
\end{filecontents}

\documentclass{article}
\usepackage{csvsimple}
\begin{document}
\begin{table}
\centering
    \caption{A table imported from csv file}
    \label{labeloftable1}
    \csvautotabular{dataimport.csv}
\end{table}
\end{document}
\end{lstlisting}

\subsection{创建三线表}
也可以结合booktabs宏包,使用\textbackslash csvautobooktabular命令自动读取csv文件并创建更专业的三线表格。
\begin{lstlisting}
使用filecontents环境创建一个名为“dataimport.csv”的csv文件,并调用csvsimple和booktabs宏包,使用\csvautobooktabular命令快速根据csv文件创建三线表格。
\begin{filecontents}{dataimport.csv}
COLUMNa,COLUMNb,COLUMNc,COLUMNd
1.1,2.2,3.3,4.4
11.1,22.2,33.3,44.4
1.111,2.222,3.333,4.444
\end{filecontents}

\documentclass{article}
\usepackage{csvsimple}
\usepackage{booktabs}
\begin{document}
\begin{table}
\centering
    \caption{A table imported from csv file}
    \label{labeloftable1}
    \csvautobooktabular{dataimport.csv}
\end{table}
\end{document}
\end{lstlisting}

\subsection{设置表格属性}
如果想要调整导入的表格样式、表头、指定导入列等属性,可以使用\textbackslash csvreader[属性设置]\{csv文件名或文件路径\}\{定义数据列名\}\{需要导入的数据列名\}命令读取csv文件创建表格,
并通过设置属性选项、指定需要导入的数据列名从而调整表格属性。在属性设置选项中,主要包括以下属性:
\begin{itemize}
    \item tabular:定义列类型。列类型个数应与需要导入的列数一致;
    \item table head:定义表头,包括标题行的顶线、列名、以及底线。由此可以对各列名进行重定义或省略;
    \item late after line:定义行分隔线。如,单行分隔线设置表示为late after line=\textbackslash \textbackslash \textbackslash hline。
\end{itemize}
\newpage
\begin{lstlisting}
使用csvreader命令读取csv文件创建表格,并将表格列名从“COLUMNa,COLUMNb,COLUMNc,COLUMNd”重命名为“column1,column2,column3,column4”,并分别设置数据列名“ca,cb,cc,cd”用于指定导入哪些列。
\begin{filecontents*}{dataimport.csv}
COLUMNa,COLUMNb,COLUMNc,COLUMNd
1.1,2.2,3.3,4.4
11.1,22.2,33.3,44.4
1.111,2.222,3.333,4.444
\end{filecontents*}

\documentclass{article}
\usepackage{csvsimple}
\begin{document}
\begin{table}
\centering
    \caption{A table imported from csv file}
    \label{labeloftable1}
    \csvreader[tabular=|l|l|l|l|,
    table head=\hline column1 & column2 & column3 & column4\\\hline,
    late after line=\\\hline] % 表格属性设置
    {dataimport.csv} % csv文件名
    {COLUMNa =\ca, COLUMNb =\cb, COLUMNc =\cc, COLUMNd =\cd} % 定义数据列名
    {\ca & \cb & \cc & \cd} % 需要导入的数据列名
\end{table}
\end{document}
\end{lstlisting}
\newpage
\begin{lstlisting}
使用csvreader命令读取csv文件创建表格,指定导入的数据列为前两列,并使用\thecsvrow命令增加行标签列。
\begin{filecontents}{dataimport.csv}
COLUMNa,COLUMNb,COLUMNc,COLUMNd
1.1,2.2,3.3,4.4
11.1,22.2,33.3,44.4
1.111,2.222,3.333,4.444
\end{filecontents}

\documentclass{article}
\usepackage{csvsimple}
\begin{document}
\begin{table}
\centering
    \caption{A table imported from csv file}
    \label{labeloftable1}
    \csvreader[tabular=|l|l|l|,
    table head=\hline & column1 & column2\\\hline,
    late after line=\\\hline] % 表格属性设置
    {dataimport.csv} % csv文件名
    {COLUMNa =\ca, COLUMNb =\cb, COLUMNc =\cc, COLUMNd =\cd} % 定义数据列名
    {\thecsvrow & \ca & \cb} % 需要导入的数据列名
\end{table}
\end{document}
\end{lstlisting}

\newpage
\part{图形插入}
\setcounter{section}{0}
\section{插入浮动图片}
LaTeX中可以支持插入.pdf、.jpg、.jpeg、.png、*.eps等常见格式的图片,而对于LaTeX不支持的图片文件格式,如SVG格式的矢量图,
则需要先转换再插入。一般而言,读者可以通过截图、MS Visio等绘图工具、或者Matlab等编程工具制作并导出目标图片。

在LaTeX中插入图片可以使用graphicx宏包,该宏包提供的\textbackslash includegraphics[参数]\{文件名或文件路径\}命令可以用于
插入图片,以及设置参数以调整图片样式,常用参数包括:
\begin{itemize}
    \item width:设置图片宽度;
    \item height:设置图片高;
    \item scale:设置图片的缩放倍数;
    \item angle:设置图片的顺时针旋转角度(负值表示逆时针旋转)等。
\end{itemize}
一般而言,对于参数height和width,只需要调整其中一个即可,另一个参数将根据图片比例进行自动缩放。
而如果同时调整了参数height和width(不推荐),可能会改变图片比例,导致图片变形。

\includegraphics[width=0.5\textwidth,scale=0.5]{LaTeX.png}

\vspace{12pt}

\includegraphics[width=0.5\textwidth,angle=90,scale=0.8]{LaTeX.png}

\begin{lstlisting}
在导言区使用\usepackage{graphicx}声明语句,在主体代码中使用\includegraphics命令插入图片,并调整图片样式参数。
\documentclass[12pt]{article}
\usepackage{graphicx}
\begin{document}
The following figure shows a beautiful butterfly.
\includegraphics[width = 0.5\textwidth]{butterfly.JPG} % 插入第一张图片
\vspace{12pt}
Rotate the figure by 90 degrees.
\includegraphics[width = 0.5\textwidth,angle = 90]{butterfly.JPG} 
\end{document}
\end{lstlisting}
\vspace{-1em}
此外,graphicx宏包提供了figure环境语句,通过嵌套\textbackslash includegraphics命令可以以浮动体的形式插入图片,
从而能够实现自动递增编号、设置位置控制参数、利用\textbackslash caption命令创建标题名称等。

Figure \ref{fig:1} shows a beautiful butterfly.

\begin{figure}[htbp]
\centering
    \includegraphics[width = 0.8\textwidth,scale=0.5]{LaTeX.png}
    \caption{A beautiful butterfly.}
    \label{fig:1}
\end{figure}

\begin{lstlisting}
使用figure环境嵌套\includegraphics命令插入浮动图片,并使用\label命令为图片创建索引标签,然后在文本内容中使用\ref命令引用该图片。
\documentclass[12pt]{article}
\usepackage{graphicx}
\begin{document}
Figure \ref{fig:1} shows a beautiful butterfly.
\begin{figure}[htbp]
\centering
    \label{fig:1}
    \includegraphics[width = 0.8\textwidth]{butterfly.JPG}
    \caption{A beautiful butterfly.}
\end{figure}
\end{document}
\end{lstlisting}

\section{插入非浮动图片}
通过figure环境插入图片虽然能够实现自动编号和创建图片标题,但创建结果为浮动图片,图片的显示位置与在代码中的位置未必一致。
然而有时我们想要以非浮动体的形式插入图片,使得图片显示位置与代码中的位置一致,同时能够实现自动编号和创建标题,要实现这一效果,
我们可以使用minipage环境或center环境替代figure环境插入图片,同时使用caption宏包提供的\textbackslash captionof\{figure\}\{图片标题名称\}命令
创建图片标题。

使用minipage环境插入图片的方式与figure环境类似,不同之处主要在于使用minipage环境插入的图片与上下文中的文本内容紧挨着,
为了避免这种情况,minipage环境前后可以使用\textbackslash vspace{纵向距离}调整图片与文本的纵向空间距离。

Figure \ref{fig:2} shows a beautiful butterfly.

\vspace{12pt}

\begin{minipage}{\linewidth}
\centering
    \includegraphics[width = 0.6\linewidth]{LaTeX.png}
    \captionof{figure}{A beautiful butterfly.}
    \label{fig:2}
\end{minipage}

\vspace{12pt}

\begin{lstlisting}
Figure \ref{fig:2} shows a beautiful butterfly.
\vspace{12pt}
\begin{minipage}{\linewidth}
\centering
    \label{fig:2}
    \includegraphics[width = 0.6\linewidth]{LaTeX.png}
    \captionof{figure}{A beautiful butterfly.}
\end{minipage}
\vspace{12pt}
\end{lstlisting}

在论文写作中,有时需要将多个图片放在同一行进行排列以便于比较。在figure环境中,使用minnipage环境即可实现图片并排显示,并连续编号。

\begin{figure}[htbp]
    \centering
    \begin{minipage}[t]{0.48\textwidth} 
    \centering
    \includegraphics[width=6cm]{LaTeX.png}
    \caption{LaTeX-1}
    \end{minipage}%
    \begin{minipage}[t]{0.48\textwidth}
    \centering
    \includegraphics[width=6cm]{LaTeX.png}
    \caption{LaTeX-2}
    \end{minipage}
\end{figure}

\begin{lstlisting}
\begin{figure}[htbp]
    \centering
    \begin{minipage}[t]{0.48\textwidth} 
    \centering
    \includegraphics[width=6cm]{LaTeX.png}
    \caption{LaTeX-1}
    \end{minipage}%
    \begin{minipage}[t]{0.48\textwidth}
    \centering
    \includegraphics[width=6cm]{LaTeX.png}
    \caption{LaTeX-2}
    \end{minipage}
\end{figure}    
\end{lstlisting}

\section{插入图表目录}
插入图目录或表目录的命令语句分别为\textbackslash listoffigures和\textbackslash listoftables,
可以罗列\textbackslash caption命令创建的图表标题名称,但对于使用\textbackslash caption*命令创建的无编号的标题名称而言,
则不会出现在目录中。

在一份专业文档中,目录总是和正文内容在不同页显示,为此可以使用\textbackslash newpage命令进行分页;
此外目录页中一般无页码编号,为此可以使用\textbackslash thispagestyle\{empty\}取消当页页码设置,
并在正文页之前使用\textbackslash pagenumbering\{页码样式\}命令表示重新从1开始设置页码、同时设置页码样式。

默认的图目录名和表目录名分别是“List of Figures”、“List of Tables”,读者可以在导言区通过使用
\textbackslash renewcommand\{\textbackslash listfigurename\}\{新图目录名\}命令修改图目录名、
使用\textbackslash renewcommand\{\textbackslash listtablename\}\{新表目录名\}命令修改表目录名。

\begin{lstlisting}
使用\listoffigures命令和\listoftables命令创建图目录和表目录,使用\renewcommand命令修改图目录名和表目录名,并使用\newpage、\thispagestyle和\pagenumbering命令对页码进行相应调整。
\documentclass{article}
\usepackage{graphicx}
\renewcommand{\listfigurename}{Figures}
\renewcommand{\listtablename}{Tables}
\begin{document}
\thispagestyle{empty} % 取消页码编号
\listoffigures
\listoftables
\newpage  % 插入新页
\pagenumbering{arabic}  % 设置页码样式为小写的阿拉伯数字
Here are three created tables...
\begin{table}[h!]
% ...
\caption{The first table.}
\caption{The second table.}
\caption{The third table.}
\end{table}
Here are three inserted figures...
\begin{figure}[h!]
% ...
\caption{The first figure.}
\caption*{The second figure.}
\caption{The third figure.}
% ...
\end{figure}
\end{document}
\end{lstlisting}

\section{定制图表标题样式}
使用\textbackslash caption命令为浮动图片或者浮动表格创建的标题主要包含四部分:1标题头部、2自动编号、3编号分隔符、
以及4标题名称,如下图所示。其中,对标题名称的设置比较简单,下面分别就标题前三部分的调整方式展开介绍。

\subsection{调整标题头部}
默认情况下,图片和表格标题头部分别为“Figure”和“Table”,我们可以分别使用\textbackslash renewcommand\{\textbackslash figurename\}\\ \{新的图片标题头部\}
和\textbackslash renewcommand\{\textbackslash tablename\}\{新的表格标题头部\}命令对其进行修改。

\renewcommand{\figurename}{Fig}
\begin{figure}[h]
    \centering
    \label{LaTeX}
    \includegraphics[width = 0.5\textwidth]{LaTeX.png}
    \caption{LaTeX}
\end{figure}

\begin{lstlisting}
使用\renewcommand命令将图表题头部改为Fig。
\documentclass[12pt]{article}
\usepackage{graphicx}
\renewcommand{\figurename}{Fig}
\begin{document}
\begin{figure}
\centering
\includegraphics[width = 0.5\textwidth]{graphics/butterfly.JPG}
\caption{There is a beautiful butterfly.}
\label{butterfly}
\end{figure}
\end{document}
\end{lstlisting}

\subsection{调整编号}
创建浮动表格或者浮动图片时,LaTeX会根据内置的计数器对其进行自动递增计数,计数值即为图表标题编号和引用编号,默认为小写的阿拉伯数字。

如果需要取消图表的自动编号,可以使用caption宏包提供的\textbackslash captionsetup[浮动体类型]\{labelformat=empty\}命令,
其中浮动体类型可以为figure、subfigure、table、或subtable,分别表示图片、子图、表格、子表。使用该命令后,指定浮动体类型下的所有浮动体将取消自动编号,
但其标题和编号仍会显示在图表目录中。

\begin{lstlisting}
使用\caption*命令取消部分表格的自动编号,同时使用\captionsetup命令取消所有图片的自动编号。
\documentclass{article}
\usepackage{graphicx}
\usepackage{caption}
\begin{document}
\captionsetup[figure]{labelformat=empty} % 取消所有图片的自动编号
Here are three created tables...
\begin{table}[h!]
% ...
\caption{The first table.}
\caption*{The second table.} % 取消该标题的自动编号
\caption{The third table.}
\end{table}
Here are three inserted figures...
\begin{figure}[h!]
% ...
\caption{The first figure.}
\caption{The second figure.}
\caption{The third figure.}
% ...
\end{figure}
\end{document}
\end{lstlisting}

如果想要修改图表编号样式,可以在导言区使用\textbackslash renewcommand\{浮动体的自动计数器\}\{计数器样式\}命令。
其中,浮动体的自动计数器名称可以为\textbackslash thefigure、\textbackslash thesubfigure、\textbackslash thetable、
或\textbackslash thesubtable;定义计数器样式可以使用\textbackslash alph\{浮动体类型\}、\textbackslash Alph\{浮动体类型\}、\textbackslash Roman\{浮动体类型\}、\textbackslash arabic\{浮动体类型\}等命令。

\begin{lstlisting}
使用\renewcommand命令调整图表标题头部和编号样式。
\documentclass{article}
\usepackage{graphicx}
\renewcommand{\figurename}{Fig} % 调整图片头部
\renewcommand{\tablename}{Tab} % 调整新表格头部
\renewcommand{\thefigure}{\Alph{figure}} % 调整图片编号样式为大写字母
\renewcommand{\thetable}{\alph{table}} % 调整表格编号样式为小写字母
\begin{document}
Here are three created tables...
\begin{table}[h!]
% ...
\caption{The first table.}
\caption{The second table.}
\caption{The third table.}
\end{table}
Here are three inserted figures...
\begin{figure}[h!]
% ...
\caption{The first figure.}
\caption{The second figure.}
\caption{The third figure.}
% ...
\end{figure}
\end{document}
\end{lstlisting}


\subsection{调整编号分隔符}
图表中的编号分隔符默认为英文冒号“:”,如果需要对其进行调整,可以使用caption宏包提供的\textbackslash captionsetup[浮动体类型]\{设置labelsep选项\}命令,通过设置不同的labelsep选项实现,各选项及其含义如下:
\begin{itemize}
    \item colon:默认值,即英文冒号“:”
    \item none:无编号分隔符
    \item period:英文句号“.”;
    \item space:一个空格
    \item quad:一个字符“M”大小的空格
    \item newline:换行
\end{itemize}

\begin{lstlisting}
使用caption宏包调整图标题编号分隔符为换行。
\documentclass[12pt]{article}

\usepackage{graphicx}
\usepackage{caption}
\renewcommand{\figurename}{Fig}
\captionsetup[figure]{labelsep=newline}

\begin{document}

\begin{figure}
\centering
\includegraphics[width = 0.5\textwidth]{graphics/butterfly.JPG}
\caption{There is a beautiful butterfly.}
\label{butterfly}
\end{figure}

\end{document}
\end{lstlisting}

\section{插入子图}
有时候需要将一组图片以子图的方式呈现,达到对比或者互补的效果。在LaTeX中,插入子图比较常用的方式是使用subfigure环境。

\subsection{基本介绍}
子图一般在subfigure环境中创建,多个子图环境嵌套在figure环境中从而形成同一组子图。
subfigure环境与figure环境的使用方式基本类似,可以为每个子图分别创建标题和索引标签,方便说明和引用。

Figure \ref{fig:fig1} contains sub-figure \ref{subfig:subfig1}, sub-figure \ref{subfig:subfig2} and sub-figure \ref{subfig:subfig3}.

\begin{figure}[h!]
\centering
    % 插入第一张子图
    \begin{subfigure}{.3\linewidth}
        \centering
        \includegraphics[width=.5\linewidth]{LaTeX.png}
        \caption{A red flower.}
        \label{subfig:subfig1}
    \end{subfigure}
    % 插入第二张子图
    \begin{subfigure}{.3\linewidth}
        \centering
        \includegraphics[width=.5\linewidth]{LaTeX.png}
        \caption{A yellow flower.}
        \label{subfig:subfig2}
    \end{subfigure}
    % 插入第三张子图
    \begin{subfigure}{.3\linewidth}
        \centering
        \includegraphics[width=.5\linewidth]{LaTeX.png}
        \caption{A blue flower.}
        \label{subfig:subfig3}
    \end{subfigure}
\caption{Three flowers with different colors.}
\label{fig:fig1}
\end{figure}

在上例中,每个子图都用到了两次宽度设置选项,分别具有不同含义:
\begin{enumerate}
    \item \textbackslash begin\{subfigure\}\{.3\textbackslash linewidth\}表示将该子图环境的宽度设置为页面宽度的0.3倍;
    \item \textbackslash includegraphics[width=.5\textbackslash linewidth]表示将该图片的宽度设置为当前子图环境宽度的0.5倍。
\end{enumerate}


\begin{lstlisting}
在导言区使用\usepackage{subcaption}语句,在代码主体区域使用figure环境嵌套三个subfigure环境从而创建三个子图,并为各子图分别创建索引和标题。
\documentclass[12pt]{article}
\usepackage{graphicx}
\usepackage{subcaption}
\begin{document}

Figure \ref{fig:fig1} contains sub-figure \ref{subfig:subfig1}, sub-figure \ref{subfig:subfig2} and sub-figure \ref{subfig:subfig3}.

\begin{figure}[h!]
\centering
    % 插入第一张子图
    \begin{subfigure}{.3\linewidth}
        \centering
        \includegraphics[width=.5\linewidth]{redflower.png}
        \caption{A red flower.}
        \label{subfig:subfig1}
    \end{subfigure}
    % 插入第二张子图
    \begin{subfigure}{.3\linewidth}
        \centering
        \includegraphics[width=.5\linewidth]{yellowFlower.png}
        \caption{A yellow flower.}
        \label{subfig:subfig2}
    \end{subfigure}
    % 插入第三张子图
    \begin{subfigure}{.3\linewidth}
        \centering
        \includegraphics[width=.5\linewidth]{blueFlower.png}
        \caption{A blue flower.}
        \label{subfig:subfig3}
    \end{subfigure}
\caption{Three flowers with different colors.}
\label{fig:fig1}
\end{figure}

\end{document}
\end{lstlisting}


\subsection{调整子图间距}
通过调整子图的横向和纵向间距,可以创建更协调美观的图片。

具体而言,存在以下三类命令可用于调整图片的横向间距:
\begin{itemize}
    \item \textbackslash hfill:对于位于相同行的子图,通过在相邻的subfigure环境间使用该命令,可以实现多个子图横向等距分布的效果;
    \item \textbackslash hspace{横向距离}:定制任意长度的横向图片距离。当该值设为负值时,可以产生图片重叠的效果;
    \item \textbackslash quad、\textbackslash qquad等:设置不同预设长度的横向图片距离。
\end{itemize}

\begin{lstlisting}
使用\hfill命令实现子图横向等距分布,以及使用\hspace{}命令实现子图重叠效果。
\documentclass[12pt]{article}
\usepackage{graphicx}
\usepackage{subcaption}
\usepackage{amssymb}
\usepackage{amsmath}
\begin{document}

% 使用\hfill命令调整子图横向间距

The horizontal space among Sub-figures in figure \ref{fig:fig1} is controlled by $\backslash$\textit{hfill}.

\begin{figure}[h!]
\centering
    % 插入第一张子图
    \begin{subfigure}{.3\linewidth}
        \centering
        \includegraphics[width=.5\linewidth]{redflower.png}
        \caption{A red flower.}
    \end{subfigure}
    \hfill
    % 插入第二张子图
    \begin{subfigure}{.3\linewidth}
        \centering
        \includegraphics[width=.5\linewidth]{yellowFlower.png}
        \caption{A yellow flower.}
    \end{subfigure}
    \hfill
    % 插入第三张子图
    \begin{subfigure}{.3\linewidth}
        \centering
        \includegraphics[width=.5\linewidth]{blueFlower.png}
        \caption{A blue flower.}
    \end{subfigure}
\caption{Three flowers with different colors.}
\label{fig:fig1}
\end{figure}

% 使用\space{}命令调整子图横向间距

The horizontal space among Sub-figures in figure \ref{fig:fig2} is controlled by $\backslash$\textit{space}.

\begin{figure}[h!]
\centering
    % 插入第一张子图
    \begin{subfigure}{.3\linewidth}
        \centering
        \includegraphics[width=.5\linewidth]{redflower.png}
    \end{subfigure}
    \hspace{-5cm}
    % 插入第二张子图
    \begin{subfigure}{.3\linewidth}
        \centering
        \includegraphics[width=.5\linewidth]{yellowFlower.png}
    \end{subfigure}
    \hspace{-5cm}
    % 插入第三张子图
    \begin{subfigure}{.3\linewidth}
        \centering
        \includegraphics[width=.5\linewidth]{blueFlower.png}
    \end{subfigure}
\caption{Three flowers with different colors.}
\label{fig:fig2}
\end{figure}

\end{document}
\end{lstlisting}

\begin{lstlisting}
使用\vfill命令实现子图纵向等距分布。
\documentclass[12pt]{article}
\usepackage{graphicx}
\usepackage{subcaption}
\begin{document}

\begin{figure}[h!]
\centering
    % 插入第一张子图
    \begin{subfigure}{.3\linewidth}
        \centering
        \includegraphics[width=.5\linewidth]{redflower.png}
        \caption{A red flower.}
    \end{subfigure}
    \vfill
    % 插入第二张子图
    \begin{subfigure}{.3\linewidth}
        \centering
        \includegraphics[width=.5\linewidth]{yellowFlower.png}
        \caption{A yellow flower.}
    \end{subfigure}
    \vfill
    % 插入第三张子图
    \begin{subfigure}{.3\linewidth}
        \centering
        \includegraphics[width=.5\linewidth]{blueFlower.png}
        \caption{A blue flower.}
    \end{subfigure}
\caption{Three flowers with different colors.}
\label{fig:fig1}
\end{figure}

\end{document}
\end{lstlisting}

\begin{lstlisting}
位置调整
\documentclass[12pt]{article}
\usepackage{graphicx}
\begin{document}

There is a blue and black butterfly dancing among the colorful flowers. 

\begin{figure}[t!]
\centering
\includegraphics[width = 0.5\textwidth]{graphics/butterfly.JPG}
\caption{There is a beautiful butterfly.}
\label{butterfly}
\end{figure}

\end{document}
\end{lstlisting}

\begin{lstlisting}
文字环绕

\documentclass[12pt]{article}
\usepackage{graphicx}
\usepackage{wrapfig}
\begin{document}

There is a blue and black butterfly dancing among the colorful flowers. 

\begin{wrapfigure}{r}{8cm}
\centering
\includegraphics[width =0.35\textwidth]{graphics/butterfly.JPG}
\caption{There is a beautiful butterfly.}
\label{butterfly}
\end{wrapfigure}

In descriptive statistics, a box plot or boxplot is a method for graphically depicting groups of numerical data through their quartiles. Box plots may also have lines extending from the boxes (whiskers) indicating variability outside the upper and lower quartiles, hence the terms box-and-whisker plot and box-and-whisker diagram. Outliers may be plotted as individual points. Box plots are non-parametric: they display variation in samples of a statistical population without making any assumptions of the underlying statistical distribution (though Tukey's boxplot assumes symmetry for the whiskers and normality for their length).

\end{document}    
\end{lstlisting}

\newpage
\part{图形绘制}

\setcounter{section}{0}
\section{基本介绍}
TikZ宏包是在LaTeX中创建图形元素的最复杂和最强大的工具。在本节中,我们将通过一些简单的示例来介绍如何在tikzpicture环境中创建基本的图形元素,如:线、点、曲线、圆、矩形等。

\subsection{使用tikzpicture环境创建图形元素}
首先,我们需要通过\textbackslash usepackage\{tikz\}命令调用TikZ宏包。在绘制图形之前,需要声明tikzpicture环境。
在此我们先给出两个用TikZ绘图的例子,其后再进一步详细介绍具体的绘图命令。

\begin{center}
\begin{tikzpicture}

    \draw[red,fill=red] (0,0) .. controls (0,0.75) and (-1.5,1.00) .. (-1.5,2)  arc (180:0:0.75)  -- cycle;
    \draw[red,fill=red] (0,0) .. controls (0,0.75) and ( 1.5,1.00) .. ( 1.5,2)  arc (0:180:0.75) -- cycle;
  
\end{tikzpicture}
\end{center}

\begin{lstlisting}
使用tikzpicture环境制作一个简单的图形。
\documentclass[12pt]{article}
\usepackage{tikz}
\begin{document}

\begin{tikzpicture}

  \draw[red,fill=red] (0,0) .. controls (0,0.75) and (-1.5,1.00) .. (-1.5,2)  arc (180:0:0.75)  -- cycle;
  \draw[red,fill=red] (0,0) .. controls (0,0.75) and ( 1.5,1.00) .. ( 1.5,2)  arc (0:180:0.75) -- cycle;

\end{tikzpicture}

\end{document}
\end{lstlisting}

\begin{center}
\begin{tikzpicture}

    \node[circle, line width = 0.4mm, draw = black, fill = red!45, inner sep = 0pt, minimum size = 0.4cm] (w) at (0, 0) {};
    \node at (0, 0.5) {\small{$\boldsymbol{W}$}};
    
    \node[circle, line width = 0.4mm, draw = black, fill = red!45, inner sep = 0pt, minimum size = 0.4cm] (g) at (1.5, 0) {};
    \node at (1.5, 0.5) {\small{$\boldsymbol{\mathcal{G}}$}};
    
    \node[circle, line width = 0.4mm, draw = black, fill = red!45, inner sep = 0pt, minimum size = 0.4cm] (v) at (3, 0) {};
    \node at (3, 0.5) {\small{$\boldsymbol{V}$}};
    
    \path [draw, line width = 0.4mm, -] (w) edge (g);
    \node at (0.75, 0.25) {\small{$R$}};
    \path [draw, line width = 0.4mm, -] (g) edge (v);
    \node at (2.25, 0.25) {\small{$K$}};
    
    \draw [line width = 0.4mm] (w) -- (0, -0.8);
    \node at (-0.25, -0.4) {\small{$N$}};
    \draw [line width = 0.4mm] (g) -- (1.5, -0.8);
    \node at (1.5-0.25, -0.4) {\small{$N$}};
    \draw [line width = 0.4mm] (v) -- (3, -0.8);
    \node at (3-0.25, -0.4) {\small{$d$}};
\end{tikzpicture}
\end{center}

\begin{lstlisting}
使用tikz宏包中的tikzpicture环境创建一个张量网络图。
\begin{tikzpicture}

    \node[circle, line width = 0.4mm, draw = black, fill = red!45, inner sep = 0pt, minimum size = 0.4cm] (w) at (0, 0) {};
    \node at (0, 0.5) {\small{$\boldsymbol{W}$}};
    
    \node[circle, line width = 0.4mm, draw = black, fill = red!45, inner sep = 0pt, minimum size = 0.4cm] (g) at (1.5, 0) {};
    \node at (1.5, 0.5) {\small{$\boldsymbol{\mathcal{G}}$}};
    
    \node[circle, line width = 0.4mm, draw = black, fill = red!45, inner sep = 0pt, minimum size = 0.4cm] (v) at (3, 0) {};
    \node at (3, 0.5) {\small{$\boldsymbol{V}$}};
    
    \path [draw, line width = 0.4mm, -] (w) edge (g);
    \node at (0.75, 0.25) {\small{$R$}};
    \path [draw, line width = 0.4mm, -] (g) edge (v);
    \node at (2.25, 0.25) {\small{$K$}};
    
    \draw [line width = 0.4mm] (w) -- (0, -0.8);
    \node at (-0.25, -0.4) {\small{$N$}};
    \draw [line width = 0.4mm] (g) -- (1.5, -0.8);
    \node at (1.5-0.25, -0.4) {\small{$N$}};
    \draw [line width = 0.4mm] (v) -- (3, -0.8);
    \node at (3-0.25, -0.4) {\small{$d$}};
    
\end{tikzpicture}
\end{lstlisting}

\subsection{绘制直线}
从绘制一条直线开始,入门这个强大的LaTeX绘图工具。首先,画一条直线需要给出起始点坐标和终止点坐标,我们可以简单地通过如下代码

\begin{figure}[h]
    \begin{center}
        \begin{tikzpicture}
            \draw (1,1) -- (1,3);
        \end{tikzpicture}
        \caption{直线}
    \end{center}
\end{figure}


\begin{lstlisting}
绘制直线显示标题并居中显示
\begin{figure}[h]
    \begin{center}
        \begin{tikzpicture}
            \draw (1,1) -- (1,3);
        \end{tikzpicture}
        \caption{直线}
    \end{center}
\end{figure}
\end{lstlisting}

我们可以通过设定一系列的坐标点,来实现多条线段的连续绘制。

\begin{figure}[h]
    \begin{center}
        \begin{tikzpicture}
            \draw (-2,0) -- (2,0) -- (2,2) -- (-2,2) -- (-2,0);
        \end{tikzpicture}
        \caption{多条线段直线}
    \end{center}
\end{figure}

\begin{lstlisting}
\begin{figure}[h]
    \begin{center}
        \begin{tikzpicture}
            \draw (-2,0) -- (2,0) -- (2,2) -- (-2,2) -- (-2,0);
        \end{tikzpicture}
        \caption{多条线段直线}
    \end{center}
\end{figure}
\end{lstlisting}

多行命令,实现多段线条分开绘制

值得注意的是,在tikzpicture环境中,像Matlab语言一样,我们需要采用;符号来标记一个指令的结束。
这样的指令结束标记让我们不但可以在多行完成一条指令,同时也可以在一行内实现多条指令。



\begin{figure}[!h]
    \begin{center}
        \begin{tikzpicture}[scale=0.3]
            \draw (-2,0) -- (2,0) -- (2,2) -- (-2,2) -- (-2,0);
            \draw (0,4) -- (0,-2);
            \draw (3,-2) -- (3,4) -- (7,4) -- (7,-2) -- (3,-2);
            \draw (4,3) -- (6,3); \draw (4,1) -- (6,1); \draw (4,-1) -- (6,-1);
            \draw (5,3) -- (5,-1); \draw (5.75,0.25) -- (6.25,-0.25);
        \end{tikzpicture}
        \caption{多条线段分开绘制}
    \end{center}
\end{figure}

\begin{lstlisting}
多段线条分开绘制
\begin{figure}[h]
    \begin{center}
        \begin{tikzpicture}
            \draw (-2,0) -- (2,0) -- (2,2) -- (-2,2) -- (-2,0);
            \draw (0,4) -- (0,-2);
            \draw (3,-2) -- (3,4) -- (7,4) -- (7,-2) -- (3,-2);
            \draw (4,3) -- (6,3); \draw (4,1) -- (6,1); \draw (4,-1) -- (6,-1);
            \draw (5,3) -- (5,-1); \draw (5.75,0.25) -- (6.25,-0.25);
        \end{tikzpicture}
        \caption{多条线段分开绘制}
    \end{center}
\end{figure}    
\end{lstlisting}
\vspace{-1.5cm}
\subsection{图形缩放}
在上小节中,我们绘制图形需要给出精确的坐标点。但是在绘制好之后,如果需要调整图形大小,我们可以采用scale的方式对图形进行缩放。

横向缩放使用命令[xscale=],纵向缩放使用命令[yscale=],横纵缩放使用命令[xscale=,yscale=]

\begin{figure}[!h]
    \begin{center}
        \begin{tikzpicture}[scale=0.45]
            \draw (-2,0) -- (2,0) -- (2,2) -- (-2,2) -- (-2,0);
            \draw (0,4) -- (0,-2);
            \draw (3,-2) -- (3,4) -- (7,4) -- (7,-2) -- (3,-2);
            \draw (4,3) -- (6,3); \draw (4,1) -- (6,1); \draw (4,-1) -- (6,-1);
            \draw (5,3) -- (5,-1); \draw (5.75,0.25) -- (6.25,-0.25);
        \end{tikzpicture}
        \caption{多条线段分开绘制}
    \end{center}
\end{figure}
\vspace{-1cm}
\begin{lstlisting}
整体缩放。
\documentclass[12pt]{article}
\usepackage{tikz}
\begin{document}
\begin{tikzpicture}[scale=0.5]
    \draw (-2,0) -- (2,0) -- (2,2) -- (-2,2) -- (-2,0);
    \draw (0,4) -- (0,-2);
    \draw (3,-2) -- (3,4) -- (7,4) -- (7,-2) -- (3,-2);
    \draw (4,3) -- (6,3); \draw (4,1) -- (6,1); \draw (4,-1) -- (6,-1);
    \draw (5,3) -- (5,-1); \draw (5.75,0.25) -- (6.25,-0.25);
\end{tikzpicture}
\end{document}    
\end{lstlisting}


\begin{lstlisting}
横向缩放。
\documentclass[12pt]{article}
\usepackage{tikz}
\begin{document}
\begin{tikzpicture}[xscale=1.5]
    \draw (-2,0) -- (2,0) -- (2,2) -- (-2,2) -- (-2,0);
    \draw (0,4) -- (0,-2);
    \draw (3,-2) -- (3,4) -- (7,4) -- (7,-2) -- (3,-2);
    \draw (4,3) -- (6,3); \draw (4,1) -- (6,1); \draw (4,-1) -- (6,-1);
    \draw (5,3) -- (5,-1); \draw (5.75,0.25) -- (6.25,-0.25);
\end{tikzpicture}
\end{document}    

横纵缩放。
\documentclass[12pt]{article}
\usepackage{tikz}
\begin{document}
\begin{tikzpicture}[xscale=1.5, yscale = 2]
    \draw (-2,0) -- (2,0) -- (2,2) -- (-2,2) -- (-2,0);
    \draw (0,4) -- (0,-2);
    \draw (3,-2) -- (3,4) -- (7,4) -- (7,-2) -- (3,-2);
    \draw (4,3) -- (6,3); \draw (4,1) -- (6,1); \draw (4,-1) -- (6,-1);
    \draw (5,3) -- (5,-1); \draw (5.75,0.25) -- (6.25,-0.25);
\end{tikzpicture}
\end{document}
\end{lstlisting}

\subsection{绘制箭头}
在绘制直线的基础上,我们往往需要通过绘制箭头来指向性地表达意图。箭头的绘制只需要在直线绘制的基础上,增加[option]进行声明即可。


\begin{figure}[h!]
    \begin{center}
        \begin{tikzpicture}
            \draw [->] (0,0) -- (2,0);
            \draw [<-] (0,-0.5) -- (2,-0.5);
            \draw [|->] (0,-1) -- (2,-1);
        \end{tikzpicture}
        \caption{箭头}
    \end{center} 
\end{figure}

\begin{lstlisting}
\begin{figure}[h!]
    \begin{center}
        \begin{tikzpicture}
            \draw [->] (0,0) -- (2,0);
            \draw [<-] (0,-0.5) -- (2,-0.5);
            \draw [|->] (0,-1) -- (2,-1);
        \end{tikzpicture}
        \caption{箭头}
    \end{center} 
\end{figure}    
\end{lstlisting}

\begin{figure}[h!]
    \begin{center}
        \begin{tikzpicture}
            \draw [<->] (0,2) -- (0,0) -- (3,0);
        \end{tikzpicture}
        \caption{坐标轴}
    \end{center} 
\end{figure}

\begin{lstlisting}
利用绘制箭头的例子以及多条线段连续绘制的例子,用一行命令绘制一个直角坐标系。   
\begin{figure}[h!]
    \begin{center}
        \begin{tikzpicture}
            \draw [<->] (0,2) -- (0,0) -- (3,0);
        \end{tikzpicture}
        \caption{坐标轴}
    \end{center} 
\end{figure}
\end{lstlisting}

\subsection{调整线条粗细}
采用\textbackslash draw命令时,增加的[option]声明也可以用来调整线条的粗细

\begin{figure}[h!]
    \begin{center}
        \begin{tikzpicture}
            \draw [ultra thick] (0,1) -- (2,1);
            \draw [thick] (0,0.5) -- (2,0.5);
            \draw [thin] (0,0) -- (2,0);
        \end{tikzpicture}
        \caption{线条粗细}
    \end{center} 
\end{figure}

\begin{lstlisting}
绘制不同粗细的线条。
\documentclass[12pt]{article}
\usepackage{tikz}
\begin{document}
\begin{figure}[h!]
    \begin{center}
        \begin{tikzpicture}
            \draw [ultra thick] (0,1) -- (2,1);
            \draw [thick] (0,0.5) -- (2,0.5);
            \draw [thin] (0,0) -- (2,0);
        \end{tikzpicture}
        \caption{线条粗细}
    \end{center} 
\end{figure}
\end{document}
\end{lstlisting}

线条的粗细可以通过不同的指令来控制,从细到粗分别可调用:ultra thin,very thin,thin,semithick,thick,very thick,ultra thick。

除此之外,我们也可以自行定义线条的粗细,如[line width=5]、[line width=0.2cm]。值得注意的是,当我们直接声明数值而不声明单位时,其默认单位均为pt。

\begin{figure}[h!]
    \begin{center}
        \begin{tikzpicture}
            \draw [ultra thin] (0,0) -- (2,0);
            \draw [very thin] (0,0.5) -- (2,0.5);
            \draw [thin] (0,1) -- (2,1);
            \draw [semithick] (0,1.5) -- (2,1.5);
            \draw [thick] (0,2) -- (2,2);
            \draw [very thick] (0,2.5) -- (2,2.5);
            \draw [ultra thick] (0,3) -- (2,3);
        \end{tikzpicture}
        \caption{不同线条粗细}
    \end{center} 
\end{figure}

\begin{lstlisting}
\begin{figure}[h!]
    \begin{center}
        \begin{tikzpicture}
            \draw [ultra thin] (0,0) -- (2,0);
            \draw [very thin] (0,0.5) -- (2,0.5);
            \draw [thin] (0,1) -- (2,1);
            \draw [semithick] (0,1.5) -- (2,1.5);
            \draw [thick] (0,2) -- (2,2);
            \draw [very thick] (0,2.5) -- (2,2.5);
            \draw [ultra thick] (0,3) -- (2,3);
        \end{tikzpicture}
        \caption{不同线条粗细}
    \end{center} 
\end{figure}    
\end{lstlisting}


\subsection{虚线}
我们也可以在[option]声明中增加对于线条形状的定义。如虚线dashed和点线dotted。

\begin{figure}[h!]
    \begin{center}
        \begin{tikzpicture}
            \draw [dashed, ultra thick] (0,1) -- (2,1); %我们可以通过组合多种option来声明线条的多种特征。
            \draw [dashed] (0, 0.5) -- (2,0.5);
            \draw [dotted] (0,0) -- (2,0);
        \end{tikzpicture}
        \caption{虚线}
    \end{center} 
\end{figure}   

\begin{lstlisting}
绘制虚线
\begin{figure}[h!]
    \begin{center}
        \begin{tikzpicture}
            \draw [dashed, ultra thick] (0,1) -- (2,1); %我们可以通过组合多种option来声明线条的多种特征。
            \draw [dashed] (0, 0.5) -- (2,0.5);
            \draw [dotted] (0,0) -- (2,0);
        \end{tikzpicture}
        \caption{虚线}
    \end{center} 
\end{figure} 
\end{lstlisting}
\vspace*{-1.5cm}
\subsection{颜色}
我们也可以在[option]声明中增加对于线条颜色的定义。如红色red、 绿色green、蓝色blue等等。

\begin{figure}[h!]
    \begin{center}
        \begin{tikzpicture}
            \draw [red, dashed, ultra thick] (0,1) -- (2,1); %我们可以通过组合多种option来声明线条的多种特征。
            \draw [green, dashed] (0, 0.5) -- (2,0.5);
            \draw [blue, dotted] (0,0) -- (2,0);
        \end{tikzpicture}
        \caption{不同颜色虚线}
    \end{center} 
\end{figure}   
\vspace*{-1cm}
\begin{lstlisting}
绘制不同颜色的直线
\begin{figure}[h!]
    \begin{center}
        \begin{tikzpicture}
            \draw [red, dashed, ultra thick] (0,1) -- (2,1); %我们可以通过组合多种option来声明线条的多种特征。
            \draw [green, dashed] (0, 0.5) -- (2,0.5);
            \draw [blue, dotted] (0,0) -- (2,0);
        \end{tikzpicture}
        \caption{不同颜色虚线}
    \end{center} 
\end{figure}   
\end{lstlisting}

\section{节点介绍}
节点是TikZ中的一个常用功能。在绘制节点时,通常需要声明其位置和形状,部分节点可以在其中添加文字,
同时也可以为节点赋予一个名称,用于后续参考。在本节中,我们将详细介绍节点的相关功能及其应用。

\subsection{节点基本介绍}

这里需要注意的是,在[shape=circle,draw=blue!50,fill=blue!20]中,shape指令声明节点形状,draw指令声明是否显现该形状的边框,
并用draw=来声明边框颜色,fill命令指明该节点是否要填充,并用fill=来声明填充颜色。若要在节点中显示文字,
需在后面\{\}中填写对应的文字即可。

\begin{figure}[h!]
    \begin{center}
        \begin{tikzpicture}
            \path (0,2) node [shape=circle,draw=blue!50,fill=blue!20,thick] {}
                  (0,1) node [shape=circle,draw=blue!50,fill=blue!20,thick] {$C$}
                  (0,0) node [shape=circle,draw=blue!50,fill=blue!20,thick] {}
                  (1,1) node [shape=rectangle,draw=black!50,fill=black!20] {}
                  (-1,1) node [shape=rectangle,draw=black!50,fill=black!20] {};
        \end{tikzpicture}
        \caption{绘制节点}
    \end{center} 
\end{figure}  

\begin{lstlisting}
\begin{figure}[h!]
    \begin{center}
        \begin{tikzpicture}
            \path (0,2) node [shape=circle,draw=blue!50,fill=blue!20,thick] {}
                    (0,1) node [shape=circle,draw=blue!50,fill=blue!20,thick] {$C$}
                    (0,0) node [shape=circle,draw=blue!50,fill=blue!20,thick] {}
                    (1,1) node [shape=rectangle,draw=black!50,fill=black!20] {}
                    (-1,1) node [shape=rectangle,draw=black!50,fill=black!20] {};
        \end{tikzpicture}
        \caption{绘制节点}
    \end{center} 
\end{figure}  
等价于
\begin{figure}[h!]
    \begin{center}
        \begin{tikzpicture}
            \node [shape=circle,draw=blue!50,fill=blue!20,thick] at (0,2)  {};
            \node [shape=circle,draw=blue!50,fill=blue!20,thick] at (0,1) {$C$};
            \node [shape=circle,draw=blue!50,fill=blue!20,thick] at (0,0)  {};
            \node [shape=rectangle,draw=black!50,fill=black!20] at (1,1) {};
            \node [shape=rectangle,draw=black!50,fill=black!20] at (-1,1) {};
        \end{tikzpicture}
        \caption{绘制节点}
    \end{center} 
\end{figure}  
\end{lstlisting}


\subsection{节点样式}

当某一种形状及颜色的节点需要在不同位置多次出现时,上述代码显得不够优美。我们可以通过一段代码提前声明该节点的样式,并反复调用这段代码即可。

\tikzstyle{aaa}=[circle,draw=blue!50,fill=blue!20,thick]
\tikzstyle{bbb}=[rectangle,draw=black!50,fill=black!20]
\begin{figure}[h!]
    \begin{center}
        \begin{tikzpicture}
            \path (0,2) node [aaa, name=a1] {}
                  (0,1) node [aaa, name=a2] {$C$}
                  (0,0) node [aaa, name=a3] {}
                  (1,1) node [bbb, name=b1] {}
                  (-1,1) node [bbb, name=b2] {};
        \end{tikzpicture}
        \caption{节点样式}
    \end{center} 
\end{figure} 

\begin{lstlisting}
\tikzstyle{aaa}=[circle,draw=blue!50,fill=blue!20,thick]
\tikzstyle{bbb}=[rectangle,draw=black!50,fill=black!20]
\begin{figure}[h!]
    \begin{center}
        \begin{tikzpicture}
            \path (0,2) node [aaa] {}
                    (0,1) node [aaa] {$C$}
                    (0,0) node [aaa] {}
                    (1,1) node [bbb] {}
                    (-1,1) node [bbb] {};
        \end{tikzpicture}
        \caption{节点样式}
    \end{center} 
\end{figure}     
\end{lstlisting}

\subsection{节点命名}
为了将节点相互连接起来,我们需要指明连接哪两个节点。因此,每个节点需要我们声明一个名称。声明名称有两种方式,
一种是采用name=的方式,另一种是在\textbackslash node后用括号声明,如\textbackslash node (name) ...。

\begin{lstlisting}
绘制节点,并声明节点名称,方法一。
\documentclass[12pt]{article}
\usepackage{tikz}
\begin{document}
\tikzstyle{aaa}=[circle,draw=blue!50,fill=blue!20,thick]
\tikzstyle{bbb}=[rectangle,draw=black!50,fill=black!20]
\begin{tikzpicture}
    \path (0,2) node [aaa,name=a1] {}
          (0,1) node [aaa,name=a2] {$C$}
          (0,0) node [aaa,name=a3] {}
          (1,1) node [bbb,name=b1] {}
          (-1,1) node [bbb,name=b2] {};
\end{tikzpicture}
\end{document}

绘制节点,并声明节点名称,方法二。
\documentclass[12pt]{article}
\usepackage{tikz}
\begin{document}
\tikzstyle{aaa}=[circle,draw=blue!50,fill=blue!20,thick]
\tikzstyle{bbb}=[rectangle,draw=black!50,fill=black!20]
\begin{tikzpicture}
    \node (a1) [aaa] at (0,2)  {};
    \node (a2) [aaa] at (0,1) {$C$};
    \node (a3) [aaa] at (0,0)  {};
    \node (b1) [bbb] at (1,1) {};
    \node (b2) [bbb] at (-1,1) {};
\end{tikzpicture}
\end{document}
\end{lstlisting}

\subsection{基于相对位置绘制节点}
给每个节点命名后,我们便可以通过above of(上)、below of(下)、left of(左)、right of(右)等命令来声明新节点与某个节点的相对位置来绘制图形。

\begin{lstlisting}
利用相对位置绘制节点。
\documentclass[12pt]{article}
\usepackage{tikz}
\begin{document}
\tikzstyle{aaa}=[circle,draw=blue!50,fill=blue!20,thick]
\tikzstyle{bbb}=[rectangle,draw=black!50,fill=black!20]
\begin{tikzpicture}
    \node (a1) [aaa]                {};
    \node (a2) [aaa]  [below of=a1] {$C$};
    \node (a3) [aaa]  [below of=a2] {};
    \node (b1) [bbb]  [right of=a2] {};
    \node (b2) [bbb]  [left  of=a2] {};
\end{tikzpicture}
\end{document}
\end{lstlisting}

\subsection{ 连接节点}
有了节点名称了,我们就可以对节点进行连接。我们拿连接A与B节点为例,在连接时,我们通常需要声明A节点的哪个位置与B节点的哪个位置连接。
位置声明通常采用east(右)、west(左)、north(上)、south(下)、center(中心)等命令。


\begin{lstlisting}
利用相对位置连接节点。
\documentclass[12pt]{article}
\usepackage{tikz}
\begin{document}
\tikzstyle{aaa}=[circle,draw=blue!50,fill=blue!20,thick]
\tikzstyle{bbb}=[rectangle,draw=black!50,fill=black!20]
\begin{tikzpicture}
    \node (a1) [aaa]                {$a_1$};
    \node (a2) [aaa]  [below of=a1] {$C$};
    \node (a3) [aaa]  [below of=a2] {$a_3$};
    \node (b1) [bbb]  [right of=a2] {$b_1$};
    \node (b2) [bbb]  [left  of=a2] {$b_2$};
    \draw [->] (a2.west) -- (b2.east);
    \draw [->] (a2.east) -- (b1.west);
    \draw [->] (a2.north) -- (a1.south);
    \draw [->] (a2.south) -- (a3.north);
\end{tikzpicture}
\end{document}
该代码等价于
\documentclass[12pt]{article}
\usepackage{tikz}
\begin{document}
\tikzstyle{aaa}=[circle,draw=blue!50,fill=blue!20,thick]
\tikzstyle{bbb}=[rectangle,draw=black!50,fill=black!20]
\begin{tikzpicture}
    \node (a1) [aaa]                {$a_1$};
    \node (a2) [aaa]  [below of=a1] {$C$};
    \node (a3) [aaa]  [below of=a2] {$a_3$};
    \node (b1) [bbb]  [right of=a2] {$b_1$};
    \node (b2) [bbb]  [left  of=a2] {$b_2$};
    \draw [->] (a2) -- (b2);
    \draw [->] (a2) -- (b1);
    \draw [->] (a2) -- (a1);
    \draw [->] (a2) -- (a3);
\end{tikzpicture}
\end{document}
\end{lstlisting}

\begin{figure}[h!]
    \begin{center}
        \begin{tikzpicture}
            \node (a1) [aaa]                {$a_1$};
            \node (a2) [aaa]  [below of=a1] {$C$};
            \node (a3) [aaa]  [below of=a2] {$a_3$};
            \node (b1) [bbb]  [right of=a2] {$b_1$};
            \node (b2) [bbb]  [left  of=a2] {$b_2$};
            \draw [->] (a2) -- (b2);
            \draw [->] (a2) -- (b1);
            \draw [->] (a2) -- (a1);
            \draw [->] (a2) -- (a3);
        \end{tikzpicture}
        \caption{连接节点}
    \end{center} 
\end{figure}  

\begin{lstlisting}
利用edge命令连接节点,方法一。
\documentclass[12pt]{article}
\usepackage{tikz}
\begin{document}
\tikzstyle{aaa}=[circle,draw=blue!50,fill=blue!20,thick]
\tikzstyle{bbb}=[rectangle,draw=black!50,fill=black!20]
\begin{tikzpicture}
    \node (a1) [aaa]                {$a_1$};
    \node (a2) [aaa]  [below of=a1] {$C$}   edge [->] (a1);
    \node (a3) [aaa]  [below of=a2] {$a_3$} edge [<-] (a2);
    \node (b1) [bbb]  [right of=a2] {$b_1$} edge [<-] (a2);
    \node (b2) [bbb]  [left  of=a2] {$b_2$} edge [<-] (a2);
\end{tikzpicture}
\end{document}
等价于
\documentclass[12pt]{article}
\usepackage{tikz}
\begin{document}
\tikzstyle{aaa}=[circle,draw=blue!50,fill=blue!20,thick]
\tikzstyle{bbb}=[rectangle,draw=black!50,fill=black!20]
\begin{tikzpicture}
    \path (0,2)  node [aaa,name=a1] {$a_1$}
            (0,1)  node [aaa,name=a2] {$C$}   edge [->] (a1)
            (0,0)  node [aaa,name=a3] {$a_3$} edge [<-] (a2)
            (1,1)  node [bbb,name=b1] {$b_1$} edge [<-] (a2)
            (-1,1) node [bbb,name=b2] {$b_2$} edge [<-] (a2);
\end{tikzpicture}
\end{document}
\end{lstlisting}

\begin{lstlisting}
声明每个节点的连接位置,将周围节点边缘连接直中间节点的中心。
\documentclass[12pt]{article}
\usepackage{tikz}
\begin{document}
\tikzstyle{aaa}=[circle,draw=blue!50,fill=blue!20,thick]
\tikzstyle{bbb}=[rectangle,draw=black!50,fill=black!20]
\begin{tikzpicture}
    \node (a1) [aaa]                {$a_1$};
    \node (a2) [aaa]  [below of=a1] {};
    \node (a3) [aaa]  [below of=a2] {$a_3$};
    \node (b1) [bbb]  [right of=a2] {$b_1$};
    \node (b2) [bbb]  [left  of=a2] {$b_2$};
    \draw [->] (a2.center) -- (b2.east);
    \draw [->] (a2.center) -- (b1.west);
    \draw [->] (a2.center) -- (a1.south);
    \draw [->] (a2.center) -- (a3.north);
\end{tikzpicture}
\end{document}
\end{lstlisting}

\section{高级功能}
\subsection{矩形、圆形、曲线}
我们可以通过\textbackslash draw (x,y) rectangle (w,h);的方式绘制一个矩形,其左下角坐标位于点(x,y)处,长度为w,高度为h。
类似地,我们也可以通过\textbackslash draw (x,y) circle [radius=r];的方式绘制一个圆形,其圆心落在点(x,y)处,半径为r。
除此之外,我们可以通过\textbackslash draw (x,y) arc [radius=r, start angle=a1, end angle=a2]的方式绘制一条弧线,
它从点(x,y)处开始绘制,该弧线曲率半径为r,其起始角度为所对应曲率圆的$a_{1}$处,终止角度为所对应曲率圆的$a_{2}$处。

\newpage
\begin{figure}[!h]
    \begin{center}
        \begin{tikzpicture}
            \draw [red] (0,0) rectangle (1.5,1);
            \draw [blue,ultra thick] (3,0.5) circle [radius=0.5];
            \draw [green] (6,0) arc [radius=1.5,start angle=45,end angle=100]; 
        \end{tikzpicture}
        \caption{矩形圆形曲线}
    \end{center}
\end{figure}


\begin{lstlisting}
\begin{figure}[!h]
    \begin{center}
        \begin{tikzpicture}
            \draw [red] (0,0) rectangle (1.5,1);
            \draw [blue,ultra thick] (3,0.5) circle [radius=0.5];
            \draw [green] (6,0) arc [radius=1.5,start angle=45,end angle=100]; 
        \end{tikzpicture}
        \caption{矩形圆形曲线}
    \end{center}
\end{figure} 
\end{lstlisting}

\subsection{平滑过渡曲线}
在绘图时,一种不突兀地连接两条直线的方式是采用平滑过渡圆角/曲线。在本节中,我们将介绍两种平滑过渡曲线的绘制方法:绘制带圆角的曲线和绘制过渡曲线。


\begin{figure}[!h]
    \begin{center}
        \begin{tikzpicture}
            \draw [<->, rounded corners, thick, purple] (0,2) -- (0,0) -- (3,0);
        \end{tikzpicture}
        \caption{圆角坐标系}
    \end{center} 
\end{figure}


\begin{lstlisting}
\begin{figure}[!h]
    \begin{center}
        \begin{tikzpicture}
            \draw [<->, rounded corners, thick, purple] (0,2) -- (0,0) -- (3,0);
        \end{tikzpicture}
        \caption{圆角坐标系}
    \end{center} 
\end{figure}  
\end{lstlisting}


\begin{figure}[!h]
    \begin{center}
        \begin{tikzpicture}
            \draw[<-, thick] (0,2) -- (0,0.5);
            \draw[thick,red] (0,0.5) to [out=270,in=180] (0.5,0);
            \draw[->, thick] (0.5,0) -- (3,0);
        \end{tikzpicture}
        \caption{过渡曲线}
    \end{center}
\end{figure}


\begin{lstlisting}
\begin{figure}
    \begin{center}
        \begin{tikzpicture}
            \draw[<-, thick] (0,2) -- (0,0.5);
            \draw[thick,red] (0,0.5) to [out=270,in=180] (0.5,0);
            \draw[->, thick] (0.5,0) -- (3,0);
        \end{tikzpicture}
        \caption{过渡曲线}
    \end{center}
\end{figure}      
\end{lstlisting}

\begin{figure}[!h]
    \begin{center}
        \begin{tikzpicture}
            \draw [<->,thick, blue] (0,0) to [out=90,in=180] (1,1) to [out=0,in=180] (3,0) to [out=0,in=-90] (4,1) ;
        \end{tikzpicture}
        \caption{S曲线}
    \end{center}
\end{figure}

\begin{lstlisting}
\begin{figure}
    \begin{center}
        \begin{tikzpicture}
            \draw [<->,thick, blue] (0,0) to [out=90,in=180] (1,1) to [out=0,in=180] (3,0) to [out=0,in=-90] (4,1) ;
        \end{tikzpicture}
        \caption{S曲线}
    \end{center}
\end{figure}
\end{lstlisting}

\subsection{根据函数绘制曲线}
TikZ宏包的强大之处在于,它还提供了可供绘制函数的数学引擎。在此我们先给出一个示例,再详细讲解如何利用该宏包绘制函数所对应的曲线。
\newpage
\begin{figure}[!h]
    \begin{center}
        \begin{tikzpicture}[xscale=0.01,yscale=1]
            \draw [<->] (0,1) -- (0,0) -- (370,0);
            \draw[green, thick, domain=0:360] plot (\x, {sin(\x)}); % 这里需要注意带上{}
        \end{tikzpicture}
        \caption{S曲线}
    \end{center}
\end{figure}

\begin{lstlisting}
利用函数绘制正弦曲线。
\documentclass[12pt]{article}
\usepackage{tikz}
\begin{document}
\begin{tikzpicture}[xscale=0.01,yscale=1]
    \draw [<->] (0,1) -- (0,0) -- (370,0);
    \draw[green, thick, domain=0:360] plot (\x, {sin(\x)}); % 这里需要注意带上{}
\end{tikzpicture}
\end{document}
\end{lstlisting}

\vspace{-0.8cm}
在上述例子中,domain指令声明了横坐标x的范围。在本示例中,我们利用sin函数绘制了一段正弦曲线。

除了本示例中的正弦曲线sin函数,我们还可以调用大量其他函数,在此列举一部分作为示例: 
阶乘函数:factorial(\textbackslash x)、 平方根函数:sqrt(\textbackslash x)、 幂函数:pow(\textbackslash x,y)
(该函数为xy)、 指数函数:exp(\textbackslash x)、 对数函数:ln(\textbackslash x)、log10(\textbackslash x)、log2(\textbackslash x)、
绝对值函数:abs(\textbackslash x)、 取余函数:mod(\textbackslash x,y)(即求x被y除后的余数)、 圆整函数:round(\textbackslash x)、floor(\textbackslash x)、ceil(\textbackslash x)、
三角函数:sin(\textbackslash x)、cos(\textbackslash x)、tan(\textbackslash x)等等。
sin(\textbackslash x)、cos(\textbackslash x)、tan(\textbackslash x) 
值得注意的是,在三角函数中,通常默认自变量x以度(°)为单位。
若要采用弧度制,则需要将函数分别改写为sin(\textbackslash x r)、cos(\textbackslash x r)、tan(\textbackslash x r)。
除了这部分常用函数之外,我们通常还会使用两个常数:$e$($e$=2.718281828)和$\pi$($\pi$=3.141592654)。

\begin{figure}[!h]
    \begin{center}
        \begin{tikzpicture}[yscale=1.5]
            \draw [thick, ->] (0,0) -- (6.5,0);
            \draw [thick, ->] (0,-1.1) -- (0,1.1);
            \draw [green,domain=0:2*pi] plot (\x, {(sin(\x r)* ln(\x+1))/2});
            \draw [red,domain=0:pi] plot (\x, {sin(\x r)});
            \draw [blue, domain=pi:2*pi] plot (\x, {cos(\x r)*exp(\x/exp(2*pi))});
        \end{tikzpicture}
        \caption{S曲线}
    \end{center}
\end{figure}
\newpage
\begin{lstlisting}
\begin{figure}[!h]
    \begin{center}
        \begin{tikzpicture}[yscale=1.5]
            \draw [thick, ->] (0,0) -- (6.5,0);
            \draw [thick, ->] (0,-1.1) -- (0,1.1);
            \draw [green,domain=0:2*pi] plot (\x, {(sin(\x r)* ln(\x+1))/2});
            \draw [red,domain=0:pi] plot (\x, {sin(\x r)});
            \draw [blue, domain=pi:2*pi] plot (\x, {cos(\x r)*exp(\x/exp(2*pi))});
        \end{tikzpicture}
        \caption{S曲线}
    \end{center}
\end{figure}    
\end{lstlisting}

\subsection{简单图形的区域填充}

\begin{figure}[!h]
    \begin{center}
        \begin{tikzpicture}[yscale=1.5]
            \draw [fill=red,ultra thick] (0,0) rectangle (1,1);
            \draw [fill=red,ultra thick,red] (2,0) rectangle (3,1); % 这里的第二个red声明了区域周围边框线的颜色
            \draw [blue, fill=blue] (4,0) -- (5,1) -- (4.75,0.15) -- (4,0);
            \draw [fill] (7,0.5) circle [radius=0.1];
            \draw [fill=orange] (9,0) rectangle (11,1);
            \draw [fill=white] (9.25,0.25) rectangle (10,1.5);
        \end{tikzpicture}
        \caption{简单形状区域填充}
    \end{center}
\end{figure}

\begin{lstlisting}
简单形状的区域填充。
\documentclass[12pt]{article}
\usepackage{tikz}
\begin{document}
\begin{figure}[!h]
    \begin{center}
        \begin{tikzpicture}[yscale=1.5]
            \draw [fill=red,ultra thick] (0,0) rectangle (1,1);
            \draw [fill=red,ultra thick,red] (2,0) rectangle (3,1); % 这里的第二个red声明了区域周围边框线的颜色
            \draw [blue, fill=blue] (4,0) -- (5,1) -- (4.75,0.15) -- (4,0);
            \draw [fill] (7,0.5) circle [radius=0.1];
            \draw [fill=orange] (9,0) rectangle (11,1);
            \draw [fill=white] (9.25,0.25) rectangle (10,1.5);
        \end{tikzpicture}
        \caption{简单形状区域填充}
    \end{center}
\end{figure}
\end{document}    
\end{lstlisting}

如上例中的注释所提,我们可以通过声明图形边框线的颜色来对边框线进行个性化更改。若并不希望出现边框线,我们可以采用path命令替换\textbackslash draw命令。

\begin{figure}[!h]
    \begin{center}
        \begin{tikzpicture}[yscale=1.5]
            \path [fill=red,thick] (0,0) rectangle (1.5,1);
            \draw [fill=red,thick] (2,0) rectangle (3.5,1);
        \end{tikzpicture}
        \caption{path区域填充}
    \end{center}
\end{figure}

\begin{lstlisting}
简单形状的区域填充。
\documentclass[12pt]{article}
\usepackage{tikz}
\begin{document}
\begin{figure}[!h]
    \begin{center}
        \begin{tikzpicture}[yscale=1.5]
            \path [fill=red,thick] (0,0) rectangle (1.5,1);
            \draw [fill=red,thick] (2,0) rectangle (3.5,1);
        \end{tikzpicture}
        \caption{path区域填充}
    \end{center}
\end{figure}
\end{document}
\end{lstlisting}

\subsection{在图形中填写标签}
在绘图时,在合适的位置加入适当的文字进行说明,对内容的表达具有很重要的作用。在本节中,我们将通过\textbackslash node来实现这一功能。

\begin{figure}[!h]
    \begin{center}
        \begin{tikzpicture}[scale=1.5]
            \draw [thick,<->] (0,1)--(0,0)--(1,0);
            \node at (0.5,0.5) {\LaTeX};
        \end{tikzpicture}
        \caption{直角坐标系插入文字}
    \end{center}
\end{figure}

\begin{lstlisting}
\begin{figure}[!h]
    \begin{center}
        \begin{tikzpicture}[scale=1.5]
            \draw [thick,<->] (0,1)--(0,0)--(1,0);
            \node at (0.5,0.5) {\LaTeX};
        \end{tikzpicture}
        \caption{直角坐标系插入文字}
    \end{center}
\end{figure}     
\end{lstlisting}

\begin{figure}[!h]
    \begin{center}
        \begin{tikzpicture}[scale=1.5]
            \draw [thick,<->] (0,1)--(0,0)--(1,0);
            \draw [fill] (0.5,0.5) circle [radius=0.05];
            \node [below] at (0.5,0.5) {\LaTeX};
        \end{tikzpicture}
        \caption{直角坐标系指定位置下方插入文字}
    \end{center}
\end{figure}  

\begin{lstlisting}
\begin{figure}[!h]
    \begin{center}
        \begin{tikzpicture}[scale=1.5]
            \draw [thick,<->] (0,1)--(0,0)--(1,0);
            \draw [fill] (0.5,0.5) circle [radius=0.05];
            \node [below] at (0.5,0.5) {\LaTeX};
        \end{tikzpicture}
        \caption{直角坐标系指定位置下方插入文字}
    \end{center}
\end{figure}     
\end{lstlisting}

\begin{figure}[!h]
    \begin{center}
        \begin{tikzpicture}[scale=5]
            \draw [thick,<->] (0,1)--(0,0)--(1,0);
            \draw [fill] (0.5,0.5) circle [radius=0.025];
            \node [below] at (0.5,0.5) {\LaTeX};
            \node [above] at (0.5,0.5) {\LaTeX};
            \node [left] at (0.5,0.5) {\LaTeX};
            \node [right] at (0.5,0.5) {\LaTeX};
        \end{tikzpicture}
        \caption{直角坐标系指定位置插入文字}
    \end{center}
\end{figure}  

\begin{lstlisting}
\begin{figure}[!h]
    \begin{center}
        \begin{tikzpicture}[scale=5]
            \draw [thick,<->] (0,1)--(0,0)--(1,0);
            \draw [fill] (0.5,0.5) circle [radius=0.025];
            \node [below] at (0.5,0.5) {\LaTeX};
            \node [above] at (0.5,0.5) {\LaTeX};
            \node [left] at (0.5,0.5) {\LaTeX};
            \node [right] at (0.5,0.5) {\LaTeX};
        \end{tikzpicture}
        \caption{直角坐标系指定位置插入文字}
    \end{center}
\end{figure}  
\end{lstlisting}

\begin{lstlisting}
\draw [thick, <->] (0,2) -- (0,0) -- (2,0);
\draw [fill] (1,1) circle [radius=0.025];
\node [below right, red] at (1,1) {below right};
\node [above left, green] at (1,1) {above left};
\node [below left, purple] at (1,1) {below left};
\node [above right, magenta] at (1,1) {above right};
\draw [thick, <->] (0,2) -- (0,0) -- (2,0);
\node [below right] at (2,0) {$x$};
\node [left] at (0,2) {$y$};
\draw[fill] (1,1) circle [radius=.5pt];
\node[above right] at (1,1) {$\theta$};
\end{lstlisting}

\subsection{复杂模型实战解析}
科研论文中的复杂模型供读者们进一步解析学习。

\newpage
\tikzset{>=latex}
\tikzstyle{plate caption} = [caption, node distance=0, inner sep=0pt,
below left=5pt and 0pt of #1.south]
\begin{figure}[!h]
    \begin{center}
        \begin{tikzpicture}
            \node [obs] (x) at (0,0) {\large $x_{\boldsymbol{i}}$};
            \node [circle,draw=black,fill=white,inner sep=0pt,minimum size=0.6cm] (u1) at (-1.2,1.6) { $\boldsymbol{u}_{i_1}^{(1)}$};
            \node [circle,draw=black,fill=white,inner sep=0pt,minimum size=0.6cm] (u3) at (1.2,1.6) { $\boldsymbol{u}_{i_d}^{(d)}$};
            \node [circle,draw=black,fill=white,inner sep=0pt,minimum size=0.65cm] (lambda) at (0,3.0) {\large $\boldsymbol{\lambda}$};
            \node[mark size=1pt,color=black] at (0,1.6) {\pgfuseplotmark{*}};
            \node[mark size=1pt,color=black] at (-0.2,1.6) {\pgfuseplotmark{*}};
            \node[mark size=1pt,color=black] at (0.2,1.6) {\pgfuseplotmark{*}};
            \node [text width=0.5cm] (c0) at (0,4) {$\alpha,\beta$};
            \node [text width=0.5cm] (a0) at (2.5,2.6) {$\alpha,\beta$};
            \node [circle,draw=black,fill=white,inner sep=0pt,minimum size=0.65cm] (tau_epsilon) at (2.5,1.6) {\large $\tau_{\epsilon}$};
            \path [draw,->] (u1) edge (x);
            \path [draw,->] (u3) edge (x);
            \path [draw,->] (lambda) edge (u1);
            \path [draw,->] (lambda) edge (u3);
            \path [draw,->] (c0) edge (lambda);
            \path [draw,->] (tau_epsilon) edge (x);
            \path [draw,->] (a0) edge (tau_epsilon);
            \plate [color=red] {part1} {(x)(u1)} { };
            \plate [color=blue] {part3} {(x)(u3)(part1.north east)} { };
            \node [text width=2cm] at (-0.6,-0.5) {\large $n_1$};
            \node [text width=2cm] at (2,-0.5) {\large $n_d$};
        \end{tikzpicture}
        \caption{BCPF}
    \end{center}
\end{figure}  

\begin{figure}[!h]
    \begin{center}
        \begin{tikzpicture}[x=0.75pt,y=0.75pt,yscale=-1,xscale=1]
            \node [obs] (o1) at (20,20) {$\boldsymbol{o}_1$};
            \node [circle,draw=black,fill=white, inner sep=0pt,minimum size=0.75cm] (p1) at (20,-60) {$z_1$};
            \node [circle,draw=black,fill=white, inner sep=0pt,minimum size=0.75cm] (p0) at (20,-130) {$z_0$};
            \node [obs] (o2) at (70,20) {$\boldsymbol{o}_2$};
            \node [circle,draw=black,fill=white, inner sep=0pt,minimum size=0.75cm] (p2) at (70,-60) {$z_2$};
            \node [obs] (o3) at (160,20) {$\boldsymbol{o}_T$};
            \node [circle,draw=black,fill=white, inner sep=0pt,minimum size=0.75cm] (p3) at (160,-60) {$z_T$};
            \node [circle,draw=black,fill=white,inner sep=0pt,minimum size=0.75cm] (theta) at (-60,-20) {$\theta_i$};
            \node [circle,draw=black,fill=white,inner sep=0pt,minimum size=0.75cm] (pi) at (-60,-100) {$\pi_i$};
            \node [circle,draw=black,fill=white,inner sep=0pt,minimum size=0.75cm] (beta) at (-60,-150) {$\beta$};
            \node [text width=2cm] (inf1) at (-57,10) {\small{$i\in\left\{1,...,\infty \right\}$}};
            \node [text width=2cm] (inf2) at (-57,-70) {\small{$i\in\left\{1,...,\infty \right\}$}};
            \node [text width=.2cm] (lambda) at (-130,-20) {$\lambda$};
            \node [text width=.2cm] (alpha) at (-130,-100) {$\alpha$};
            \node [text width=.2cm] (kappa) at (-130,-120) {$\kappa$};
            \node [text width=.2cm] (gamma) at (-130,-150) {$\gamma$};
            \node [text width=.2cm] (gamma) at (-130,-150) {$\gamma$};
            \node [text width=.4cm] (1dot) at (115,-60) {$...$};
            \node [text width=.4cm] (2dot) at (115,20) {$...$};
            \path [draw,->] (p2) edge (1dot);
            \path [draw,->] (1dot) edge (p3);
            \path [draw,->] (p1) edge (o1);
            \path [draw,->] (p2) edge (o2);
            \path [draw,->] (p3) edge (o3);
            \path [draw,->] (p1) edge (p2);
            \path [draw,->] (p0) edge (p1);
            \path [draw,->] (alpha) edge (pi);
            \path [draw,->] (kappa) edge (pi);
            \path [draw,->] (beta) edge (pi);
            \path [draw,->] (gamma) edge (beta);
            \path [draw,->] (lambda) edge (theta);
            \plate [color=black] {part2} {(theta)(inf1)} { };
            \plate [color=black] {part3} {(pi)(inf2)} { };
            \draw[black,->,thick] (pi.east) to [in=-150,out=-16] (147,-65);
            \draw[black,->,thick] (pi.east) to [in=-143,out=-5] (57,-65);
            \draw[black,->,thick] (pi.east) to [in=-135,out=5] (7,-65);
            \draw[black,->,thick] (theta.east) to [in=-150,out=-16] (147,15);
            \draw[black,->,thick] (theta.east) to [in=-143,out=-5] (57,15);
            \draw[black,->,thick] (theta.east) to [in=-135,out=5] (7,15);
        \end{tikzpicture}
        \caption{Sticky HDP-HMM}
    \end{center}
\end{figure}  

\begin{figure}[!h]
    \begin{center}
        \begin{tikzpicture}
            \draw [very thick] (0,0) rectangle (3.6/2,2.4/2);
            \filldraw [fill=green!20!white,draw=green!40!black] (0,0) rectangle (3.6/2,2.4/2);
            \filldraw [fill=white] (0.4/2,0.4/2) rectangle (0.8/2,0.8/2);
            \filldraw [fill=white] (2.4/2,0.4/2) rectangle (2.8/2,0.8/2);
            \filldraw [fill=white] (0.8/2,1.2/2) rectangle (1.2/2,1.6/2);
            \filldraw [fill=white] (2.0/2,1.6/2) rectangle (2.4/2,2.0/2);
            \filldraw [fill=white] (0.4/2,2.0/2) rectangle (0.8/2,2.4/2);
            \filldraw [fill=white] (2.4/2,2.0/2) rectangle (2.8/2,2.4/2);
            \filldraw [fill=white] (2.8/2,1.2/2) rectangle (3.2/2,2.0/2);
            \draw [step=0.4/2, very thin, color=gray] (0,0) grid (3.6/2,2.4/2);
            \draw (1.8/2,-0.3) node {{\color{red}\scriptsize{$Y\in\mathbb{R}^{m\times f}$}}};
            \draw (4.4/2,1.2/2) node {{\color{black}\large{$\approx$}}};
            \draw [very thick] (5.2/2,0) rectangle (6.0/2,2.4/2);
            \filldraw [fill=green!20!white,draw=green!40!black] (5.2/2,0) rectangle (6.0/2,2.4/2);
            \draw [step=0.4/2, very thin, color=gray] (5.2/2,0) grid (6.0/2,2.4/2);
            \draw (5.6/2,-0.3) node {{\color{black}\scriptsize{$W\in\mathbb{R}^{m\times r}$}}};
            \draw (6.8/2,1.2/2) node {{\color{black}\large{$\times$}}};
            \draw [very thick] (7.6/2,0.8/2) rectangle (11.2/2,1.6/2);
            \filldraw [fill=green!20!white,draw=green!40!black] (7.6/2,0.8/2) rectangle (11.2/2,1.6/2);
            \draw [step=0.4/2, very thin, color=gray] (7.6/2,0.8/2) grid (11.2/2,1.6/2);
            \draw (9.4/2,0) node {{\color{red}\scriptsize{$X^{T}\in\mathbb{R}^{r\times f}$}}};       
        \end{tikzpicture}
        \caption{矩阵分解}
    \end{center}
\end{figure}  

\newpage

\begin{lstlisting}
BCPF
\tikzset{>=latex}
\tikzstyle{plate caption} = [caption, node distance=0, inner sep=0pt,
below left=5pt and 0pt of #1.south]
\begin{figure}[!h]
    \begin{center}
        \begin{tikzpicture}
            \node [obs] (x) at (0,0) {\large $x_{\boldsymbol{i}}$};
            \node [circle,draw=black,fill=white,inner sep=0pt,minimum size=0.6cm] (u1) at (-1.2,1.6) { $\boldsymbol{u}_{i_1}^{(1)}$};
            \node [circle,draw=black,fill=white,inner sep=0pt,minimum size=0.6cm] (u3) at (1.2,1.6) { $\boldsymbol{u}_{i_d}^{(d)}$};
            \node [circle,draw=black,fill=white,inner sep=0pt,minimum size=0.65cm] (lambda) at (0,3.0) {\large $\boldsymbol{\lambda}$};
            \node[mark size=1pt,color=black] at (0,1.6) {\pgfuseplotmark{*}};
            \node[mark size=1pt,color=black] at (-0.2,1.6) {\pgfuseplotmark{*}};
            \node[mark size=1pt,color=black] at (0.2,1.6) {\pgfuseplotmark{*}};
            \node [text width=0.5cm] (c0) at (0,4) {$\alpha,\beta$};
            \node [text width=0.5cm] (a0) at (2.5,2.6) {$\alpha,\beta$};
            \node [circle,draw=black,fill=white,inner sep=0pt,minimum size=0.65cm] (tau_epsilon) at (2.5,1.6) {\large $\tau_{\epsilon}$};
            \path [draw,->] (u1) edge (x);
            \path [draw,->] (u3) edge (x);
            \path [draw,->] (lambda) edge (u1);
            \path [draw,->] (lambda) edge (u3);
            \path [draw,->] (c0) edge (lambda);
            \path [draw,->] (tau_epsilon) edge (x);
            \path [draw,->] (a0) edge (tau_epsilon);
            \plate [color=red] {part1} {(x)(u1)} { };
            \plate [color=blue] {part3} {(x)(u3)(part1.north east)} { };
            \node [text width=2cm] at (-0.6,-0.5) {\large $n_1$};
            \node [text width=2cm] at (2,-0.5) {\large $n_d$};
        \end{tikzpicture}
        \caption{BCPF}
    \end{center}
\end{figure}  
\end{lstlisting}

\newpage
\begin{lstlisting}
Sticky HDP-HMM
\begin{figure}[!h]
    \begin{center}
        \begin{tikzpicture}[x=0.75pt,y=0.75pt,yscale=-1,xscale=1]
            \node [obs] (o1) at (20,20) {$\boldsymbol{o}_1$};
            \node [circle,draw=black,fill=white, inner sep=0pt,minimum size=0.75cm] (p1) at (20,-60) {$z_1$};
            \node [circle,draw=black,fill=white, inner sep=0pt,minimum size=0.75cm] (p0) at (20,-130) {$z_0$};
            \node [obs] (o2) at (70,20) {$\boldsymbol{o}_2$};
            \node [circle,draw=black,fill=white, inner sep=0pt,minimum size=0.75cm] (p2) at (70,-60) {$z_2$};
            \node [obs] (o3) at (160,20) {$\boldsymbol{o}_T$};
            \node [circle,draw=black,fill=white, inner sep=0pt,minimum size=0.75cm] (p3) at (160,-60) {$z_T$};
            \node [circle,draw=black,fill=white,inner sep=0pt,minimum size=0.75cm] (theta) at (-60,-20) {$\theta_i$};
            \node [circle,draw=black,fill=white,inner sep=0pt,minimum size=0.75cm] (pi) at (-60,-100) {$\pi_i$};
            \node [circle,draw=black,fill=white,inner sep=0pt,minimum size=0.75cm] (beta) at (-60,-150) {$\beta$};
            \node [text width=2cm] (inf1) at (-57,10) {\small{$i\in\left\{1,...,\infty \right\}$}};
            \node [text width=2cm] (inf2) at (-57,-70) {\small{$i\in\left\{1,...,\infty \right\}$}};
            \node [text width=.2cm] (lambda) at (-130,-20) {$\lambda$};
            \node [text width=.2cm] (alpha) at (-130,-100) {$\alpha$};
            \node [text width=.2cm] (kappa) at (-130,-120) {$\kappa$};
            \node [text width=.2cm] (gamma) at (-130,-150) {$\gamma$};
            \node [text width=.2cm] (gamma) at (-130,-150) {$\gamma$};
            \node [text width=.4cm] (1dot) at (115,-60) {$...$};
            \node [text width=.4cm] (2dot) at (115,20) {$...$};
            \path [draw,->] (p2) edge (1dot);
            \path [draw,->] (1dot) edge (p3);
            \path [draw,->] (p1) edge (o1);
            \path [draw,->] (p2) edge (o2);
            \path [draw,->] (p3) edge (o3);
            \path [draw,->] (p1) edge (p2);
            \path [draw,->] (p0) edge (p1);
            \path [draw,->] (alpha) edge (pi);
            \path [draw,->] (kappa) edge (pi);
            \path [draw,->] (beta) edge (pi);
            \path [draw,->] (gamma) edge (beta);
            \path [draw,->] (lambda) edge (theta);
            \plate [color=black] {part2} {(theta)(inf1)} { };
            \plate [color=black] {part3} {(pi)(inf2)} { };
            \draw[black,->,thick] (pi.east) to [in=-150,out=-16] (147,-65);
            \draw[black,->,thick] (pi.east) to [in=-143,out=-5] (57,-65);
            \draw[black,->,thick] (pi.east) to [in=-135,out=5] (7,-65);
            \draw[black,->,thick] (theta.east) to [in=-150,out=-16] (147,15);
            \draw[black,->,thick] (theta.east) to [in=-143,out=-5] (57,15);
            \draw[black,->,thick] (theta.east) to [in=-135,out=5] (7,15);
        \end{tikzpicture}
        \caption{Sticky HDP-HMM}
    \end{center}
\end{figure}       
\end{lstlisting}
\newpage
\begin{lstlisting}
矩阵分解
\begin{figure}[!h]
    \begin{center}
        \begin{tikzpicture}
            \draw [very thick] (0,0) rectangle (3.6/2,2.4/2);
            \filldraw [fill=green!20!white,draw=green!40!black] (0,0) rectangle (3.6/2,2.4/2);
            \filldraw [fill=white] (0.4/2,0.4/2) rectangle (0.8/2,0.8/2);
            \filldraw [fill=white] (2.4/2,0.4/2) rectangle (2.8/2,0.8/2);
            \filldraw [fill=white] (0.8/2,1.2/2) rectangle (1.2/2,1.6/2);
            \filldraw [fill=white] (2.0/2,1.6/2) rectangle (2.4/2,2.0/2);
            \filldraw [fill=white] (0.4/2,2.0/2) rectangle (0.8/2,2.4/2);
            \filldraw [fill=white] (2.4/2,2.0/2) rectangle (2.8/2,2.4/2);
            \filldraw [fill=white] (2.8/2,1.2/2) rectangle (3.2/2,2.0/2);
            \draw [step=0.4/2, very thin, color=gray] (0,0) grid (3.6/2,2.4/2);
            \draw (1.8/2,-0.3) node {{\color{red}\scriptsize{$Y\in\mathbb{R}^{m\times f}$}}};
            \draw (4.4/2,1.2/2) node {{\color{black}\large{$\approx$}}};
            \draw [very thick] (5.2/2,0) rectangle (6.0/2,2.4/2);
            \filldraw [fill=green!20!white,draw=green!40!black] (5.2/2,0) rectangle (6.0/2,2.4/2);
            \draw [step=0.4/2, very thin, color=gray] (5.2/2,0) grid (6.0/2,2.4/2);
            \draw (5.6/2,-0.3) node {{\color{black}\scriptsize{$W\in\mathbb{R}^{m\times r}$}}};
            \draw (6.8/2,1.2/2) node {{\color{black}\large{$\times$}}};
            \draw [very thick] (7.6/2,0.8/2) rectangle (11.2/2,1.6/2);
            \filldraw [fill=green!20!white,draw=green!40!black] (7.6/2,0.8/2) rectangle (11.2/2,1.6/2);
            \draw [step=0.4/2, very thin, color=gray] (7.6/2,0.8/2) grid (11.2/2,1.6/2);
            \draw (9.4/2,0) node {{\color{red}\scriptsize{$X^{T}\in\mathbb{R}^{r\times f}$}}};       
        \end{tikzpicture}
        \caption{矩阵分解}
    \end{center}
\end{figure} 
\end{lstlisting}

\newpage
\setcounter{equation}{1}
\part{建立索引及文献引用}
科技论文或技术报告中的图片、表格、公式和参考文献往往会被编号,方便读者进行查看。
在实际写作过程中,有时图片、表格、公式会不在文本引用位置附近,而参考文献更是一般都总结放在文末结尾处。
因此,读者在阅读过程中,为了查看该处引用内容的详细信息,需要翻看全文查找被引用的内容。 
这个过程一定是非常繁琐低效、并且会影响读者的阅读流畅性。为此,建立索引和文献引用就非常有必要了。
建立索引一般指对文档中的图片、表格、公式等进行索引的设置,这个建立是自动编号完成;
而文献引用同样是对文中存在引用参考文献的地方自动编号建立索引。
因此,建立索引和文献引用的过程较为简单,并且建立索引和引用后的效果非常好,
可以实现读者根据索引和文献引用直接跳转至想要查看的内容,能够有效提高读者的阅读效率和流畅性。
\setcounter{section}{0}
\section{图表和公式的索引}
\subsection{公式的索引}
LaTeX中,公式的索引分为主要分为两个部分,一部分是给公式添加标签,可以使用\textbackslash label\{标签名\}命令。
可以使用equation环境插入带标签的公式;另一部分是在文档中进行引用,可以使用\textbackslash ref\{标签名\}命令。

(\ref{aaaaaa}) is a binary equation

(\ref{bbbbbb}) is a binary quadratic equation.

\begin{equation}
    \label{aaaaaa} x+y=2 
\end{equation}

\begin{equation}
    \label{bbbbbb} x^{2}+y^{2}=2  
\end{equation}

\begin{lstlisting}
使用\label{标签名}及\ref{标签名}在文中对公式进行索引。
\documentclass[12pt]{article}
\begin{document}
(\ref{eq1}) is a binary equation
(\ref{eq2}) is a binary quadratic equation.
\begin{equation}
x+y=2\label{eq1}
\end{equation}

\begin{equation}
x^{2}+y^{2}=2\label{eq2}
\end{equation}
\end{document}
\end{lstlisting}

\subsection{图形的索引}
插入图片需要使用graphicx宏包,图形的索引与公式的索引类似。同样分为两部分,
一部分是使用\textbackslash label\{标签名\}命令给添加图形标签,
另一部分是使用\textbackslash ref\{标签名\}在文档中进行引用。

Figure \ref{texx} is a photo of Latex.

\begin{figure}[!h]
\centering
\includegraphics[width = 0.8\textwidth]{LaTeX.png}
\caption{LaTex}
\label{texx}
\end{figure}

\subsection{表格的索引}
表格的索引与公式及图形的索引类似,同样分为两部分,一部分是使用\textbackslash label\{标签名\}
命令给添加表格标签。根据第5章,可以使用tabular和table两种环境制作带标签的表格;
另一部分是使用\textbackslash ref\{标签名\}在文档中进行引用。

Table~\ref{table71} shows the values of some basic functions.

\begin{table}[!h]
    \centering
    \caption{The values of some basic functions.}
    \begin{tabular}{l|cccc}
        \hline
        & $x=1$ & $x=2$ & $x=3$ & $x=4$ \\
        \hline
        $y=x$ & 1 & 2 & 3 & 4 \\
        $y=x^{2}$ & 1 & 4 & 9 & 16 \\
        $y=x^{3}$ & 1 & 8 & 27 & 64 \\
        \hline
    \end{tabular}
    \label{table71}% 索引标签
\end{table}
\newpage
\begin{lstlisting}
Table~\ref{table1} shows the values of some basic functions.

\begin{table}
    \centering
    \caption{The values of some basic functions.}
    \begin{tabular}{l|cccc}
        \hline
        & $x=1$ & $x=2$ & $x=3$ & $x=4$ \\
        \hline
        $y=x$ & 1 & 2 & 3 & 4 \\
        $y=x^{2}$ & 1 & 4 & 9 & 16 \\
        $y=x^{3}$ & 1 & 8 & 27 & 64 \\
        \hline
    \end{tabular}
    \label{table1}% 索引标签
\end{table}    
\end{lstlisting}


\section{创建超链接}

超链接指按内容链接,可以从一个文本内容指向文本其他内容或其他文件、网址等。
超链接可以分为文本内链接、网页链接以及本地文件链接。\LaTeX 提供了hyperref宏包,
可用于生成超链接。在使用时,只需在前导代码中申明宏包即可,即\textbackslash usepackage\{hyperref\}。

\subsection{超链接类型}
\subsubsection{文本内链接}
文本内链接
在篇幅较大的文档中,查阅内容会比较繁琐,因此,往往会在目录中使用超链接来进行文本内容的快速高效浏览。可以使用hyperref宏包创建文本内超链接。
在导入 hyperref 时必须非常小心,一般而言,它必须是最后一个要导入的包。

\begin{lstlisting}
使用\usepackage{hyperref}创建一个简单的目录链接文本内容的例子。

\documentclass{book}
\usepackage{blindtext}
\usepackage{hyperref} %超链接包

\begin{document}

\frontmatter
\tableofcontents
\clearpage

\addcontentsline{toc}{chapter}{Foreword}
{\huge {\bf Foreword}}

This is foreword.
\clearpage

\mainmatter

\chapter{First Chapter}

This is chapter 1.


\clearpage

\section{First section} \label{second}

This is section 1.1.
\end{document}
\end{lstlisting}

\vspace{-1cm}
\subsubsection{网址链接}
众所周知,在文档中插入网址之类的文本同样需要用到超链接,同样的,使用hyperref宏包可以创建网页超链接。
有时我们需要将超链接命名并隐藏网址,这时我们可以使用href命令进行插入;有时,我们插入的网址链接太长,
但LaTeX不会自动换行,往往会造成格式混乱的问题,这时,我们可以使用url工具包,
并在该工具包中声明一个参数即可解决这个问题,相关命令为\textbackslash usepackage[hyphens]\{url\}。

\begin{lstlisting}
在LaTeX中使用hyperref及url工具包插入网页链接并设置自动换行。
\documentclass[12pt]{article}
\usepackage[hyphens]{url}
\usepackage{hyperref}

\begin{document}

This is the website of open-source latex-cookbook repository: \href{https://github.com/xinychen/latex-cookbook}{LaTeX-cookbook} or go to the next url: \url{https://github.com/xinychen/latex-cookbook}.

\end{document}
\end{lstlisting}

\subsubsection{本地文件链接}

\begin{lstlisting}
在LaTeX中使用href命令打开本地文件。
\documentclass[12pt]{article}
\usepackage[hyphens]{url}
\usepackage{hyperref}

\begin{document}

This is the text of open-source latex-cookbook repository: \href{run:./LaTeX-cookbook.dox}{LaTeX-cookbook}.

\end{document}
\end{lstlisting}
\subsection{超链接格式}

当然,有时候为了突出超链接,也可以在工具包hyperref中设置特定的颜色,设置的命令为\textbackslash hypersetup,一般放在前导代码中,
例如colorlinks = true, linkcolor=blue, urlcolor = blue, filecolor=magenta。
默认设置以单色样式的空间字体打印链接,\textbackslash urlstyle\{same\}命令将改变这个样式,并以与文本其余部分相同的样式显示链接。

\begin{lstlisting}
在LaTeX中使用hyperref工具包插入超链接并设置超链接颜色为蓝色。
\documentclass{book}
\usepackage{blindtext}
\usepackage{hyperref} %超链接包
\hypersetup{colorlinks = true, %链接将被着色,默认颜色是红色
            linkcolor=blue, % 内部链接显示为蓝色
            urlcolor = cyan, % 网址链接为青色
            filecolor=magenta} % 本地文件链接为洋红色
\urlstyle{same}

\begin{document}

\frontmatter
\tableofcontents
\clearpage

\addcontentsline{toc}{chapter}{Foreword}
{\huge {\bf Foreword}}

This is foreword.
\clearpage

\mainmatter

\chapter{First Chapter}

This is chapter 1.
\clearpage

\section{First section} \label{second}

This is section 1.1.

This is the website of open-source latex-cookbook repository: \href{https://github.com/xinychen/latex-cookbook}{LaTeX-cookbook} or go to the next url: \url{https://github.com/xinychen/latex-cookbook}.

This is the text of open-source latex-cookbook repository: \href{run:./LaTeX-cookbook.dox}{LaTeX-cookbook} 

\end{document}
\end{lstlisting}

\section{Bibtex用法}

LaTeX主要有两种管理参考文献的方法,第一种方法是在.tex文档中嵌入参考文献,参考文献格式需符合特定的文献引用格式;另一种方法则是使用 BibTeX进行文献管理,文件的拓展名为.bib。其中,使用外部文件BibTeX管理文献更加便捷高效。

\subsection{创建参考文献}
在\LaTeX 中,插入参考文献的一种直接方式是使用thebibliography环境,以列表的形式将参考文献进行整理起来,
配以标签,以供正文引用,文档中引用的命令为\textbackslash cite\{\}

Some examples for showing how to use \texttt{thebibliography} environment:
\begin{itemize}
    \item Book reference sample: The \LaTeX\ companion book \cite{latexcompanion}.
    \item Paper reference sample: On the electrodynamics of moving bodies \cite{einstein}.
    \item Open-source reference sample: Knuth: Computers and Typesetting \cite{knuthwebsite}.
\end{itemize}

\begin{thebibliography}{9}
\bibitem{latexcompanion} 
Michel Goossens, Frank Mittelbach, and Alexander Samarin. 
\textit{The \LaTeX\ Companion}. 
Addison-Wesley, Reading, Massachusetts, 1993.

\bibitem{einstein} 
Albert Einstein. 
\textit{Zur Elektrodynamik bewegter K{\"o}rper}. (German) 
[\textit{On the electrodynamics of moving bodies}]. 
Annalen der Physik, 322(10):891–921, 1905.

\bibitem{knuthwebsite} 
Knuth: Computers and Typesetting,
\\\texttt{http://www-cs-faculty.stanford.edu/\~{}uno/abcde.html}
\end{thebibliography}

\begin{lstlisting}
使用thebibliography环境在文档中插入参考文献并进行编译。
\documentclass[12pt]{article}
\begin{document}

Some examples for showing how to use \texttt{thebibliography} environment:
\begin{itemize}
    \item Book reference sample: The \LaTeX\ companion book \cite{latexcompanion}.
    \item Paper reference sample: On the electrodynamics of moving bodies \cite{einstein}.
    \item Open-source reference sample: Knuth: Computers and Typesetting \cite{knuthwebsite}.
\end{itemize}

\begin{thebibliography}{9}
\bibitem{latexcompanion} 
Michel Goossens, Frank Mittelbach, and Alexander Samarin. 
\textit{The \LaTeX\ Companion}. 
Addison-Wesley, Reading, Massachusetts, 1993.

\bibitem{einstein} 
Albert Einstein. 
\textit{Zur Elektrodynamik bewegter K{\"o}rper}. (German) 
[\textit{On the electrodynamics of moving bodies}]. 
Annalen der Physik, 322(10):891–921, 1905.

\bibitem{knuthwebsite} 
Knuth: Computers and Typesetting,
\\\texttt{http://www-cs-faculty.stanford.edu/\~{}uno/abcde.html}
\end{thebibliography}
\end{document}
\end{lstlisting}


\subsection{使用BibTeX文件}

BibTeX是\LaTeX 最为常用的一个文献管理工具,它通常以一个独立的文件出现,其拓展名为.bib。
它是伴随着\LaTeX 文档排版系统出现的,1985年兰伯特博士与其合作者开发了这一工具。
作为一种特殊的且独立于\LaTeX 文件.tex之外的数据库,它能大大简化\LaTeX 文档中的文献引用。
实际上,BibTeX文件中的文献都是用列表的形式罗列的,且不分先后顺序。
通过使用引用命令如\textbackslash cite\{\}即可在文档中自动生成特定格式的参考文献,其中,文档中的参考文献格式一般是提前设定好的。

\begin{lstlisting}
使用Bibtex命令一个文献管理文件为sample.bib,将文献按照指定格式进行整理,插入参考文献并进行编译。
% 创建Bibtex文件,并将其命名为sample.bib
@article{einstein,
    author =       "Albert Einstein",
    title =        "{Zur Elektrodynamik bewegter K{\"o}rper}. ({German})
        [{On} the electrodynamics of moving bodies]",
    journal =      "Annalen der Physik",
    volume =       "322",
    number =       "10",
    pages =        "891--921",
    year =         "1905",
    DOI =          "http://dx.doi.org/10.1002/andp.19053221004"
}

@book{latexcompanion,
    author    = "Michel Goossens and Frank Mittelbach and Alexander Samarin",
    title     = "The \LaTeX\ Companion",
    year      = "1993",
    publisher = "Addison-Wesley",
    address   = "Reading, Massachusetts"
}

@misc{knuthwebsite,
    author    = "Donald Knuth",
    title     = "Knuth: Computers and Typesetting",
    url       = "http://www-cs-faculty.stanford.edu/\~{}uno/abcde.html"
}

在这三条文献中,einstein、latexcompanion、knuthwebsite是文献的标签,在文档中,只需要在适当位置用引用命令如\cite{}便可以引用这些文献,例如,\cite{einstein}。

\documentclass[12pt]{article}

\begin{document}

Some examples for showing how to use \texttt{thebibliography} environment:
\begin{itemize}
    \item Book reference sample: The \LaTeX\ companion book \cite{latexcompanion}.
    \item Paper reference sample: On the electrodynamics of moving bodies \cite{einstein}.
    \item Open-source reference sample: Knuth: Computers and Typesetting \cite{knuthwebsite}.
\end{itemize}

\bibliographystyle{unsrt}
\bibliography{sample}

\end{document}
\end{lstlisting}

在sample.bib文件中,根据文献类型可定义文献列表,
对于每篇文献,需要整理author(作者信息)、title(文献标题)等基本信息。
在\LaTeX 文档中,我们需要使用\textbackslash bibliographystyle命令申明参考文献的格式,
如本例中的unsrt,同时,使用\textbackslash bibliography命令申明参考文献的源文件,即sample.bib。

当然,BibTeX文献管理也有很多优点:
\begin{itemize}
    \item 无需重复输入每条参考文献。文献放在BibTeX之后,引用文献的标签即可在文档中显示参考文献。
    \item 文档中的参考文献格式是根据文档样式自动设置的,且所有文献的引用风格是一致的。
    \item 引用同一作者同年的文献过多时,引用格式会自动调整。
    \item BibTeX文件中的文献只有在文档中明确引用才会显示在文档的参考文献中。
\end{itemize}

在BibTeX文件中,不同类型的文献是需要进行分类的:
\begin{itemize}
    \item article:对应着期刊或杂志上发表的论文,必须添加的信息有author(作者)、title(标题)、journal(期刊)、year(年份)、volume(卷),可供选择添加的信息包括number(期)、pages(页码)、month(月份)、doi(数字对象识别码)等。
    \item book:对应着书籍,必须添加的信息有author/editor(作者或主编)、title(书名)、publisher(出版社)、year(年份),可供选择添加的信息包括volume/number(卷/期)、series(系列)、address(出版地址)、edition(版号)、month(月份)、url(网址)等。
    \item inbook:书籍中的一部分或者某一章节,必须添加的信息有author/editor、title(标题)、chapter/pages(章节/页码)、publisher(出版社)、year(年份),其他可供选择添加的信息与book一致。
    \item inproceedings:对应着会议论文,必须添加的信息有author(作者)、title(论文标题)、booktitle(论文集标题)、year(年份),可供选择添加的信息包括editor(版号)、volume/number(卷或期)、series(系列)、pages(页码)、address(地址)、month(月份)、organization(组织方)、publisher(出版社)等。
    \item conference:对应着会议论文,与inproceedings用法一致。
    \item mastersthesis和proceedings:分别对应着硕士学位论文和博士学位论文,必须添加的信息有author(作者)、title(标题)、school(学校或研究机构)、year(年份)
\end{itemize}

\section{文献引用格式}

Bibtex的最大特点是采用了标准化的数据库,对于论文、著作以及其他类型的文献,我们可以自定义文献的引用格式。Bibtex的样式会改变所引用文献的引用格式。

一般而言,LaTeX中有一系列标准样式 (standard styles) 可供选择和使用。具体而言,这些标准样式对应的文件包括:
\begin{itemize}
    \item plain.bst
    \item acm.bst:对应于Association for Computing Machinery期刊
    \item ieeetr.bst:对应于IEEE Transactions期刊
    \item alpha.bst
    \item abbrv.bst
    \item siam.bst:对应于SIAM
\end{itemize}

当然,实际上还有很多.bst文件,这里给出的几个文件只是最为常用的。不得不提的是natbib工具包,这一工具包对一系列引用命令进行了标准化,而这种标准化不受不同文献样式的影响。

\newpage
\part{幻灯片制作}
\setcounter{section}{0}

{\Large *该部分代码未经全部验证,故不能保证正确运行*}

2003年,柏林工业大学的博士生Till Tantau发布了一款用于制作演示文稿的工具Beamer。Beamer是Till Tantau利用业余时间开发的,他的初衷是方便自己使用LaTeX制作幻灯片,不过出乎意料的是,后来Beamer的受欢迎程度完全超出了他的想象。在Till Tantau开发Beamer期间,他收到了很多人的建议和反馈,这些都直接推动了开发工作。2010年,Till Tantau将Beamer交由他人维护和管理。

Beamer作为LaTeX的一种特殊文档类型,它的出现无疑丰富了LaTeX制作演示文稿的功能。虽然Beamer并非LaTeX中第一款用于制作演示文稿的工具,但直到今日,Beamer却是最受大家欢迎的一款。Beamer的使用方式简单灵活,只需设定LaTeX文档类型为beamer,便可开始创作。同时,Beamer提供了大量的幻灯片模板,这些模板包含了丰富多彩的视觉效果,创作者可以直接使用。毫不夸张地说,Beamer的出现极大地提高了人们使用LaTeX制作幻灯片的热情。值得一提的是,在2005年,Till Tantau又发布了一款非常便捷的LaTeX绘图工具TikZ。TikZ不仅可以辅助Beamer幻灯片的制作,也可以应用于科技文档中的各类绘图任务。

Beamer是随着LaTeX的发展而衍生出来的一种特殊文档类型,也常常被看作是一个功能强大的宏包,可以支撑科研工作者制作幻灯片的需求。使用Beamer制作幻灯片与Office办公软件(如PowerPoint)完全不同,从本质上来说,使用Beamer制作幻灯片其实和LaTeX中的其他文档类型没有太大区别:代码结构都是由前导代码和主体代码组成,完全沿用了LaTeX的文档环境与基本命令。因此,使用Beamer制作幻灯片也有一个缺点,那就是必须掌握LaTeX制作文档的基础。

在呈现方式上,Beamer制作的幻灯片最终会在LaTeX中被编译成PDF文档,在不同的系统上打开幻灯片时不存在不兼容等问题。在功能上,使用Beamer制作幻灯片时,我们可以对常规文本、公式、列表、图表甚至动画效果、视觉效果和主题样式等进行调整,最终达到想要的视觉效果。

事实上,LaTeX中用于制作演示文稿的工具并非只有Beamer,但相比其他工具,Beamer具有如下优点:
\begin{itemize}
    \item 拥有海量的模板和丰富的主题样式,且使用方便;
    \item 能满足制作幻灯片的功能性需求,从创建标题、文本和段落到插入图表、参考文献等操作,且与常规文档中的使用规则几乎一致;
    \item 使用方式简单,在主体代码中使用\textbackslash begin\{frame\} \textbackslash end\{frame\}环境就能创建一页幻灯片。
\end{itemize}

\section{基本介绍}

\subsection{Beamer简介}
在上述章节中,我们主要介绍了LaTeX中比较常用的文档类型article,可用于创建期刊论文、技术报告等。
本章中我们将介绍另一种文档类型:beamer。
Beamer的开发者Till Tantau说,“BEAMER is a LATEX class for creating presentations”,显然,
Beamer是一种用于制作演示文稿或者幻灯片的文档类型。

从使用角度来说,beamer文档类型和book、article等文档类型一样,
都是在以.tex为拓展名的文件上编写程序和文档内容,然后再通过编译生成PDF文档。
当然,Beamer也兼具常用演示文稿如PowerPoint的主要功能,可以自行设置动态效果、
甚至使用主题样式修改幻灯片的外观。

与其他文档类型相似的是,beamer文档类型中拥有很多视觉效果极好的模板,这些模版已经设置好了特定的主题样式,
有时候甚至只需要加入创作内容即可得到心仪的幻灯片。使用Beamer制作幻灯片时,
我们可以体验LaTeX排版论文的几乎所有优点,公式排版、图表排版、参考文献设置等也非常便捷,
有时候甚至可以将常规文档中的内容直接复制到Beamer文档类型中,稍加调整便能得到样式合适的幻灯片。
另外,我们也可以根据需要,在前导代码中使用全局设置调整幻灯片的主题样式、颜色主题、字体主题等。

使用beamer制作幻灯片仍然遵循着LaTeX的一般使用方法,代码结构分为前导代码和主体代码,
前导代码除了申明文档类型为beamer外,即\textbackslash documentclass\{beamer\},
调用宏包等与常规文档的制作基本是一致的。

\begin{lstlisting}
使用beamer文档类型创建一个简单的幻灯片。
\documentclass{beamer}

\title{A Simple Beamer Example}
\author{Author's Name}
\institute{Author's Institute}
\date{\today} 

\begin{document}

\frame{\titlepage}

\end{document}
\end{lstlisting}

在例子中,\textbackslash title\{\}、\textbackslash author\{\}和\textbackslash date\{\}这几个命令分别对应着标题、作者以及日期,一般放在标题页,如果想在幻灯片首页显示这些信息,可以在使用\textbackslash frame\{\textbackslash titlepage\}命令新建标题页。

总结来说,标题及作者信息对应的特定命令包括:
\begin{itemize}
    \item 标题:对应的命令为\textbackslash title[A]\{B\},其中,位置A一般填写的是简化标题,而位置B则填写的是完整标题,这里的完整标题有时候可能会很长。
    \item 副标题:对应的命令为\textbackslash subtitle[A]\{B\},其中,位置A一般填写的是简化副标题,而位置B则填写的是完整副标题,这里的完整副标题有时候也可能会很长。
    \item 作者:对应的命令为\textbackslash author[A]\{B\},用法类似。
    \item 日期:对应的命令为\textbackslash date[A]\{B\},用法类似。
    \item 单位:对应的命令为\textbackslash institution[A]\{B\},用法类似。
\end{itemize}

我们知道,在常规文档article中,申明文档类型时可以指定正文字体大小,
在文档类型的申明语句\textbackslash documentclass\{beamer\}中,
我们也可以通过特定选项调整幻灯片内容的字体大小,一般默认为11pt,
我们也可以根据需要使用8pt、9pt、10pt、12pt、14pt、17pt、20pt字体大小,
例如使用\textbackslash documentclass[12pt]\{beamer\}可以将字体大小设置为12pt。

制作幻灯片时,有时候为了达到特定的投影效果,会设置幻灯片的长宽比例,
比较常用的两种长宽比例分别为4:3和16:9。
一般来说,Beamer制作出来的幻灯片默认大小为长128毫米、宽96毫米,对应着默认的长宽比例4:3,
有时候,我们也可以根据需要将幻灯片的长宽比例调整为16:9、14:9、5:4甚至3:2。

\begin{lstlisting}
使用beamer文档类型创建一个简单的幻灯片,将幻灯片的长宽比例调整为16:9。
在例子中,选项aspectratio对应着长宽比例,数字169对应着长宽比例16:9,类似地,149、54、32分别对应着长宽比例14:9、5:4、3:2。
\documentclass[aspectratio = 169]{beamer}

\title{A Simple Beamer Example}
\author{Author's Name}
\institute{Author's Institute}
\date{\today} 

\begin{document}

\frame{\titlepage}

\end{document}
\end{lstlisting}

\subsection{创建幻灯片}
frame这个词在计算机编程中非常常见,这一英文单词的字面意思可以翻译为“帧”,
假如我们将幻灯片视作“连环画”,是由一页一页单独的幻灯片组成,
那么每一页幻灯片则对应着连环画中的帧。使用Beamer制作幻灯片时,幻灯片就是用frame环境创建出来的,
然而,有时候为了让幻灯片产生动画视觉效果,Beamer中的帧(即frame)与每页幻灯片并非严格意义上的一一对应。

在beamer文档类型中,制作幻灯片的环境一般为\textbackslash begin\{frame\} \textbackslash end\{frame\}。\,
在\textbackslash begin\{document\} \textbackslash end\{document\}构成的主体代码中,一个frame环境一般对应着一页幻灯片。

每张幻灯片一般都有一个标题,有时也会有一个副标题。若要创建标题和副标题,
用户可以通过使用\textbackslash begin\{frame\}\{\}\{\}的命令格式,其中第一、二个\{\}中分别为幻灯片的标题和副标题;
此外,用户也可以通过在frame环境中,使用\textbackslash frametitle\{\}和\textbackslash framesubtitle\{\}命令分别创建标题和副标题。
由此创建的标题和副标题一般位于幻灯片的顶部,标题相对于副标题字体稍大一点。

实际上,Beamer与其他文档类型并没有特别大的差异,常规文档中的基本列表环境都可以在Beamer中使用,
包括:有序列表环境\textbackslash begin\{enumerate\} \textbackslash end\{enumerate\}、无序列表环境\textbackslash begin\{itemize\} \textbackslash end\{itemize\}以及解释性列表环境\textbackslash begin\{description\} \textbackslash end\{description\}。

\begin{lstlisting}
使用beamer文档类型中的\begin{frame} \end{frame}环境、\frametitle{}和\framesubtitle{}命令创建一个简单的幻灯片。
有时为了简化代码,也可以直接用\frame{}命令取代\begin{frame} \end{frame}环境囊括幻灯片内容
\documentclass{beamer}
\usefonttheme{professionalfonts}
\begin{document}
\begin{frame}
\frametitle{Parent function}
\framesubtitle{A short list}
Please check out the following parent function list.
\begin{enumerate}
\item $y=x$
\item $y=|x|$
\item $y=x^{2}$
\item $y=x^{3}$
\item $y=x^{b}$
\end{enumerate}
\end{frame}
\end{document}
\end{lstlisting}

使用Beamer制作幻灯片时,幻灯片内容会在标题下方自动居中对齐,如果想调整对其方式,可以在\textbackslash begin\{frame\} \textbackslash end\{frame\}环境中设置参数,具体而言,有以下几种:
\begin{enumerate}
    \item \textbackslash begin\{frame\}[c] \textbackslash end\{frame\}是居中对齐,字母c对应着英文单词center的首字母,一般而言,[c]作为默认参数,无需专门设置;
    \item \textbackslash begin\{frame\}[t] \textbackslash end\{frame\}中的[t]可以让幻灯片内容进行顶部对齐,其中,字母t对应着英文单词top的首字母;
    \item \textbackslash begin\{frame\}[b] \textbackslash end\{frame\}中的[b]可以让幻灯片内容进行底部对齐,其中,字母b对应着英文单词bottom的首字母。
\end{enumerate}

\begin{lstlisting}
使用beamer文档类型中的\begin{frame} \end{frame}环境创建一个简单的幻灯片,并让幻灯片内容进行顶部对齐。
\documentclass{beamer}
\usefonttheme{professionalfonts}
\begin{document}
\begin{frame}[t]
\frametitle{Parent function}
\framesubtitle{A short list}
Please check out the following parent function list.
\begin{enumerate}
\item $y=x$
\item $y=|x|$
\item $y=x^{2}$
\item $y=x^{3}$
\item $y=x^{b}$
\end{enumerate}
\end{frame}
\end{document}
\end{lstlisting}

\subsection{创建章节与生成目录}
类似article文档类,beamer中可以利用\textbackslash part\{\}、\textbackslash section\{\}、\textbackslash subsection\{\}、以及\textbackslash subsubsection\{\}等命令构建演示稿中的章节层次,但此时\textbackslash chapter\{\}命令无效。其中,章节标题写在\{\}中,但编译后不会出现在创建章节的位置,仅在目录和导航条中显示。类似地,可以通过加*号使得章节标题不出现在目录中,但仍然会在导航条中显示。

在beamer中,可以使用\textbackslash tableofcontents命令自动生成演示稿目录,
通过在frame幻灯片页中添加该命令即可。由此生成的目录实际上是超链接,点击之后会自动跳转到相应章节

\begin{lstlisting}
在beamer文档类型中使用\tableofcontents命令为幻灯片生成目录,并使用\section{}和\subsection{}创建章节。
\documentclass{beamer}

\begin{document}

\begin{frame}{Table of contents}
\tableofcontents
\end{frame}

\section{Section A}
\begin{frame}{frame1}
\subsection{a1}
This is subsection a1. This is subsection a1.
\subsection{a2}
This is subsection a2. This is subsection a2.
\subsection{a3}
This is subsection a3. This is subsection a3.
\end{frame}

\section{Section B}
\begin{frame}{frame2}
\subsection{b1}
This is subsection b1. This is subsection b1. % 在下方插入空行,使得内容分行显示.

\subsection{b2}
This is subsection b2. % 在下方插入空行,使得内容分行显示.

This is subsection b2.
\end{frame}

\section*{Section C}
\begin{frame}{frame3}
\subsection*{c1}
This is subsection c1. This is subsection c1. % 在下方插入空行,使得内容分行显示.

\subsection*{c2}
This is subsection c2. This is subsection c2.
\end{frame}
\end{document}
\end{lstlisting}

如果想让相邻章节或者同章节的内容分行显示,只需要在相应位置插入空行即可。

默认情况下,目录页中包含所有不含*号的章节标题,甚至是三级节标题。
但有时目录只需要显示到一级节标题即可,而二级节标题及其次级标题则不需要显示,为此,
只需要在\textbackslash tableofcontents命令后设置选项[hideallsubsections]即可。

\begin{lstlisting}
在beamer文档类型中使用\tableofcontents[hideallsubsections]命令为幻灯片生成一级节目录。
\documentclass{beamer}

\begin{document}

\begin{frame}{table of contents}
\tableofcontents[hideallsubsections]
\end{frame}

\section{Section A}
\begin{frame}{frame1}
\subsection{a1}
This is subsection a1. This is subsection a1.

\subsection{a2}
This is subsection a2. This is subsection a2.

\subsection{a3}
This is subsection a3. This is subsection a3.

\end{frame}

\section{Section B}
\begin{frame}{frame2}
\subsection{b1}
This is subsection b1. This is subsection b1.

\subsection{b2}
This is subsection b2. This is subsection b2.
\end{frame}

\end{document}
\end{lstlisting}

一般而言,使用\textbackslash tableofcontents命令生成的目录只会显示在相应的幻灯片页。
有时候为了更好地梳理演示稿脉络,需要在各章节前均插入目录页,为此,
一种更简便的方式是使用\textbackslash AtBeginSection\{\}、\textbackslash AtBeginSubsection\{\}、
或\textbackslash AtBeginSubsubsection\{\}命令分别在一级节、二级节、三级节前均插入目录页。
此外,使用\textbackslash tableofcontents[currentsection]命令或\textbackslash tableofcontents[currentsubsection]命令可以在各章节前的目录页中突出显示当前一级节标题或二级节标题。

\begin{lstlisting}
在beamer文档类型中使用\AtBeginSection{}以及\tableofcontents[currentsection]命令在幻灯片的各一级节前均插入目录页,并突出显示当前一级节标题。
\documentclass{beamer}
\begin{document}
\AtBeginSection
{
\begin{frame}{table of contents}
\tableofcontents[currentsection]
\end{frame}
}

\section{Section A}
\begin{frame}{frame1}
\subsection{a1}
This is subsection a1. This is subsection a1.

\subsection{a2}
This is subsection a2. This is subsection a2.

\subsection{a3}
This is subsection a3. This is subsection a3.
\end{frame}

\section{Section B}
\begin{frame}{frame2}
\subsection{b1}
This is subsection b1. This is subsection b1.

\subsection{b2}
This is subsection b2. This is subsection b2.
\end{frame}

\section{Section C}
\begin{frame}{frame3}
\subsection{c1}
This is subsection c1. This is subsection c1.

\subsection{c2}
This is subsection c2. This is subsection c2.
\end{frame}
\end{document}
\end{lstlisting}

生成目录时,自定义目录显示的动画格式,通过使用\textbackslash tableofcontents[pausesections]命令,
同时在前导代码中申明\textbackslash setbeamercovered{dynamic}语句即可。

\begin{lstlisting}
在beamer文档类型中使用\tableofcontents命令生成幻灯片的目录,同时使用\tableofcontents[pausesections]对目录设置动画格式。
\documentclass{beamer}
\setbeamercovered{dynamic}

\begin{document}

\begin{frame}
\frametitle{Table of Contents}

\tableofcontents[pausesections]

\end{frame}

\section{Intro to Beamer}
\subsection{About Beamer}
\subsection[Basic Structure]{Basic Structure}
\subsection{How to Compile}
\section{Overlaying Concepts}
\subsection{Specifications}
\subsection[Examples]{Examples: Lists, Graphics, Tables}
\section[Sparkle]{Adding that Sparkle}
\subsection{Sections}
\subsection{Themes}
\section*{References}

\begin{frame}

\end{frame}

\end{document}
\end{lstlisting}


\subsection{幻灯片分栏}
对幻灯片内容进行分栏有两种常用方式,第一种是使用multicol宏包中的\\ \textbackslash begin\{multicols\}\{A\} 
\textbackslash end\{multicols\}环境,其中位置A可用于设定分栏列数;第二种是使用\textbackslash begin\{columns\} \textbackslash end\{columns\}环境。
\newpage
\begin{lstlisting}
在beamer文档类型中使用multicol宏包对列表内容进行分栏处理。

\documentclass{beamer}
\usefonttheme{professionalfonts}
\usepackage{multicol}

\begin{document}

\begin{frame}
\frametitle{Parent function}
\framesubtitle{A short list}

Please check out the following parent function list.
\begin{enumerate}
\begin{multicols}{3}
\item $y=x$
\item $y=|x|$
\item $y=x^{2}$
\item $y=x^{3}$
\item $y=x^{b}$
\end{multicols}
\end{enumerate}

\end{frame}

\end{document}
\end{lstlisting}

\begin{lstlisting}
在beamer文档类型中使用\begin{columns} \end{columns}环境对幻灯片内容进行分栏处理。
\documentclass{beamer}
\usefonttheme{professionalfonts}

\begin{document}

\begin{frame}
\frametitle{Parent function}
\framesubtitle{A short list}

\begin{columns}
\begin{column}{0.5\textwidth}

Please check out the following parent function list.
\begin{enumerate}
\item $y=x$
\item $y=|x|$
\item $y=x^{2}$
\item $y=x^{3}$
\item $y=x^{b}$
\end{enumerate}

\end{column}

\begin{column}{0.5\textwidth}

Please check out the following parent function list.
\begin{enumerate}
\item $y=x$
\item $y=|x|$
\item $y=x^{2}$
\item $y=x^{3}$
\item $y=x^{b}$
\end{enumerate}

\end{column}
\end{columns}

\end{frame}

\end{document}
\end{lstlisting}

\section{添加动画效果}
在制作幻灯片时有时需要添加动画效果。由于LaTeX制作幻灯片会被编译成PDF文档,因此,在Beamer中,实现动画效果的方式是将具有动画内容的幻灯片按照次序拆分成若干页内容,在播放时通过翻页达到“动态”视觉效果。为了便于说明,以下将一个frame环境创建的内容称为一页幻灯片或幻灯片页、将动画效果拆分后得到的每一页内容称为该幻灯片的某一帧。

\subsection{\textbackslash pause 命令}。
\textbackslash pause是Beamer中最常用的一种动画效果命令,它的使用方式极其简单,
通过在文本或段落中添加\textbackslash pause命令,便可将一页幻灯片拆分成若干帧。
一般来说,\textbackslash pause命令后的内容将会在下一帧中显示,从而使幻灯片在内容显示上呈现出动画效果。
比如,一般情况下,使用列表环境创建的每项内容(使用\textbackslash item创建)都会在同一帧幻灯片中显示,
为了达到各项内容逐个显示的动画效果,可以在两个相邻的\textbackslash item语句之间插入\textbackslash pause命令。

\begin{lstlisting}
在beamer文档类型中使用\pause命令实现一个简单的动画效果。
\documentclass{beamer}
\usefonttheme{professionalfonts}

\begin{document}

\begin{frame}
\frametitle{Parent function}
\framesubtitle{A short list}

Please check out the following parent function list.
\begin{enumerate}
\item $y=x$
\pause
\item $y=|x|$
\pause
\item $y=x^{2}$
\pause
\item $y=x^{3}$
\pause
\item $y=x^{b}$
\end{enumerate}
\end{frame}

\end{document}
\end{lstlisting}


\subsection{\textbackslash item<> 命令}
对于列表环境中的各项内容,也可以通过设置选项指定在该幻灯片的哪些步骤中显示该项内容,
从而更加灵活地定制动画效果。具体是使用\textbackslash item<>命令,其中<>用于指定显示步骤,
对于没有指定<>显示范围的内容项默认会在所有幻灯片页面中显示。具体而言,<>中的内容存在以下四种格式:
\begin{itemize}
    \item <A,B,C>:表示内容项将在第A、B、C步中显示。如,\textbackslash item<2, 3, 4> $y=x^{2}$表示该项内容将出现在该页幻灯片放映的第2、3、4步;
    \item <A-B>:表示内容项将在第A至B步中显示。如,\textbackslash item<1-4> $y=x$表示该项内容将出现在该页幻灯片放映的第1~4步;
    \item <A->:表示内容项将在第A步及以后显示。如,\textbackslash item<2-> $y=x$表示该项内容将出现在该页幻灯片放映的第2步及之后的步骤中;  
    \item <-A>:表示内容项将在第A步及之前显示。如,\textbackslash item<-3> $y=x$表示该项内容将出现在该页幻灯片放映的第3步及之前的步骤中。
\end{itemize}

如果想要在某一帧中突出某项内容:

使用\textbackslash alert命令为该项内容指定需要高亮显示的步骤。具体用法如: \\ \textbackslash item<2-| alert@3-4> The second item.,
此时内容项“The second item.”将出现在第2步之后的步骤中,并通过命令\textbackslash alert及前缀@使其在第3~4步中高亮显示。

使用\textbackslash color<范围>\{显示颜色\}命令改变内容项的颜色。
如\\ \textbackslash item<1-> \textbackslash color<1>\{red\} The first item.语句,内容The first item.将出现在第一步及之后的步骤中,
通过\textbackslash color<1>\{red\}令该项内容在第一步显示颜色为红色,而在其它步骤中仍为默认颜色。

\begin{lstlisting}
在beamer文档类型中使用\item<>定制内容显示范围并使用\alert对内容项进行高亮显示,从而实现一个简单的动画效果。
\documentclass{beamer}
\usefonttheme{professionalfonts}

\begin{document}

\begin{frame}
\frametitle{Parent function}
\framesubtitle{A short list}

Please check out the following parent function list.
\begin{enumerate}
\item<1-| alert@1> $y=x$
\item<2-| alert@2> $y=|x|$
\item<3-| alert@3> $y=x^{2}$
\item<4-| alert@4> $y=x^{3}$
\end{enumerate}

\end{frame}

\end{document}   
\end{lstlisting}

\begin{lstlisting}
在beamer文档类型中使用\item<>定制内容显示范围并使用\color<>{}对内容项进行高亮显示。
\documentclass{beamer}
\usefonttheme{professionalfonts}

\begin{document}

\begin{frame}
\frametitle{Parent function}
\framesubtitle{A short list}

Please check out the following parent function list.
\begin{enumerate}
\item<1-> \color<1>{red} $y=x$
\item<2-> \color<2>{red} $y=|x|$
\item<3-> \color<3>{red} $y=x^{2}$
\item<4-> \color<4>{red} $y=x^{3}$
\end{enumerate}

\end{frame}

\end{document}
\end{lstlisting}

\subsection{其他命令}
LaTeX还提供了其它命令可以实现类似的动画效果,同样可以在可选参数<>中指定内容项或内容块的显示范围,
主要包括\textbackslash onslide、\textbackslash uncover、\textbackslash only、\textbackslash visible、\textbackslash invisible等命令:

\textbackslash onslide<>\{\}:该命令可以指定内容在当前幻灯片页放映的第几步显示。使用该命令时不显示的内容将被“遮挡”,仍将占用其指定的位置(\textbackslash uncover<>\{\}或\textbackslash visible<>\{\}也能实现类似效果);

\textbackslash only<>\{\}:该命令可以指定内容在当前幻灯片页放映的第几步插入。使用该命令时,不显示的内容对应的位置将腾空出来,可以用于显示其它内容

\textbackslash invisible<>\{\}:该命令的作用效果与\textbackslash visible<>\{\}相反,用于指定内容在当前幻灯片页放映的第几步不可见。但与\textbackslash visible<>\{\}相同的是,使用\textbackslash invisible<>\{\}命令时,不可见的内容仍占据着对应的位置,不可用于显示其它内容。如下例中的代码所示,编译后得到的效果与图9.1.8一致

\begin{lstlisting}
在beamer文档类型中使用\onslide<>{}命令实现一个简单的动画效果。

\documentclass{beamer}
    \usefonttheme{professionalfonts}

    \begin{document}

    \begin{frame}
    \frametitle{Parent function}
    \framesubtitle{A short list}

    \onslide<1->{Please check out the following parent function list.}

    \onslide<2,4>{1. $y=x$}

    \onslide<1-4>{2. $y=|x|$}

    \onslide<2->{3. $y=x^{2}$}

    \onslide<3->{4. $y=x^{3}$}

    \onslide<4>{5. $y=x^{b}$}

    \end{frame}

\end{document}
\end{lstlisting}

\begin{lstlisting}
在beamer文档类型中使用\only<>{}命令实现一个简单的动画效果。
\documentclass{beamer}
    \usefonttheme{professionalfonts}
    \begin{document}
    \begin{frame}
    \frametitle{Parent function}
    \framesubtitle{A short list}

    \only<1->{Please check out the following parent function list.}

    \only<2,4>{1. $y=x$}

    \only<1-4>{2. $y=|x|$}

    \only<2->{3. $y=x^{2}$}

    \only<3->{4. $y=x^{3}$}

    \only<4>{5. $y=x^{b}$}

    \end{frame}
\end{document}    
\end{lstlisting}

\begin{lstlisting}
在beamer文档类型中使用\invisible<>{}命令实现一个简单的动画效果。
\documentclass{beamer}
    \usefonttheme{professionalfonts}

    \begin{document}

    \begin{frame}
    \frametitle{Parent function}
    \framesubtitle{A short list}

    \visible<1-4>{Please check out the following parent function list.}

    \invisible<1,3>{1. $y=x$}

    \invisible<>{2. $y=|x|$}

    \invisible<1>{3. $y=x^{2}$}

    \invisible<1-2>{4. $y=x^{3}$}

    \invisible<1-3>{5. $y=x^{b}$}

    \end{frame}

    \end{document}
\end{lstlisting}

\subsection{自动计数}
上述介绍的动画效果定制命令均通过在<>中指定具体的数字指定内容显示范围。
此时,如果想要在开始处或中间插入新的内容项,则其余所有内容项的<>显示范围都必须作出相应调整,
显然不够灵活。\LaTeX 提供了一种更巧妙的方式可以解决这类问题:使用“+”号代替具体数字,
从1开始进行自动递增计数。就例9-13而言,可以用“+”符号代替各<>中的具体数字,可以得到完全相同的编译效果。

\begin{lstlisting}
在beamer文档类型中使用+符号灵活定制幻灯片效果。
\documentclass{beamer}
\usefonttheme{professionalfonts}

\begin{document}

\begin{frame}
\frametitle{Parent function}
\framesubtitle{A short list}

Please check out the following parent function list.
\begin{enumerate}
\item<+-| alert@+> $y=x$  % 此时“+”号对应数字1
\item<+-| alert@+> $y=|x|$  % 此时“+”号对应数字2
\item<+-| alert@+> $y=x^{2}$  % 此时“+”号对应数字3
\item<+-| alert@+> $y=x^{3}$  % 此时“+”号对应数字4
\end{enumerate}

\end{frame}

\end{document}
上述语句每一条\item格式相同,因此也可以简写为如下语句:
\documentclass{beamer}
\usefonttheme{professionalfonts}

\begin{document}

\begin{frame}
\frametitle{Parent function}
\framesubtitle{A short list}

Please check out the following parent function list.
\begin{enumerate}[<+-| alert@+>]
\item $y=x$
\item $y=|x|$
\item $y=x^{2}$
\item $y=x^{3}$
\end{enumerate}

\end{frame}

\end{document}    
\end{lstlisting}

有时在<>中使用的数字不总是从1开始递增,那么就需要使用“+(偏移量)”的命令格式。
比如,如果当前“+”号对应的计数器值为3,那么<+(2)->意味着在当前计数器值的基础上加2,
<+(-2)->则意味着在当前计数器值的基础上减2。

\begin{lstlisting}
在beamer文档类型中使用+(偏移量)符号灵活定制任意显示步骤的幻灯片效果。
\documentclass{beamer}
\usefonttheme{professionalfonts}

\begin{document}

\begin{frame}
\frametitle{Parent function}
\framesubtitle{A short list}

Please check out the following parent function list.
\begin{enumerate}
\item<+(1)-> $y=x$  % 相当于`\item<2-> $y=x$`
\item<-+(2)> $y=|x|$  % 相当于`\item<-4> $y=|x|$`
\item<+(-1)-+(1)> $y=x^{2}$  % 相当于`\item<2-4> $y=x^{2}$`
\item<+(-1)> $y=x^{3}$  % 相当于`\item<3> $y=x^{3}$`
\end{enumerate}

\end{frame}

\end{document}    
\end{lstlisting}

\section{创建文本框}
在幻灯片中框选文本或图片等元素是常见的操作,可以对幻灯片内容进行划分或者突出重点内容。
在Beamer中,可以通过添加区块环境(block environments)或创建盒子(box)结构的方式将文本等元素放在各式各样的框中。

\subsection{区块环境}
Beamer提供了区块环境(block)可用于编辑文本内容,通过block环境创建的文本内容将放置在一个框中,使其与普通文本区分开。根据内容样式和使用目的的不同,包括三种区块环境:

\begin{itemize}
    \item block:一般性区块环境。使用语法为\textbackslash begin\{block\}<指定显示步骤>\{设置标题\} \textbackslash end\{block\};
    \item alertblock:警示性区块环境,主要用于创建警示信息。使用语法为\\ \textbackslash begin\{alertblock\}<指定显示步骤>\{设置标题\} \textbackslash end\{alertblock\};
    \item exampleblock:示例性区块环境,主要用于创建示例文本。使用语法为\\ \textbackslash begin\{exampleblock\}<指定显示步骤>\{设置标题\} \textbackslash end\{exampleblock\}。
\end{itemize}

在三种区块环境的开始命令中(如:\textbackslash begin\{block\}<指定显示步骤>\{设置标题\}),\\
“<>”可用于指定当前区块内容显示的步骤,实现动画效果;第二个“\{\}”可用于设置该区块内容的标题,
标题将显示在区块内容的上面。此外,区块内容的样式由使用的Beamer主题样式决定。

\begin{lstlisting}
在beamer文档类型中使用block环境插入一个一般文本框、使用alertblock环境插入一个警示性文本框、以及使用exampleblock环境插入一个示例性文本框。
\documentclass{beamer}
\usefonttheme{professionalfonts}
\usetheme{Copenhagen}

\begin{document}

\begin{frame}
\begin{block}<1>{Block1}
This is a generic block.
\end{block}

\begin{alertblock}<1>{Block2}
This is an alert block.
\end{alertblock}

\begin{exampleblock}<1>{Block3}
This is an example block.
\end{exampleblock}
\end{frame}

\end{document}
\end{lstlisting}

\subsection{定理类环境}
对于定理、引理、推论、示例等定理类文本,除了可以考虑使用区块环境创建之外,Beamer也预定义了相应的命令环境可供使用,包括:
\begin{itemize}
    \item definition:定义环境。使用语法为\textbackslash begin\{definition\}<指定显示步骤>\{设置名称\} \textbackslash end\{definition\};
    \item fact:事实环境。使用语法为\textbackslash begin\{fact\}<指定显示步骤>\{设置名称\} \textbackslash end\{fact\};
    \item theorem:定理环境。使用语法为\textbackslash begin\{theorem\}<指定显示步骤>\{设置名称\} \textbackslash end\{theorem\};
    \item lemma:引理环境。使用语法为\textbackslash begin\{lemma\}<指定显示步骤>\{设置名称\} \\ \textbackslash end\{lemma\};
    \item proof:证明环境。使用语法为\textbackslash begin\{proof\}<指定显示步骤>\{设置名称\} \\ \textbackslash end\{proof\};
    \item corollary:推论环境。使用语法为\textbackslash begin\{corollary\}<指定显示步骤>\{设置名称\} \textbackslash end\{corollary\};
    \item example:示例环境等。使用语法为\textbackslash begin\{example\}<指定显示步骤>\{设置名称\} \textbackslash end\{example\}。    
\end{itemize}

定理类环境的使用与区块环境类似:使用定理类环境可以创建文本框;开始命令中的“<>”可用于指定当前内容显示的步骤,实现动画效果;第二个“{}”可用于设置该定理类内容的名称。不同于区块环境的是:定理类内容的标题默认为对应的定理类型,如在definition环境下,标题即为“Definition”,显示在定理类内容的上方;而定理类内容的名称允许用户自行定制,通常位于定理类内容的左侧,以较大的斜体字标示。

\begin{lstlisting}
在beamer文档类型中使用definition环境插入一个定义文本框、使用theorem环境插入一个定理文本框、以及使用example环境插入一个示例文本框。
\documentclass{beamer}
\usefonttheme{professionalfonts}
\usetheme{Copenhagen}

\begin{document}

\begin{frame}{Definition, theorem and example}
\begin{definition}<1>{Definition Demo}
This is a definition.
\end{definition}

\begin{theorem}<1>{Theorem Demo}
This is a theorem.
\end{theorem}

\begin{example}<1>{Example Demo}
This is an example.
\end{example}
\end{frame}

\end{document}
\end{lstlisting}


\subsection{ Beamer中的盒子}

Beamer也支持通过绘制外框的方式为幻灯片的元素(如,文本、图片等)加上外框,或者说创建盒子(box)。
常用的语法包括调用\textbackslash fbox\{\}命令绘制简单矩形框、或调用fancybox宏包提供的命令(\textbackslash shadowbox,\textbackslash doublebox,\textbackslash ovalbox和\textbackslash Ovalbox)创建不同类型的外框。

使用\textbackslash fbox\{\}命令可以创建简单的矩形盒子,调用以下命令可以对盒子参数进行修改:
\begin{itemize}
    \item \textbackslash setlength\{\textbackslash fboxsep\}\{\}:设置盒子内的元素与其边框之间的距离,默认值为3pt;
    \item \textbackslash setlength\{\textbackslash fboxrule\}\{\}:设置盒子边框线的粗细,默认值为0.4pt。 
\end{itemize}


此外,盒子之间的行间距可以使用\textbackslash vskip命令进行修改。

\begin{lstlisting}
在beamer文档类型中使用\fbox{}命令三个文本盒子、使用\setlength命令设置不同参数、并使用\vskip命令设置行间距。
\documentclass{beamer}

\begin{document}

\begin{frame}
\setlength{\fboxsep}{3pt}
\setlength{\fboxrule}{0.4pt}
\fbox{This is our 1st text box.}
\vskip 5mm
\setlength{\fboxsep}{6pt}
\setlength{\fboxrule}{0.8pt}
\fbox{This is our 2nd text box.}
\vskip 5mm
\setlength{\fboxsep}{9pt}
\setlength{\fboxrule}{1.2pt}
\fbox{This is our 3rd text box.}
\end{frame}

\end{document}
\end{lstlisting}

对于较短的文本内容,使用\textbackslash fbox\{\}命令可以实现较好的效果。
但由于在\textbackslash fbox\{\}命令中换行符\textbackslash \textbackslash 不起作用,
因此如果要对段落文本或长文本创建盒子,需要先将文本内容放置到段落环境中,
然后再调用\textbackslash fbox\{\}命令。
其中,\textbackslash begin\{minipage\}[外部对齐方式][高度][内部对齐方式]\{宽度\}\{内容\} \textbackslash end\{minipage\}环境和\textbackslash parbox[外部对齐][高度][内部对齐]\{宽度\}\{内容\}命令是比较常用的处理段落的语法。

\begin{lstlisting}
在beamer文档类型中使用\fbox{}命令和minipage环境创建段落文本盒子。
\documentclass{beamer}

\begin{document}

\begin{frame}

\fbox{
\begin{minipage}[c][1.8cm][t]{5cm}
{This is our paragraph text box. This is our paragraph text box. This is our paragraph text box. This is our paragraph text box.}
\end{minipage}}

\end{frame}

\end{document}
\end{lstlisting}

\begin{lstlisting}
在beamer文档类型中使用\fbox{}命令和\parbox命令创建段落文本盒子。
\documentclass{beamer}

\begin{document}

\begin{frame}

\fbox{
\parbox[c][1.8cm][t]{5cm}
{This is our paragraph text box. This is our paragraph text box. This is our paragraph text box. This is our paragraph text box.}}

\end{frame}

\end{document}
\end{lstlisting}

\begin{lstlisting}
当然,除了为文本内容创建盒子之外,\fbox命令也能为图片等非文本内容创建盒子。
在beamer文档类型中使用figure环境插入三张图片,并使用\fbox{}命令将三种图片装入一个盒子中。
\documentclass{beamer}

\begin{document}

\begin{frame}

\begin{figure}
\centering
\fbox{
\includegraphics[width=0.2\linewidth]{redflower.png}
\includegraphics[width=0.2\linewidth]{yellowflower.png}
\includegraphics[width=0.2\linewidth]{blueflower.png}
}
\caption{Here is a figure box.}
\end{figure}

\end{frame}

\end{document}
\end{lstlisting}

在fancybox宏包中,提供了以下四个命令用来创建不同样式的盒子:
\begin{itemize}
    \item \textbackslash shadowbox\{\}:创建阴影盒子;
    \item \textbackslash doublebox\{\}:创建两重线盒子;
    \item \textbackslash ovalbox\{\}:创建细边线椭圆盒子;
    \item \textbackslash Ovalbox\{\}:创建粗边线椭圆盒子。
\end{itemize}

\begin{lstlisting}
在beamer文档类型中使用\shadowbox,\doublebox,\ovalbox和Ovalbox命令创建不同样式的盒子。
\documentclass{beamer}
\usepackage{fancybox}
\begin{document}

\begin{frame}

\setlength{\fboxsep}{5pt}
\setlength{\fboxrule}{2pt}

\shadowbox{This is a shadowbox.}

\vskip 5mm

\doublebox{This is a doublebox.}

\vskip 5mm

\ovalbox{This is an ovalbox.}

\vskip 5mm

\Ovalbox{This is an Ovalbox.}

\end{frame}

\end{document}
\end{lstlisting}

\begin{lstlisting}
在beamer文档类型中使用figure环境插入四张图片,使用\shadowbox,\doublebox,\ovalbox和Ovalbox命令分别为每张图片创建盒子,并使用\parbox命令把图片和标题均包含在盒子中。
\documentclass{beamer}
\usepackage{fancybox}
\begin{document}

\setlength{\fboxsep}{5pt}
\setlength{\fboxrule}{2pt}

\begin{frame}
\begin{figure}
\centering
\shadowbox{
\parbox[c][6cm][t]{5cm}{
\includegraphics[width=1\linewidth]{redflower.png}
\caption{A red flower.}}}
\end{figure}
\end{frame}

\begin{frame}
\begin{figure}
\centering
\doublebox{
\parbox[c][6cm][t]{5cm}{
\includegraphics[width=1\linewidth]{yellowflower.png}
\caption{A yellow flower.}}}
\end{figure}
\end{frame}

\begin{frame}
\begin{figure}
\centering
\ovalbox{
\parbox[c][6cm][t]{5cm}{
\includegraphics[width=1\linewidth]{blueflower.png}
\caption{A blue flower.}}}
\end{figure}
\end{frame}

\begin{frame}
\begin{figure}
\centering
\Ovalbox{
\parbox[c][6cm][t]{5cm}{
\includegraphics[width=1\linewidth]{magentaflower.png}
\caption{A magenta flower.}}}
\end{figure}
\end{frame}

\end{document}
\end{lstlisting}


\section{设置主题样式}
使用Beamer制作幻灯片的一道特色就是有现成的主题样式可供选择和直接使用,其中,主题样式对于幻灯片的演示效果而言十分重要,简言之,主题样式就是幻灯片的“外观”,改变幻灯片最简单的方式就是变换不同的主题样式。Beamer中提供的每种主题样式都具有良好的可用性和可读性,这也使得Beamer制作出来的幻灯片看起来十分专业,同时,反复使用的难度也不大。

在英文中,主题对应的英文单词为theme。狭义来看,幻灯片主题是指幻灯片的主题样式;但从广义来看,其实幻灯片主题包括了包括主题样式、颜色主题、字体主题、内部主题、外部主题。

\subsection{基本介绍}
使用Beamer制作幻灯片时,我们可以选择很多已经封装好的幻灯片主题样式,不同样式可以达到不同的视觉效果。
其实,使用这些主题样式的方法非常简单。通常来说,在前导代码中插入\textbackslash usetheme\{\}命令即可,
例如使用Copenhagen(哥本哈根主题样式)只需要在前导代码中申明\textbackslash usetheme\{Copenhagen\},
这种方式调用主题样式是非常省事。

在Beamer文档类型中,有几十种主题样式可供选择和使用,比较常用的主题样式包括以下这些:
\begin{itemize}
    \item Berlin:柏林主题样式,默认样式为蓝色调。
    \item Copenhagen: 哥本哈根主题样式,默认样式为蓝色调。
    \item CambridgeUS:美国剑桥主题样式,默认样式为红色调。
    \item Berkeley:伯克利主题样式,默认样式为蓝色调。
    \item Singapore:新加坡主题样式。
    \item Warsaw:默认样式为蓝色调。
\end{itemize}

\begin{lstlisting}
在beamer文档类型中使用CambridgeUS主题样式制作一个简单的幻灯片。

\documentclass{beamer}
\usetheme{CambridgeUS}

\begin{document}

\begin{frame}{Example}

This is a simple example for the CambridgeUS theme.

\end{frame}

\end{document}
\end{lstlisting}

当然,在这些主题样式基础上,我们也能够使用一些特定的主题样式如颜色主题、字体主题、内部主题、外部主题对幻灯片的显示效果进行调整。

\subsection{颜色主题}
使用Beamer制作幻灯片时,我们能够自行设置幻灯片主题样式的色调,使用\\ \textbackslash usecolortheme\{\}命令即可,
这些色调包括beetle、beaver、orchid、whale、dolphin等。这里的色调又被称为颜色主题,它定义了幻灯片各部分的颜色搭配,
设置特定的颜色主题后,我们能够得到不同的组合样式,具体可参考\href{https://hartwork.org/beamer-theme-matrix/}{网站提供的组合样式矩阵} 

\begin{lstlisting}
在beamer文档类型中使用CambridgeUS主题样式和dolphin色调制作一个简单的幻灯片。
\documentclass{beamer}
\usetheme{CambridgeUS}
\usecolortheme{dolphin}
\begin{document}
\begin{frame}{Example}

This is a simple example for the CambridgeUS theme with dolphin (color theme).

\end{frame}
\end{document}
\end{lstlisting}

\subsection{字体主题}
实际上,对于幻灯片的文本字体,我们可以调用字体样式对其进行调整。
在Beamer中,字体样式被称为字体主题,它定义了幻灯片的字体搭配。
具体使用方法是:在前导代码中要用到的命令为\textbackslash usefonttheme\{A\},位置A填写的一般是字体类型,例如serif。

我们知道:在常规文档中,可以使用各种字体对应的宏包达到调用字体的作用,使用规则为\textbackslash usepackage\{A\},位置A填写的一般是字体类型,
包括serif、avant、bookman、chancery、charter、euler、helvet、mathtime、mathptm、mathptmx、newcent、palatino、pifont、utopia等。

\begin{lstlisting}
使用beamer文档类型创建一个简单的幻灯片,并在前导代码中申明使用serif对应的字体样式。
\documentclass{beamer}
\usefonttheme{serif}

\begin{document}

\begin{frame}

This is a simple example for using \alert{serif} font theme.

\end{frame}

\end{document}
\end{lstlisting}

\begin{lstlisting}
使用beamer文档类型创建一个简单的幻灯片,并在前导代码中申明使用字体palatino对应的宏包。
\documentclass{beamer}
\usepackage{palatino}

\begin{document}

\begin{frame}

This is a simple example for using \alert{palatino} font.

\end{frame}

\end{document}
\end{lstlisting}


\subsection{内部主题}
内部主题定义了幻灯片展示区域的样式,如列表、定理等,内部主题不包括页眉、页脚、导航栏等部分。
每一种主题样式都有默认的内部主题,更换内部主题需使用\textbackslash useinnertheme\{A\}命令,
位置A可供选择的内部主题包括circles、rectangles、rounded和inmargin。

\begin{lstlisting}
在beamer文档类型中分别使用circles和inmargin两种内部主题制作幻灯片。
使用circles内部主题:
\documentclass{beamer}
\usetheme{CambridgeUS}
\usefonttheme{professionalfonts}
\useinnertheme{circles}

\begin{document}

\begin{frame}
\frametitle{Parent function}
\framesubtitle{A short list}

Please check out the following parent function list.
\begin{enumerate}
\item $y=x$
\item $y=|x|$
\item $y=x^{2}$
\item $y=x^{3}$
\item $y=x^{b}$
\end{enumerate}

\end{frame}

\end{document}
\end{lstlisting}

\begin{lstlisting}
使用inmargin内部主题:
\documentclass{beamer}
\usetheme{CambridgeUS}
\usefonttheme{professionalfonts}
\useinnertheme{inmargin}

\begin{document}

\begin{frame}
\frametitle{Parent function}
\framesubtitle{A short list}

Please check out the following parent function list.
\begin{enumerate}
\item $y=x$
\item $y=|x|$
\item $y=x^{2}$
\item $y=x^{3}$
\item $y=x^{b}$
\end{enumerate}

\end{frame}

\end{document}
\end{lstlisting}

\subsection{外部主题}
外部主题定义了幻灯片的边框、页眉、页脚、导航栏等部分的样式。更换外部主题需使用\textbackslash useoutertheme\{A\},
位置A可供选择的外部主题包括infolines、smoothbars、sidebar、split和tree。

\subsection{表格字体大小}
在Beamer中制作表格,当我们想对表头或者表格内容文字大小进行调整时,
可以使用在前导代码中申明使用caption宏包,即\textbackslash usepackage\{caption\},
然后设置具体的字体大小即可,如\textbackslash captionsetup\{font = scriptsize, labelfont = scriptsize\}可以将表头和表格内容字体大小调整为scriptsize。

\begin{lstlisting}
使用\begin{table} \end{table}环境创建一个简单表格,并使用caption宏包将表头字体大小设置为Large、将表格内容字体大小设置为large。
其中,单就设置表头字体大小而言,除了使用caption宏包之外,还可以通过对幻灯片设置全局参数达到调整字体大小的效果,例如\setbeamerfont{caption}{size = \Large}。
\documentclass{beamer}
\usepackage{booktabs}
\usepackage{caption}
\captionsetup{font = large, labelfont = Large}

\begin{document}

\begin{frame}

\begin{table}
\caption{A simple table.}
\begin{tabular}{l|ccc}
\toprule
& \textbf{header3} & \textbf{header4} & \textbf{header5} \\
\midrule
\textbf{header1} & cell1 & cell2 & cell3 \\
\midrule
\textbf{header2} & cell4 & cell5 & cell6 \\
\bottomrule
\end{tabular}
\end{table}

\end{frame}

\end{document}
\end{lstlisting}
\vspace{-1cm}
\subsection{样式调整}
在Beamer文档类型中,除了可以使用各种主题样式,另外也可以根据幻灯片组成部分,分别对侧边栏、导航栏以及Logo等进行调整。其中,侧边栏是由所选幻灯片主题样式自动生成的,主要用于显示幻灯片目录。有时为了显示幻灯片的层次,使用侧边栏进行目录索引。

\begin{lstlisting}
使用Berkeley主题样式,并将侧边栏显示在右侧。
\documentclass{beamer}
\PassOptionsToPackage{right}{beamerouterthemesidebar}
\usetheme{Berkeley}
\usefonttheme{professionalfonts}
\begin{document}
\begin{frame}
\frametitle{Parent function}
\framesubtitle{A short list}
Please check out the following parent function list.
\begin{enumerate}
\item $y=x$
\item $y=|x|$
\item $y=x^{2}$
\item $y=x^{3}$
\item $y=x^{b}$
\end{enumerate}
\end{frame}
\end{document}
\end{lstlisting}

\begin{lstlisting}
很多时候我们会发现,在各类学术汇报中,幻灯片的首页通常会有主讲人所在的研究机构Logo。在Beamer文档类型中,有\logo和\titlegraphic两个命令可供使用,使用\logo命令添加的Logo会在每一页幻灯片中都显示,而使用\titlegraphic命令添加的Logo只出现在标题页。
使用\logo命令在幻灯片中添加Logo。
\documentclass{beamer}
\usefonttheme{professionalfonts}

\title{A Simple Beamer Example}
\author{Author's Name}
\institute{Author's Institute}

\logo{\includegraphics[width=2cm]{logopolito}}

\begin{document}·
\begin{frame}
\titlepage
\end{frame}
\begin{frame}{Parent function}{A short list}
Please check out the following parent function list.
\begin{enumerate}
\item $y=x$
\item $y=|x|$
\item $y=x^{2}$
\item $y=x^{3}$
\item $y=x^{b}$
\end{enumerate}
\end{frame}
\end{document}
\end{lstlisting}

\begin{lstlisting}
使用\titlegraphic命令在幻灯片的标题页添加Logo。
\documentclass{beamer}
\usefonttheme{professionalfonts}

\title{A Simple Beamer Example}
\author{Author's Name}
\institute{Author's Institute}

\titlegraphic{\includegraphics[width=2cm]{logopolito}\hspace*{4.75cm}~
   \includegraphics[width=2cm]{logopolito}
}

\begin{document}

\begin{frame}
\titlepage
\end{frame}

\begin{frame}{Parent function}{A short list}
Please check out the following parent function list.
\begin{enumerate}
\item $y=x$
\item $y=|x|$
\item $y=x^{2}$
\item $y=x^{3}$
\item $y=x^{b}$
\end{enumerate}
\end{frame}

\end{document}
\end{lstlisting}


\section{插入程序源代码}
使用Beamer制作幻灯片时,可以使用verbatim宏包中的\textbackslash begin\{verbatim\} \\ \textbackslash end\{verbatim\}环境插入程序源代码,
相应地,\textbackslash begin\{frame\}[fragile] \textbackslash end\{frame\}环境中需要添加fragile选项,否则会导致编译报错。

\begin{lstlisting}
在beamer文档类型中使用\begin{frame}[fragile] \end{frame}和\begin{verbatim} \end{verbatim}环境插入几行简单的Python程序。
\documentclass{beamer}
\usefonttheme{professionalfonts}
\usepackage{verbatim}

\begin{document}

\begin{frame}[fragile]
\frametitle{Parent function}
\framesubtitle{A short list}

Please check out the following parent function list.
\begin{enumerate}
\item $y=x$
\item $y=|x|$
\item $y=x^{2}$
\item $y=x^{3}$
\item $y=x^{b}$
\end{enumerate}

\textbf{Python code:}

\begin{verbatim}
import numpy as np

b = 5
y = np.zeros(100)
for x in range(1, 101):
    y[x] = x ** b
\end{verbatim}

\end{frame}

\end{document}
\end{lstlisting}

\begin{lstlisting}
除了verbatim宏包,还可以使用listings宏包中的\begin{lstlistings} \end{lstlistings}插入程序源代码。
在beamer文档类型中使用\begin{frame}[fragile] \end{frame}和\begin{lstlistings} \end{lstlistings}环境插入几行简单的Python程序。
\documentclass{beamer}
\usefonttheme{professionalfonts}
\usepackage{listings}

\begin{document}

\begin{frame}[fragile]
\frametitle{Parent function}
\framesubtitle{A short list}

Please check out the following parent function list.
\begin{enumerate}
\item $y=x$
\item $y=|x|$
\item $y=x^{2}$
\item $y=x^{3}$
\item $y=x^{b}$
\end{enumerate}

\textbf{Python code:}

\begin{lstlisting}
import numpy as np

b = 5
y = np.zeros(100)
for x in range(1, 101):
    y[x] = x ** b
\ end{lstlisting}
\end{frame}
\end{document}
\end{lstlisting}

\section{添加参考文献}
通常来说,学术汇报的幻灯片时常需要添加与本研究相关的参考文献作为支撑材料。
使用LaTeX制作常规文档时,最常用的文献管理工具是Bibtex,但事实上,Beamer并不支持编译Bibtex。
因此,拓展名为.bib的文献管理文件在Beamer中是无法使用的,
不过我们可以使用\textbackslash begin\{thebibliography\} \textbackslash end\{thebibliography\}环境添加参考文献。

在常规文档中,使用\LaTeX 创建参考文献的简单方式是使用\textbackslash begin\{thebibliography\} \textbackslash end\{thebibliography\}环境添加参考文献。
有了参考文献的条目和标签,在正文中使用\textbackslash cite\{\}命令进行引用便可让参考文献显示出来,这种做法在Beamer中也是适用的。只不过在添加参考文献时,
我们需要对文献的类型进行指定,著作对应着\\ \textbackslash beamertemplatebookbibitems命令,而论文则对应着\textbackslash beamertemplatearticlebibitems。
需要注意的是,为了避免文献数量过多而导致的参考文献页面排版问题,可以在\\ \textbackslash begin\{frame\}[allowframebreaks] \textbackslash end\{frame\}环境中申明自动跨页。

\begin{lstlisting}
使用beamer文档类型创建幻灯片并在\begin{thebibliography} \end{thebibliography}环境中创建参考文献。
\documentclass{beamer}
\usetheme{CambridgeUS}

\begin{document}

\begin{frame}
\frametitle{Reference Example}

If you have any interest in matrix computations, please referring to \cite{golub2013matrix, petersen2008the}.

\end{frame}

\begin{frame}
\frametitle<presentation>{Further Reading}

\begin{thebibliography}{10}

  \beamertemplatebookbibitems
  \bibitem{golub2013matrix}
    Gene H. Golub and Charles F. Van Loan.
    \newblock {\em Matrix computations}.
    \newblock JHU press, 2013.

  \beamertemplatearticlebibitems
  \bibitem{petersen2008the}
    Kaare Brandt Petersen, Michael Syskind Pedersen.
    \newblock The matrix cookbook.
    \newblock {\em Technical University of Denmark}, 510, 2008.

\end{thebibliography}
\end{frame}
\end{document}
\end{lstlisting}

\begin{lstlisting}
使用beamer文档类型创建幻灯片并在\begin{thebibliography} \end{thebibliography}环境中创建参考文献,其中,需要对幻灯片申明自动跨页。

\documentclass{beamer}
\usetheme{CambridgeUS}

\begin{document}

\begin{frame}
\frametitle{Reference Example}

If you have any interest in matrix computations, please referring to \cite{golub2013matrix, petersen2008the}.

\end{frame}

\begin{frame}[allowframebreaks]
\frametitle<presentation>{Further Reading}

\begin{thebibliography}{10}

    \beamertemplatebookbibitems
    \bibitem{golub2013matrix}
    Gene H. Golub and Charles F. Van Loan.
    \newblock {\em Matrix computations}.
    \newblock JHU press, 2013.

    \beamertemplatearticlebibitems
    \bibitem{petersen2008the}
    Kaare Brandt Petersen, Michael Syskind Pedersen.
    \newblock The matrix cookbook.
    \newblock {\em Technical University of Denmark}, 510, 2008.

    \beamertemplatebookbibitems
    \bibitem{golub2013matrix}
    Gene H. Golub and Charles F. Van Loan.
    \newblock {\em Matrix computations}.
    \newblock JHU press, 2013.

    \beamertemplatearticlebibitems
    \bibitem{petersen2008the}
    Kaare Brandt Petersen, Michael Syskind Pedersen.
    \newblock The matrix cookbook.
    \newblock {\em Technical University of Denmark}, 510, 2008.

    \beamertemplatebookbibitems
    \bibitem{golub2013matrix}
    Gene H. Golub and Charles F. Van Loan.
    \newblock {\em Matrix computations}.
    \newblock JHU press, 2013.

    \beamertemplatearticlebibitems
    \bibitem{petersen2008the}
    Kaare Brandt Petersen, Michael Syskind Pedersen.
    \newblock The matrix cookbook.
    \newblock {\em Technical University of Denmark}, 510, 2008.

\end{thebibliography}

\end{frame}

\end{document}
\end{lstlisting}


\section{插入表格}

在Beamer幻灯片中插入表格的规则与常规文档一致,可以使用tabular环境创建表格内容,
或者在tabular环境外额外嵌套一层table环境以创建标题、索引标签等元素,
当然对于合并单元格、创建三线表、创建彩色表格等操作也是类似的。

\begin{lstlisting}
在beamer文档类型中使用table环境和tabular环境制作一个简单表格。
\documentclass{beamer}

\begin{document}

\begin{frame}

The created table\ref{tab:table_a} is shown as follows.
\begin{table}[bt]
\caption{A table in Beamer}
\label{tab:table_a}
    \begin{tabular}{|l|c|c|} \hline
    \textbf{Code Structure} & \textbf{Component} & \textbf{Others} \\
    \hline
    preamble & figures & title \\
    body & tables & footline \\
    & equations & list \\
    & normal texts & block \\
    \hline
    \end{tabular}
\end{table}
\end{frame}  
\end{document}   
\end{lstlisting}

在Beamer中,我们可以使用前面介绍的动画效果命令(如\textbackslash uncover\{\})为表格添加动画,让表格中的各单元格内容分步呈现。

\begin{lstlisting}
在beamer文档类型中使用\begin{table} \end{table}环境制作一个简单表格,同时使用\uncover{}设置动画格式。
\documentclass{beamer}
\begin{document}
\begin{frame}
\begin{table}[bt]
\begin{tabular}{|l|c|c|} \hline
\textbf{Code Structure} & \textbf{Component} & \textbf{Others} \\
\hline
\uncover<1->{preamble} & \uncover<2->{figures} & \uncover<3->{title} \\
\uncover<1->{body} & \uncover<2->{tables} & \uncover<3->{footline} \\
& \uncover<2->{equations} & \uncover<3->{list} \\
& \uncover<2->{normal texts} & \uncover<3->{block} \\
\hline
\end{tabular}
\end{table}
\end{frame}
\end{document}
\end{lstlisting}

\section{插入与调整图片}
在制作幻灯片时经常需要插入图片以辅助演讲汇报,对此,Beamer提供了相关宏包及命令可以支持在幻灯片中插入图片并对其进行一系列操作,包括编辑图片、插入子图、调整排列方式、调整布局方式、设置背景图片、在标题页中插入图片、添加动画效果等。

\subsection{插入图片}
类似于article文档中插入图片操作,在Beamer中可以基于\textbackslash begin\{figure\} \\ \textbackslash end\{figure\}环境、并使用\textbackslash includegraphics[图片参数]\{图片名或图片路径\}命令为幻灯片插入图片、以及设置height、width、angle等参数。
不同之处主要在于:
\begin{itemize}
    \item Beamer文档类自带graphicx工具包,可以省略声明语句\textbackslash usepackage\{graphicx\};
    \item 在Beamer中浮动图形位置参数h、t、b、p无效,因此需要使用其它方式调整图片位置;
    \item 在Beamer中使用\textbackslash caption\{\}命令仅创建图片标题,而不再为图片进行自动编号。为此,需要在导言区额外添加声明语句\textbackslash setbeamertemplate\{caption\}[numbered],表示对演示稿中的图片进行自动编号。
\end{itemize}

\begin{lstlisting}
在beamer文档类型中使用\begin{figure} \end{figure}环境以及\includegraphics命令插入图片,并添加声明\setbeamertemplate{caption}[numbered]使得图片进行自动编号。
\documentclass{beamer}
% 使图片进行自动编号
\setbeamertemplate{caption}[numbered]

\begin{document}

\begin{frame}{Two pictures}
\begin{figure}
\centering
    % 插入第一张图片并设置标题、索引标签
    \includegraphics[width=0.2\linewidth]{blueflower.png}
    \caption{First figure.}
    \label{fig:figlabel1}
    % 插入第二张图片并设置标题、索引标签
    \includegraphics[width=0.2\linewidth]{yellowFlower.png}
    \caption{Second figure.}
    \label{fig:figlabel2}
\end{figure}
\end{frame}

\end{document}    
\end{lstlisting}

\subsection{编辑图片}
读者可以通过修改Beamer主题选项从而为演示稿定制图片标题样式,包括:
\begin{itemize}
    \item \textbackslash setbeamerfont\{caption\}\{size=定制的字体大小\}:设置图片标题的字体大小,如设置为\textbackslash small、\textbackslash large、\textbackslash Large等;
    \item \textbackslash setbeamercolor\{caption\}\{fg=定制的颜色\}:设置图片标题编号的颜色,如设置为red、blue等;
    \item \textbackslash setbeamercolor\{caption name\}\{fg=定制的颜色\}:设置图片标题内容的颜色,如设置为red、blue等。
\end{itemize}

读者还可以改变图片的不透明度。为此,首先需要调用tikz宏包,使用语句\\ \textbackslash usepackage\{tikz\}即可;
然后将插入的图片作为一个节点\textbackslash node放置到tikzpicture环境内,并使用\textbackslash node[opacity=定制的不透明度]语句设置不透明度选项参数,
如下所示:

\begin{lstlisting}
\begin{tikzpicture}
    \node[opacity=定制的不透明度]{\includegraphics[width=0.3\textwidth]{图片路径}};   
\end{tikzpicture}   
\end{lstlisting}

\begin{lstlisting}
在beamer文档类型中使用\setbeamerfont{caption}、\setbeamercolor{caption}和\setbeamercolor{caption name}命令修改图片标题样式;并调用tikz宏包,使用tikzpicture环境和\node命令设置图片不透明度。
\documentclass{beamer}
\usepackage{tikz}
\setbeamertemplate{caption}[numbered]
% 定制图片标题样式
\setbeamerfont{caption}{size=\large}
\setbeamercolor{caption name}{fg=red}
\setbeamercolor{caption}{fg=blue}

\begin{document}

\begin{frame}{Two pictures}

\begin{figure}
\centering
    % 插入第一张图片并设置标题、创建索引标签
    \begin{tikzpicture}
    \node[opacity=0.3]
    {\includegraphics[width=0.2\linewidth]{blueflower.png}};   
    \end{tikzpicture}
    \caption{First figure.}
    \label{fig:figlabel1}
    % 插入第二张图片并设置图片不透明度、设置标题、创建索引标签
    \begin{tikzpicture}
    \node[opacity=0.3]
    {\includegraphics[width=0.2\linewidth]{yellowflower.png}};   
    \end{tikzpicture}
    \caption{Second figure.}
    \label{fig:figlabel2}
\end{figure}

\end{frame}

\end{document}
\end{lstlisting}

\subsection{插入子图}
有时需要在Beamer中插入子图,用到的宏包与命令与article文档情况类似。读者可以只需要使用语句\textbackslash usepackage\{subcaption\}调用subcaption宏包,并在figure环境中创建多个subfigure环境,每个subfigure环境内都可以进行插入子图、设置子图标题和标签等操作。下面来看一个具体的例子:

\begin{lstlisting}
在beamer文档类型中,调用subcaption宏包,并使用figure环境中的subfigure环境创建子图。
\documentclass{beamer}
% 调用关键宏包subcaption
\usepackage{subcaption}
\setbeamertemplate{caption}[numbered]

\begin{document}

\begin{frame}{Two sub-figures}
\begin{figure}
\centering
    % 插入第一张子图
    \begin{subfigure}{0.4\linewidth}
        \centering
        \includegraphics[width=0.5\linewidth]{blueflower.png}
        \caption{First subfigure.}
        \label{fig:subfiglabel1}
    \end{subfigure}
    % 插入第二张子图
    \begin{subfigure}{0.4\linewidth}
        \centering
        \includegraphics[width=0.5\linewidth]{yellowFlower.png}
        \caption{Second subfigure.}
        \label{fig:subfiglabel2}
    \end{subfigure}
\caption{A figure contains two sub-figures.}
\label{fig:figlabel}
\end{figure}
\end{frame}
\end{document}    
\end{lstlisting}

\subsection{调整排列方式}
使用\textbackslash centering命令可以将图片放置到整个幻灯片页面的中央,但有时需要对图片的排列位置进行更多样化的调整。

\subsubsection{横向分布与纵向分布}
使用\textbackslash hfill、\textbackslash vfill命令可以分别控制图片的水平间距和垂直间距。
在同一行中,每个\textbackslash hfill产生的间隔是相同的,从而达到将多个图片横向等间距分布的效果;类似地,\textbackslash vfill命令可以实现纵向等间距分布的效果。

\begin{lstlisting}
在beamer文档类型中,使用\hfill命令实现图片横向分布,以及使用\vfill命令实现图片纵向分布。
\documentclass{beamer}

\begin{document}
% 实现图片横向分布
\begin{frame}{Three pictures with the same lateral distance}
\begin{figure}
\includegraphics[width=0.2\linewidth]{redflower.png}
\hfill
\includegraphics[width=0.2\linewidth]{blueflower.png}   
\hfill
\includegraphics[width=0.2\linewidth]{yellowflower.png}
\end{figure}
\end{frame}
\begin{frame}{Three pictures with the same longitudinal distance}
% 实现图片纵向分布
\begin{figure}
\includegraphics[width=0.2\linewidth]{redflower.png}
\vfill
\includegraphics[width=0.2\linewidth]{blueflower.png}   
\vfill
\includegraphics[width=0.2\linewidth]{yellowflower.png}
\end{figure}
\end{frame}

\end{document}
\end{lstlisting}

\subsubsection{并排显示}
在幻灯片中,将不同内容(图片、文本、表格、公式等)并排分栏显示是一种常用的展示方式。
前面我们介绍了使用\textbackslash hfill命令实现多个图片并排显示的方式,这里我们接着介绍一种更通用的方式,
不仅可以实现图片的并排显示,也可以实现文本等其它内容的并排显示。
为此,需要使用\textbackslash begin\{columns\} \textbackslash end\{columns\}环境,
并嵌套多个\textbackslash begin\{column\} \textbackslash end\{column\}环境对内容进行分栏,每个column环境中的内容都独立成栏,不同栏的内容并排显示。

\begin{lstlisting}
在beamer文档类型中,使用columns环境创建多列内容实现并排显示效果。
\documentclass{beamer}

\begin{document}
\begin{frame}{Frame1}

\begin{columns}
% 创建第一列
\begin{column}{0.4\linewidth}
    Here is a description for the blue flower on the right. Here is a description for the blue flower on the right.
\end{column}
% 创建第二列   
\begin{column}{0.4\linewidth}
    \begin{figure}
    \centering
        \includegraphics[width=0.7\linewidth]{blueflower.png}
        \caption{A blue flower.}
    \end{figure}
\end{column}
\end{columns}

\end{frame}

\end{document}
\end{lstlisting}
\subsubsection{设置任意图片位置}
通过调用tikz宏包及其命令可以为图片指定任意摆放位置。首先使用\\ \textbackslash usepackage\{tikz\}语句调用tikz宏包,然后调用如下所示的语句:

\begin{lstlisting}
\begin{tikzpicture}[remember picture, overlay]
    \node[锚点位置偏移量] at (current page.锚点位置) 
    {
        \includegraphics{图片路径}
    };
\end{tikzpicture}
\end{lstlisting}

上述示例语句中,包括:
\begin{itemize}
    \item 在幻灯片中创建tikzpicture环境,使用\textbackslash node命令创建一个内容为插入图片的节点;
    \item 设置\textbackslash begin\{tikzpicture\}[]的选项为remember picture, overlay,从而允许自由放置图片; 
    \item 通过使用\textbackslash node[] at ()语句指定图片摆放位置:
    \begin{itemize}
        \item 1)()中用于设置图片的锚点位置或摆放位置,设置格式为current page.锚点位置。
        在锚点位置设置上:一种方式是使用描述性语言,如current page.center表示将图片放在页面正中央、
        current page.east表示将图片放在页面边缘的右方(正东方向)、current page.north east表示
        将图片放在页面边缘的右上方(东北方向),以此类推;另一种方式是使用相对于页面正东方向的逆时针
        角度,如current page.0表示将图片放在页面边缘右方、current page.45表示将图片放在页面边缘右
        上方、current page.-90表示将图片放在页面边缘下方;
        \item 2)[]中用于设置锚点位置的偏移量。一种方式是使用描述性语言,包括left、right、above、
        below,分别表示将图片相对于锚点位置往左、右、上、下偏移,如\textbackslash node[left=1cm,above=2cm] 
        at (current page.center)表示将图片位置设置为页面正中央往左偏移1cm、往上偏移2cm的位置;另一种方
        式是使用偏移坐标参数xshift和yshift,如\textbackslash node[xshift=-1cm,yshift=2cm] at 
        (current page.center)的效果等价于\textbackslash node[left=1cm,above=2cm] at (current page.center)。
    \end{itemize}
\end{itemize}

\begin{lstlisting}
在beamer文档类型中,使用tikzpicture环境创建以插入的图片作为内容的节点,为\begin{tikzpicture}设置选项remember picture, overlay,并使用\node[] at ()语句设置图片的摆放位置。
\documentclass{beamer}
% 调用tikz宏包
\usepackage{tikz}

\begin{document}
% 第一页幻灯片
\begin{frame}{Frame1}
% 创建tikzpicture环境,并使用描述性语言设置锚点位置
\begin{tikzpicture}[remember picture, overlay]
    % 将图片放置到幻灯片页中央
    \node at (current page.center) 
    {
        \includegraphics[width=0.2\linewidth]{blueflower.png}
    };
    % 将图片放置到幻灯片页的右边(正东方向)
    \node at (current page.east) 
    {
        \includegraphics[width=0.2\linewidth]{blueflower.png}
    };
    % 将图片放置到幻灯片页的下方(正南方向)
    \node at (current page.south) 
    {
        \includegraphics[width=0.2\linewidth]{blueflower.png}
    };
    % 将图片放置到幻灯片页的左边(正西方向)
    \node at (current page.west) 
    {
        \includegraphics[width=0.2\linewidth]{blueflower.png}
    };
    % 将图片放置到幻灯片页的上方(正北方向)
    \node at (current page.north) 
    {
        \includegraphics[width=0.2\linewidth]{blueflower.png}
    };
\end{tikzpicture}
\end{frame}
% 第二页幻灯片
\begin{frame}{Frame2}
% 创建tikzpicture环境,并使用角度设置锚点位置、使用描述性语言设置锚点偏移量
\begin{tikzpicture}[remember picture, overlay]
    % 将图片放置到幻灯片页的右边,并往左偏移0.5cm
    \node[left=0.5cm] at (current page.0) 
    {
        \includegraphics[width=0.2\linewidth]{blueflower.png}
    };
    % 将图片放置到幻灯片页的下方,并往上偏移0.5cm
    \node[above=0.5cm] at (current page.-90) 
    {
        \includegraphics[width=0.2\linewidth]{blueflower.png}
    };
    % 将图片放置到幻灯片页的左边,并往右偏移0.5cm
    \node[right=0.5cm] at (current page.180) 
    {
        \includegraphics[width=0.2\linewidth]{blueflower.png}
    };
    % 将图片放置到幻灯片页的上方,并往下偏移0.5cm
    \node[below=0.5cm] at (current page.90) 
    {
        \includegraphics[width=0.2\linewidth]{blueflower.png}
    };
\end{tikzpicture}
\end{frame}

\end{document}
\end{lstlisting}


\subsection{调整布局方式}
在幻灯片中,常见的图文布局方式有两种:文字环绕于图片、文字浮于图片上方。下面分别展开介绍。
\subsubsection{文字环绕图片}
为了实现文字环绕图片的显示效果,首先需要使用\textbackslash usepackage\{wrapfig\}语句调用wrapfig宏包,
并创建\textbackslash begin\{wrapfigure\} \textbackslash end\{wrapfigure\}环境插入图片,
在wrapfigure环境后面输入环绕文本内容,由此创建的文本内容将环绕于插入的图片周围。其中,wrapfigure
环境与我们熟悉的figure环境类似,同样可以使用\textbackslash includegraphics、\textbackslash caption等语句
实现相关功能。在\textbackslash begin\{wrapfigure\}\{\}语句的\{\}中可以设置图片位置参数,
参数选项为r或l,分别表示将图片设置为页面右侧或左侧。

\begin{lstlisting}
在beamer文档类型中,使用wrapfigure环境插入图片并设置图片位置为页面右侧,实现文字环绕图片的效果。
\documentclass{beamer}
% 调用wrapfig宏包
\usepackage{wrapfig}

\begin{document}
\begin{frame}{Frame1}
% 创建wrapfigure环境并插入图片
\begin{wrapfigure}{r}{0.5\linewidth}
    \centering
    \includegraphics[width=0.5\linewidth]{blueflower.png}
    \caption{A blue flower.}
\end{wrapfigure}
% 示意性环绕文本
Here is a description for the picture. Here is a description for the picture. Here is a description for the picture. Here is a description for the picture. Here is a description for the picture. Here is a description for the picture. Here is a description for the picture. Here is a description for the picture. Here is a description for the picture. Here is a description for the picture. Here is a description for the picture. Here is a description for the picture.
\end{frame}

\end{document}
\end{lstlisting}

\subsubsection{文字浮于图片上方}
图片浮于文字上方是另一种常用的图文布局方式。为此,读者需要使用 \\\textbackslash usepackage\{tikz\}语句调用tikz宏包,
在tikzpicture环境中使用\textbackslash node命令创建一个内容为插入图片的节点、一个内容为文本的节点,并将文本节点的位置设为图片节点的中央。

\begin{lstlisting}
在beamer文档类型中,使用tikzpicture环境创建一个内容为插入图片的节点、一个内容为文本的节点,并设置第一个节点的名称和图片透明度、第二个节点的文本样式和位置。
\documentclass{beamer}
% 调用tikz宏包
\usepackage{tikz}

\begin{document}
\begin{frame}{Frame1}

\begin{tikzpicture}
% 创建一个内容为图片的节点,节点名为myfirstfigure
\node(myfirstfigure)[opacity=0.4]
{
    \includegraphics[width=0.3\linewidth]{greyflower.png}
    \includegraphics[width=0.3\linewidth]{greyflower.png}
    \includegraphics[width=0.3\linewidth]{greyflower.png}
};
% 创建一个内容为文本的节点
\node
[ 
    text=teal,
    font={\huge\bfseries}
] at (myfirstfigure.center) {Text over the picture!};
\end{tikzpicture}

\end{frame}

\end{document}
\end{lstlisting}

\subsection{设置背景图片}
在Beamer中可以很方便地为演示稿设置全局或局部的背景图片,
通过使用\\ \textbackslash setbeamertemplate\{background canvas\}\{插入背景图片\}语句设置背景画布选项即可。
\newpage
\begin{lstlisting}
在beamer文档类型中,\setbeamertemplate{background canvas}{插入背景图片}语句设置全局及局部背景画布选项。
\documentclass{beamer}
\usetheme{AnnArbor}

% 设置全局背景画布选项
\setbeamertemplate{background canvas}
{
    \includegraphics[width=\paperwidth,height=\paperheight]{lines.png}%
}

\begin{document}
% 第一页幻灯片使用全局背景画布
\begin{frame}{Frame1}
\end{frame}
{
% 设置局部特定背景画布选项
\setbeamertemplate{background canvas}
{
    \includegraphics[width=\paperwidth,height=\paperheight]{AI.png}%
}
% 第二页幻灯片使用局部特定背景画布
\begin{frame}{Frame2}
\end{frame}
}
% 第三页幻灯片使用全局背景画布
\begin{frame}{Frame3}
\end{frame}
% 第四页幻灯片使用全局背景画布
\begin{frame}{Frame4}
\end{frame}
\end{document}
\end{lstlisting}

如上例所示,为了让插入的图片充满整个页面,应使用width=\textbackslash paperwidth和height=\textbackslash paperheight选项;
修改局部特定背景画布时,应用\{\}符号将\\\textbackslash setbeamertemplate语句及其作用的局部frame环境放在一起。

\subsection{在标题页中插入图片}
在前面的章节中我们介绍了在Beamer中使用\textbackslash titlepage命令创建标题页的语句。
事实上,除了可以在标题页中添加标题名称、作者、机构单位等信息之外,在标题页中添加图片或图标也是一种常见的操作。
为此,只需要在标题页信息中添加\textbackslash titlegraphic\{插入图片\}命令即可,如下例所示:

\begin{lstlisting}
在beamer文档类型中,使用\titlegraphic{插入图片}命令为标题页插入图片,并使用\titlepage命令创建标题页。
\documentclass{beamer}
\usetheme{AnnArbor}

\begin{document}

\title{A presentation about AI}
\author{The author's name}
\institute{The institute's name}
\date{2021/1/1}
% 在标题页中添加图片
\titlegraphic
{
    \includegraphics[width=2cm]{robot.png}
}
\begin{frame}
    \titlepage
\end{frame}

\end{document}
\end{lstlisting}

\begin{lstlisting}
在beamer文档类型中,使用\titlegraphic{插入图片}命令为标题页插入图片,并在tikzpicture环境中使用\node命令调整图片位置为右下方。
\documentclass{beamer}
% 调用tikz宏包
\usepackage{tikz}
\usetheme{AnnArbor}

\begin{document}

\title{A presentation about AI}
\author{The author's name}
\institute{The institute's name}
\date{2021-8-30}
% 在标题页中添加图片并指定位置
\titlegraphic
{
    \begin{tikzpicture}[overlay, remember picture]
        \node[above=0.8cm] at (current page.-43){
        \includegraphics[width=1.5cm]{robot.png}
        };
    \end{tikzpicture}
}
\begin{frame}
    \titlepage
\end{frame}

\end{document}
\end{lstlisting}

\subsection{添加动画效果}
如果想要为图片添加动画效果,使得不同的图片分步显示,使用设置了显示范围选项的\textbackslash includegraphics<显示范围>\{图片路径\}命令插入图片即可。

\begin{lstlisting}
在beamer文档类型中使用\includegraphics<>{}命令插入多张图片并分步显示。
\documentclass{beamer}

\begin{document}

\begin{frame}

\includegraphics<1->[width=0.2\linewidth]{redflower.png}
\includegraphics<2->[width=0.2\linewidth]{yellowflower.png}
\includegraphics<3->[width=0.2\linewidth]{blueflower.png}

\end{frame}

\end{document}
\end{lstlisting}

\newpage
\part{LaTeX进阶}
\setcounter{section}{0}
LaTeX作为一款文档排版系统,拥有众多文档类型,可用于制作科技论文、技术报告以及幻灯片等,除此之外,LaTeX还拥有包括编辑数学公式、制作图形与表格等功能,LaTeX用户可根据自身需要解锁LaTeX的各种用途。

添加程序源代码和算法伪代码对于科研报告往往是必要且有效的,因为代码可以展现计算机编程的思路和算法,可以供读者学习和使用。所以,能够学会添加简洁优美、整齐大方的源代码和伪代码是科研工作者的一项重要技能。对于科研工作者,在有些学术交流中,有一种非常重要的展现成果方式就是海报。LaTeX可以制作出优美简洁的海报,有很多模版可以方便制作者使用。另外,简历制作也是科研工作者经常需要用到的,同样LaTeX提供了很多好用的模版,可以方便其使用。

\section{添加程序源代码}
很多时候,在技术文档中添加程序源代码具有一定的必要性,这源于:
\begin{itemize}
    \item 在很多文档(如实验报告)中,程序源代码往往作为重要组成部分,必须作为辅助材料放在文档末尾的附录中。
    \item 程序源代码既可以直接展现计算机编程的实现过程和细节,又可以评估实验的真实性,同时也能供读者学习和使用。
\end{itemize}

事实上,使用LaTeX制作文档时,添加程序源代码是一件看似简单、但又比较考验技巧的事,
因为在文档中添加程序源代码并不能通过简单的“复制+粘贴”来实现。我们需要保持代码在原来程序语言中的格式,
包括代码所采用的高亮颜色和等宽字体,目的都是为了让代码本来的面貌得以完美展现。

在LaTeX中,有很多宏包可供制作文档时添加程序源代码到正文或附录中,
最常用的宏包包括listings和minted这两种,除此之外,还有一种插入程序源代码非常简便的一种方式,
即使用\textbackslash begin\{verbatim\} \textbackslash end\{verbatim\}环境。

\subsection{使用verbatim插入程序源代码}
在LaTeX中插入Python代码可以使用verbatim环境,即在\textbackslash begin\{verbatim\} \textbackslash end\{verbatim\}之间插入代码,代码的文本是等宽字体。
需要注意的是,这一环境不会对程序源代码进行高亮处理。

\newpage
\noindent
Python code example:

\begin{verbatim}
import numpy as np 
x=np.random.rand(4)
print(x)
\end{verbatim}

\begin{lstlisting}
使用verbatim环境插入如下Python代码:
\documentclass[12pt]{article}

\begin{document}

Python code example:
\begin{verbatim}
import numpy as np

x = np.random.rand(4)
print(x)
\end{verbatim}
\end{document}
\end{lstlisting}

\subsection{使用listings插入程序源代码}
如果想要对程序源代码进行高亮处理,可以使用专门排版代码的工具包listings,
除了在前导代码中申明使用listings工具包,即\textbackslash usepackage\{listings\},有时候还可以根据需要自定义一些参数。

\noindent
Python code example:
\begin{lstlisting}[language = python]
import numpy as np

x = np.random.rand(4)
print(x)
\end{lstlisting}

\newpage
\begin{lstlisting}
使用listings工具包插入Python代码,并自定义代码高亮。
\documentclass[12pt]{article}
\usepackage{listings}
\usepackage{color}
\definecolor{codegreen}{rgb}{0,0.6,0}
\definecolor{codegray}{rgb}{0.5,0.5,0.5}
\definecolor{codepurple}{rgb}{0.58,0,0.82}
\definecolor{backcolour}{rgb}{0.95,0.95,0.92}

\lstdefinestyle{mystyle}{
    backgroundcolor=\color{backcolour},   
    commentstyle=\color{codegreen},
    keywordstyle=\color{magenta},
    numberstyle=\tiny\color{codegray},
    stringstyle=\color{codepurple},
    basicstyle=\ttfamily\footnotesize,
    breakatwhitespace=false,         
    breaklines=true,                 
    captionpos=b,                    
    keepspaces=true,                 
    numbers=left,                    
    numbersep=5pt,                  
    showspaces=false,                
    showstringspaces=false,
    showtabs=false,                  
    tabsize=2
}

\lstset{style=mystyle}

\begin{document}

Python code example:

\begin{lstlisting}[language = python]
import numpy as np

x = np.random.rand(4)
print(x)

\end{lstlisting}

\section{算法伪代码}
算法这个词的英文是algorithm,它几乎贯穿了整个计算机的各个领域。
算法伪代码作为自然语言与类编程语言组成的混合结构,它在描述算法结构和思路方面要比纯编程语言更简洁且可读性更好、相比自然语言则更准确。同时,我们也能很容易地将算法伪代码转换成计算机程序。因此,在计算机相关的技术文档或文献中,适当使用算法伪代码解释技术架构会更方便读者理解。

在LaTeX中,为了便于创建算法伪代码,现有很多相关的宏包,
例如algorithm和algorithmic,在前导代码中申明使用这些宏包便可使用相应的算法伪代码环境。
宏包algorithm提供的算法伪代码环境为\textbackslash begin\{algorithm\} \textbackslash end\{algorithm\}和\\\textbackslash begin\{algorithmic\} \textbackslash end\{algorithmic\}。

\noindent
This is a simple example:
\begin{algorithm}
\renewcommand{\algorithmicrequire}{\textbf{Input:}}
\renewcommand{\algorithmicensure}{\textbf{Output:}}
\caption{Inner product of vectors}
\begin{algorithmic}[1]
\REQUIRE $\boldsymbol{x},\boldsymbol{y}\in\mathbb{R}^{n}$
\ENSURE $c$
\STATE $c=0$
\FOR{$i=1$ to $n$}
\STATE $c=c+x_iy_i$
\ENDFOR
\end{algorithmic}
\end{algorithm}

在例子中,语句\textbackslash renewcommand\{\textbackslash algorithmicrequire\}\{\textbackslash textbf\{Input:\}\}表示将算法伪代码中的关键词require替换成Input,同理,我们也能将关键词ensure替换成Output。

除了algorithm和algorithmic这两个专门用于创建算法伪代码的宏包,还有一个非常常用的宏包,叫algorithm2e,它与algorithm宏包创建出来的算法伪代码在样式上略有不同,algorithm2e也提供了一种\textbackslash begin\{algorithm\} \textbackslash end\{algorithm\}环境。

在例子中,申明使用宏包algorithm2e时将参数设置为linesnumbered和boxed,这两个参数分别表示对算法伪代码各行进行编号和对算法伪代码区域加边框,作为全局参数,会成为算法伪代码中的默认设置。
\begin{lstlisting}
使用algorithm宏包中相应的环境创建一个简单的算法伪代码。
\documentclass[12pt]{article}
\usepackage[linesnumbered, boxed]{algorithm2e}
\usepackage{amsmath, amsfonts}

\begin{document}

This is a simple example:

\IncMargin{1em}
\begin{algorithm}
\SetKwInOut{Input}{Input}
\SetKwInOut{Output}{Output}
\caption{Inner product of vectors}
\Input{$\boldsymbol{x},\boldsymbol{y}\in\mathbb{R}^{n}$}
\Output{$c$}
$c=0$\;
\For{$i=1$ \KwTo $n$}{
$c=c+x_iy_i$\;
}
\end{algorithm}

\end{document}
\end{lstlisting}

\section{简历制作}
LaTeX可制作各类文档,其中也不乏简历,与其他各类文档相比,简历在制作的过程中注重内容的简洁与清晰。使用LaTeX制作简历有诸多优势:第一,无需考虑字体、颜色、排版等问题;第二,LaTeX拥有众多简历模板可供选择,易于切换简历的排版风格。

\subsection{使用article文档类型制作简历}
article文档类型是LaTeX中极为常用的一种文档类型,使用article文档类型制作简历时,
可将文档的结构命令如\textbackslash section\{\}、\textbackslash subsection\{\}、\textbackslash subsubsection\{\}等格式稍作调整。

\subsection{自定义简历格式}
在\LaTeX 文件中编写documentclass{article}时,包括了类文件article.cls。
该类文件定义了组织文档的所有命令,比如片段和标题,它还配置这些命令如何影响页面的格式和布局。
使用\LaTeX 制作简历时,我们需要自定义文档格式。
其中最简洁的方法是将所有信息保存在个人类文件中,这样可以使文档的结构与格式完全分离,从而便于使用。
因此,我们需要创建一个简历的类文件,例如CV.cls,然后在类文件内定义文档格式。

所有类文件都应该以下面两行代码开头,应该将它们添加到CV.cls的顶部。
\begin{lstlisting}
\NeedsTeXFormat{LaTeX2e}
\ProvidesClass{CV}[2021/08/29 My custom CV class]
\end{lstlisting}

\textbackslash NeedsTeXFormat\{LaTeX2e\}命令告诉编译器该包适用于哪个版本的\LaTeX ,\LaTeX 的当前版本是\LaTeX2e 。
\textbackslash ProvidesClass\{CV\}[2021/08/29 My custom CV class]第一个参数应该匹配类文件的文件名,并告诉 LaTeX 包的名称。
第二个参数是可改变的,它提供了类的描述,这些描述将出现在日志和其他地方。

随后,我们创建一个编译文件CV.tex,并将以下代码键入文件,填写简历中的个人信息。
\begin{lstlisting}
\documentclass{CV}

\begin{document}

\section{Research Interests}
\subsection{Machine Learning}

\section{Education}
\subsection{University of Nowhere}

\end{document}
\end{lstlisting}

标准的文章标题并不适合简历,所以我们希望用更整洁的格式取代它们。
为此,我们可以在CV.cls文件中重新定义section命令以输出自定义格式。
在这里我们需要使用titlesec宏包,调用命令为\textbackslash RequirePackage\{titlesec\},
随后,我们便可以自定义标题格式。在CV.cls文件中键入以下代码:
\begin{lstlisting}
\RequirePackage{titlesec}

\titleformat{\section}         
    {\bfseries\Large\scshape\raggedright} 
    {}{0em}                      
    {}                           
    [\titlerule]   

\titleformat{\subsection}
    {\large\scshape\raggedright}
    {}{0em}
    {}   
\end{lstlisting}

自定义标题格式可以使用以下命令:
\begin{itemize}
    \item \textbackslash bfseries, \textbackslash itshape: 标题加粗或加斜体;
    \item \textbackslash scshape: 小型资本;
    \item \textbackslash small, \textbackslash  normalsize, \textbackslash large, \textbackslash Large, \textbackslash LARGE, \textbackslash huge, \textbackslash Huge: 设定字型大小;
    \item \textbackslash rmfamily, \textbackslash sffamily, \textbackslash ttfamily: 将字体类型分别设置为有衬线字体、无衬线字体或打字机字体。
\end{itemize}

简历的部分章节需要添加日期,我们在CV.cls文件中定义一个新命令\\ \textbackslash datedsubsection,命令可以让我们在子章节标题中添加日期,新命令代码为:
\begin{lstlisting}
\newcommand{\datedsubsection}[2]{%
\subsection[#1]{#1 \hfill #2}%
}    
\end{lstlisting}

键入以上代码后,在CV.tex文件中更改部分代码,使用新定义命令:
\begin{lstlisting}
\documentclass{CV}

\begin{document}

\section{Research Interests}
\subsection{Machine Learning}

\section{Education}
\datedsubsection{University of Nowhere}{2012---2016} %使用新定义命令

\end{document}
\end{lstlisting}

在简历中,名字通常在最上面,并且包含相关的联系方式,同样地我们在CV.cls文件中定义一个新命令\textbackslash name来添加名字,定义另一个新命令\textbackslash contact来添加联系方式。
\begin{lstlisting}
\newcommand{\name}[1]{%
\centerline{\Huge{#1}}
}

\newcommand\contact[5]{%
    \centerline{%
        \makebox[0pt][c]{%
            #1 {\large\textperiodcentered} #2 {\large\textperiodcentered} #3
            \ #4 \ \ #5%
        }%
    }%
}    
\end{lstlisting}

键入以上代码后,在CV.tex文件中使用新定义命令\textbackslash name和\textbackslash contact:
\begin{lstlisting}
\documentclass{CV}

\begin{document}

\name{John Kim}

\contact{123 Broadway}{London}{UK 12345}{john@kim.com}{(000)-111-1111}

\section{Research Interests}
\subsection{Machine Learning}
My research interest is machine learning. 

\section{Education}
\datedsubsection{University of Nowhere}{2012---2016}
I attended the University of Nowhere from 2012 to 2016.

\end{document}
\end{lstlisting}

\begin{lstlisting}
当然也可以自定义一些列表:

\newcommand{\researchitems}[3]{%
    \begin{itemize}
    \item #1
    \item #2
    \item #3
    \end{itemize}%
}
键入以上代码后,在CV.tex文件中使用新定义命令\researchitems:

\documentclass{CV}

\begin{document}

\name{John Kim}

\contact{123 Broadway}{London}{UK 12345}{john@kim.com}{(000)-111-1111}

\section{Research Interests}
\subsection{Machine Learning}
My research interest is machine learning.
\researchitems
{Logistic regression}
{Neural Networks}
{SVM}

\section{Education}
\datedsubsection{University of Nowhere}{2012---2016}
I attended the University of Nowhere from 2012 to 2016.

\end{document}
\end{lstlisting}
\end{document}

